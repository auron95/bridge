\documentclass[a4paper,10pt]{article}


\usepackage[utf8]{inputenc}
\usepackage[italian]{babel}
\usepackage[pdftex]{graphicx}
\usepackage{amsmath}
\usepackage{amsthm}
\usepackage{fancyhdr}
\usepackage{amsfonts}
\usepackage{amssymb}
\usepackage{xspace}
\usepackage{color}
% \usepackage[parfill]{parskip}
\usepackage{hyperref}


\topmargin -1cm
\oddsidemargin -0.5cm
\textwidth 17cm


\renewcommand{\c}{$\clubsuit$\xspace}
\renewcommand{\d}{$\diamondsuit$\xspace}
\newcommand{\h}{$\heartsuit$\xspace}
\newcommand{\s}{$\spadesuit$\xspace}
\renewcommand{\j}{$\bigstar$\xspace}
\newcommand{\sa}{SA\xspace}


\newcommand{\smallspace}{\vskip0.3cm}

\renewcommand{\tabcolsep}{0.3cm}

\newcommand{\note}[1]{\textcolor{red}{#1}}


\newenvironment{twocol}
  {\smallspace\noindent\begin{tabular}{l p{0.78\textwidth}}}
  {\end{tabular}\smallspace}

\newenvironment{twocolind}
  {\smallspace\noindent\begin{tabular}{l p{0.68\textwidth}}}
  {\end{tabular}\smallspace}

\newenvironment{threecol}
  {\smallspace\noindent\begin{tabular}{l l p{0.78\textwidth}}}
  {\end{tabular}\smallspace}

  
% Title Page
\title{Convenzioni per il corso di bridge}
\author{Ugo Bindini \& Giovanni Paolini}
\date{Autunno 2015}

\begin{document}
\maketitle

\section{Aperture}

\paragraph{Algoritmo generale.}
Si licita il colore più lungo, escludendo \s e \h se non sono almeno quinte e le \d se non sono almeno quarte.
In caso di parità di lunghezza dei semi più lunghi, si dichiara il più alto se sono almeno quinti e si dichiara il più basso se sono quarti.
Nota: con 4\s\ - 4\h\ - 3\d\ - 2\c, si apre di 1 \c.

\paragraph{Mani bilanciate.} Distribuzioni: 4-3-3-3, 4-4-3-2, 5-3-3-2 con quinta minore.

\begin{threecol}
 12-14 HCP & 1 a colore & Si licita seguendo l'algoritmo generale.\\
 15-17 HCP & 1 \sa\\
 18-19 HCP & 1 \c & Poi \sa a salto sulla risposta a livello di 1 (es: 1 \c\ - 1 \s\ - \mbox{2 \sa}; 1 \c\ - 1 \sa\ - 3 \sa).\\
 20-22 HCP & 2 \sa\\
 23+ HCP & 2 a colore & Si licita seguendo l'algoritmo generale.
\end{threecol}

\paragraph{Mani non bilanciate.} Si apre a livello di 1 con 12-20 HCP, a livello di 2 con 21+ HCP, licitando secondo l'algoritmo generale.

\paragraph{Sottoaperture.} Si può aprire a livello di 3 con colore almeno settimo con 6-10 HCP, meglio se a compagno passato.


\section{Risposte sull'apertura di 1 a colore}

Si risponde con 5+ HCP oppure con un Asso.

\paragraph{Senza fit.} Se possibile, dire un colore almeno quarto a livello di 1 (il più economico). Per salire a livello di 2 in un colore mai licitato bisogna avere almeno 11 HCP.

\paragraph{Con fit su un colore licitato dal compagno.} Ripetere il colore a livello dà mano minima, a salto invita a manche. Licitare la manche è passabile, con ragionevole certezza di manche anche se il compagno ha mano minima. 4 \sa (richiesta d'Assi) mostra interesse di slam.


\section{Risposte sull'apertura di 1 SA}

Se si licita un colore (eccetto \c, vedi oltre) è naturale (almeno quinto), alla ricerca di un fit. 2 \sa è invitante manche, 3 \sa da passare, 4 \sa richiesta d'Assi.

\paragraph{Richiesta di quarte nobili (Stayman).} Con 8+ HCP il rispondente può licitare 2 \c, cercando fit in un seme nobile. L'apertore licita come segue.

\begin{twocol}
 2 \d & Nessuna quarta nobile.\\
 2 \h & \h quarte, ma non \s quarte.\\
 2 \s & \s quarte, ma non \h quarte.\\
 2 SA & Entrambe le quarte nobili.
\end{twocol}


\section{Rever}

Dopo l'apertura di 1 \j (12-20 HCP) e una risposta a livello di 1, con la propria seconda dichiarazione l'apertore deve comunicare al compagno se ha 12-15 HCP o 16-20 HCP (rever). Per dare rever occorre superare strettamente 2 \j (colore di apertura). Fa eccezione il caso in cui l'apertore voglia dare fit al rispondente: in questo caso licitare il colore del rispondente a livello dà 12-15 HCP, a salto dà rever. Ecco alcuni esempi:

\begin{twocol}
1 \d\ - 1 \s\ - 2 \h & 16-20 HCP, con almeno 6 \d e 5 \h (perché?), e ovviamente non 4 \s. Con la stessa distribuzione ma 12-15 HCP, l'apertore licita invece 2 \d, tenendosi le \h per sé.\\
1 \s\ - 1 \sa\ - 2 \h & 12-15 HCP, con almeno 5 \s e 4 \h.\\
1 \d\ - 1 \h\ - 3 \h & 16-20 HCP, con almeno 4 \d e 4 \h (si giocherà a \h).\\
1 \h\ - 1 \s\ - 2 \s & 12-15 HCP, con almeno 5 \h e 4 \s (si giocherà a \s).
\end{twocol}


\section{Richieste d'Assi e di Re}

\paragraph{Richiesta d'Assi (4 \sa).} Risposte:

\begin{twocol}
5 \c & 0 o 4 Assi.\\
5 \d & 1 Assi.\\
5 \h & 2 Assi.\\
5 \s & 3 Assi.
\end{twocol}

\paragraph{Richiesta di Re (5 \sa).} Risposte:

\begin{twocol}
6 \c & 0 o 4 Re.\\
6 \d & 1 Re.\\
6 \h & 2 Re.\\
6 \s & 3 Re.
\end{twocol}

Dopo aver ricevuto le risposte, il richiedente sceglierà il contratto da giocare. 


\section{Interventi}

\begin{twocol}
  X & 12+ HCP, un po' di carte nei pali non dichiarati dagli avversari (soprattutto nei pali nobili). In alternativa, mano forte (16+ HCP) e distribuzione qualsiasi.\\
  1 \j & 10+ HCP, \j almeno quinto.\\
  1 \sa & Come apertura, con fermo nel palo degli avversari.\\
  2 \j a livello & 12+ HCP, \j almeno quinto.\\
  3 \j & Come apertura.
\end{twocol}

\end{document}
