\documentclass[a4paper,10pt]{article}


\usepackage[utf8]{inputenc}
\usepackage[italian]{babel}
\usepackage[pdftex]{graphicx}
\usepackage{amsmath}
\usepackage{amsthm}
\usepackage{fancyhdr}
\usepackage{amsfonts}
\usepackage{amssymb}
\usepackage{xspace}
\usepackage{color}
% \usepackage[parfill]{parskip}
\usepackage{hyperref}


\topmargin -1cm
\oddsidemargin -0.5cm
\textwidth 17cm


\renewcommand{\c}{$\clubsuit$\xspace}
\renewcommand{\d}{$\diamondsuit$\xspace}
\newcommand{\h}{$\heartsuit$\xspace}
\newcommand{\s}{$\spadesuit$\xspace}
\renewcommand{\j}{$\bigstar$\xspace}
\newcommand{\sa}{SA\xspace}


\newcommand{\smallspace}{\vskip0.3cm}

\renewcommand{\tabcolsep}{0.3cm}

\newcommand{\note}[1]{\textcolor{red}{#1}}


\newenvironment{twocol}
  {\smallspace\noindent\begin{tabular}{l p{0.78\textwidth}}}
  {\end{tabular}\smallspace}

\newenvironment{twocolind}
  {\smallspace\noindent\begin{tabular}{l p{0.68\textwidth}}}
  {\end{tabular}\smallspace}

\newenvironment{threecol}
  {\smallspace\noindent\begin{tabular}{l l p{0.78\textwidth}}}
  {\end{tabular}\smallspace}

  
% Title Page
\title{Convenzioni per il corso di bridge}
\author{Ugo Bindini, Alice Cortinovis e Giovanni Paolini}
\date{Autunno 2017}

\begin{document}
\maketitle

\section{Aperture}

\paragraph{Algoritmo generale.}
Si licita il colore più lungo, escludendo \s e \h se non sono almeno quinte e le \d se non sono almeno quarte.
In caso di parità di lunghezza dei semi più lunghi, si dichiara il più alto se sono almeno quinti e si dichiara il più basso se sono quarti.
Nota: con 4\s\ - 4\h\ - 3\d\ - 2\c, si apre di 1 \c.

\paragraph{Mani bilanciate.} Distribuzioni: 4-3-3-3, 4-4-3-2, 5-3-3-2 con quinta minore.

\begin{threecol}
 12-14 HCP & 1 a colore & Si licita seguendo l'algoritmo generale.\\
 15-17 HCP & 1 \sa\\
 18-19 HCP & 1 \c & Poi \sa a salto (ad esempio: 1 \c\ - 1 \s\ - 2 \sa, 1 \c\ - 1 \sa\ - 3 \sa).\\
 20-22 HCP & 2 \sa\\
 23+ HCP & 2 a colore & Si licita seguendo l'algoritmo generale.
\end{threecol}

\paragraph{Mani non bilanciate.} Si apre a livello di 1 con 12-20 HCP, a livello di 2 con 21+ HCP, licitando secondo l'algoritmo generale.

% \paragraph{Sottoaperture (barrage).} Si può aprire a livello di 3 con colore almeno settimo e 6-10 HCP, meglio se a compagno passato (barrage).

\section{Risposte sull'apertura di 1 a colore}

Si deve rispondere con 5+ HCP oppure con un Asso.

\paragraph{Senza fit.} Se possibile, dire il colore più lungo (almeno quarto) a livello di 1 (a parità di lunghezza, il più economico). Per salire a livello di 2 bisogna avere almeno 12 HCP (a meno che non sia per dare il fit). Se non si è in nessuno dei due casi si licita 1 \sa.% Se si hanno due colori quinti a livello di 1, dire il più alto.

\paragraph{Con fit su un colore \j licitato dal compagno.} Se il colore è \c o \d, ha comunque la priorità licitare un proprio seme almeno quarto a livello di 1 (c'è tempo per dare il fit successivamente). Per dare fit direttamente
\begin{twocol}
 2 \j & Mano minima (5-8 HCP).\\
 3 \j & Invitante a manche (9-11 HCP).\\
 4 \j & Garantisce la manche (12-15 HCP).\\
 4 \sa & Interesse di slam (16+ HCP).
\end{twocol}

\section{Risposte sull'apertura di 1 \sa}

\paragraph{Per giocare a \sa.} Il rispondente può licitare

\begin{twocol}
 PASS & \\
 2 \sa & Invitante (8-9 HCP).\\
 3 \sa & A giocare (10-14 HCP).\\
 4 \sa & Richiesta d'Assi (15+ HCP), interesse di slam.
\end{twocol}


\paragraph{Transfer.} Il rispondente può licitare

\begin{twocol}
 2 \d & Obbliga l'apertore a licitare 2 \h.\\
 2 \h & Obbliga l'apertore a licitare 2 \s.\\
 2 \s & Obbliga l'apertore a licitare 3 \c.\\
 3 \c & Obbliga l'apertore a licitare 3 \d.\\
\end{twocol}

In questo modo si seleziona l'atout; il rispondente poi guida scegliendo il livello del contratto (pass, invito a manche, manche, richiesta d'assi).

\paragraph{Richiesta di quarte nobili (Stayman).} Con 8+ HCP il rispondente può licitare 2 \c, cercando fit in un seme nobile. L'apertore licita come segue.

\begin{twocol}
 2 \d & Nessuna quarta nobile.\\
 2 \h & \h quarte, ma non \s quarte.\\
 2 \s & \s quarte, ma non \h quarte.\\
 2 SA & Entrambe le quarte nobili.
\end{twocol}

Il rispondente ora conosce talmente bene la mano dell'apertore da poter selezionare il contratto migliore.

\section{Risposte sull'apertura di 2 \sa}

\paragraph{Per giocare a \sa.} Il rispondente può licitare

\begin{twocol}
 PASS & \\
 3 \sa & A giocare (5-9 HCP).\\
 4 \sa & Richiesta d'Assi (10+ HCP), interesse di slam.
\end{twocol}

\paragraph{Transfer.} Il rispondente può licitare

\begin{twocol}
 3 \d & Obbliga l'apertore a licitare 3 \h.\\
 3 \h & Obbliga l'apertore a licitare 3 \s.\\
 3 \s & Obbliga l'apertore a licitare 4 \c.\\
 4 \c & Obbliga l'apertore a licitare 4 \d.\\
\end{twocol}

In questo modo si seleziona l'atout; il rispondente poi guida scegliendo il livello del contratto. Le ultime due possibilità vanno usate con cautela, poiché il livello sale molto e, soprattutto, vi prendete la responsabilità di saltare 3 \sa!

\paragraph{Richiesta di quarte nobili (Stayman).} Con 4/5+ HCP (manche garantita, almeno a \sa) il rispondente può licitare 3 \c, cercando fit in un seme nobile. L'apertore licita come segue.

\begin{twocol}
 3 \d & Nessuna quarta nobile.\\
 3 \h & \h quarte, ma non \s quarte.\\
 3 \s & \s quarte, ma non \h quarte.\\
 3 SA & Entrambe le quarte nobili.
\end{twocol}

Il rispondente ora conosce talmente bene la mano dell'apertore da poter selezionare il contratto migliore.

\section{Rever}

Dopo l'apertura di 1 \j (12-20 HCP) e una risposta a livello di 1, con la propria seconda dichiarazione l'apertore deve comunicare al compagno se ha 12-15 HCP o 16-20 HCP (rever). Per dare rever occorre superare strettamente 2 \j (colore di apertura). Fa eccezione il caso in cui l'apertore voglia dare fit al rispondente: in questo caso licitare il colore del rispondente a livello dà 12-15 HCP, a salto dà rever. Ecco alcuni esempi:

\begin{twocol}
1 \s\ - 1 \sa\ - 2 \h & 12-15 HCP, con almeno 5 \s e 4 \h.\\
1 \d\ - 1 \h\ - 3 \h & 16-18 HCP, con almeno 4 \d e 4 \h (si giocherà a \h). (Nota: con 19-20 HCP l'apertore deve licitare 4\h. Perché?)\\
1 \h\ - 1 \s\ - 2 \s & 12-15 HCP, con almeno 5 \h e 4 \s (si giocherà a \s).\\
1 \d\ - 1 \s\ - 2 \h & 16-20 HCP, con almeno 5 \d e 4 \h, e ovviamente non 4 \s. Con la stessa distribuzione ma 12-15 HCP, l'apertore licita invece 2 \d, tenendosi le \h per sé.
\end{twocol}


\section{Richieste d'Assi e di Re}

\paragraph{Richiesta d'Assi (4 \sa).} Risposte:

\begin{twocol}
5 \c & 0 o 4 Assi.\\
5 \d & 1 Assi.\\
5 \h & 2 Assi.\\
5 \s & 3 Assi.
\end{twocol}

\paragraph{Richiesta di Re (5 \sa).} Risposte:

\begin{twocol}
6 \c & 0 o 4 Re.\\
6 \d & 1 Re.\\
6 \h & 2 Re.\\
6 \s & 3 Re.
\end{twocol}

Dopo aver ricevuto le risposte, il richiedente sceglierà il contratto da giocare.

% \section{Interventi}
% 
% \begin{twocol}
%   X (contre) & 12+ HCP, al massimo due carte nei semi licitati dagli avversari, almeno 4 carte in un seme nobile. In alternativa, mano forte (rever) con qualsiasi distribuzione.\\
%   1 \j & 10+ HCP, \j almeno quinto.\\
%   1 \sa & Come apertura (15-17 HCP, mano bilanciata), con fermo nel seme degli avversari.\\
%   2 \j a livello & 12+ HCP, \j almeno quinto.\\
%   2 \sa & Come apertura (20-22 HCP, mano bilanciata), con fermo nel seme degli avversari.\\
%   2 \j a salto & 6-10 HCP, \j almeno sesto (mano sbilanciata).\\
%   3 \j & 6-10 HCP, \j almeno settimo (mano sbilanciata).
% \end{twocol}
% 
% \pagebreak
% 
\section{Principi generali e consigli}

\indent

Non codifichiamo un modo per dare rever dopo una risposta a livello di 2. D'altronde, poiché entrambi già sanno che si hanno almeno 24 HCP in linea, non ci si ferma prima di una manche, e non ci sono problemi a salire con il livello per continuare la descrizione della mano.

\smallspace

La licita di un nuovo colore da parte del rispondente è sempre forzante almeno un giro (i.e. l'apertore non può passare).

\smallspace

Ripetere un proprio colore in cui non si è ancora trovato un fit allunga il colore di una carta. Eccezione: dopo l'apertura di 1 \c e una risposta a livello di 1, licitare nuovamente le \c promette almeno 5 carte di \c.
% 
% \smallspace
% 
% Non abbiate paura ad usare il Contre! Molto più spesso di quanto pensiate è la cosa giusta da fare. Contre su una licita a livello di 1 o di 2 è forzante, a meno di casi molto eccezionali (es: 1 \c\ - X - Pass e il quarto giocatore ha 7/8 carte di fiori, o simili).
% 
\smallspace

Giocando ad atout è meglio avere il fit, ma non è indispensabile: talvolta può essere meglio giocare un contratto a livello di 2 con 7 carte in linea, piuttosto che intestardirsi nella ricerca di un fit e salire troppo con la dichiarazione.


\end{document}
