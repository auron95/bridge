\documentclass[a4paper,10pt]{article}


\usepackage[utf8]{inputenc}
\usepackage[italian]{babel}
\usepackage[pdftex]{graphicx}
\usepackage{amsmath}
\usepackage{amsthm}
\usepackage{fancyhdr}
\usepackage{amsfonts}
\usepackage{amssymb}
\usepackage{xspace}
\usepackage{color}
% \usepackage[parfill]{parskip}
\usepackage{hyperref}


\topmargin -0.5cm
\oddsidemargin -0.5cm
\textwidth 17cm


\renewcommand{\c}{$\clubsuit$\xspace}
\renewcommand{\d}{$\diamondsuit$\xspace}
\newcommand{\h}{$\heartsuit$\xspace}
\newcommand{\s}{$\spadesuit$\xspace}
\renewcommand{\j}{$\bigstar$\xspace}
\newcommand{\sa}{SA\xspace}


\newcommand{\smallspace}{\vskip0.3cm}

\renewcommand{\tabcolsep}{0.3cm}

\newcommand{\note}[1]{\textcolor{red}{#1}}


\newenvironment{twocol}
  {\smallspace\noindent\begin{tabular}{l p{0.78\textwidth}}}
  {\end{tabular}\smallspace}

\newenvironment{twocolind}
  {\smallspace\noindent\begin{tabular}{l p{0.68\textwidth}}}
  {\end{tabular}\smallspace}

\newenvironment{threecol}
  {\smallspace\noindent\begin{tabular}{l l p{0.78\textwidth}}}
  {\end{tabular}\smallspace}

  
% Title Page
\title{Convenzioni per il corso di bridge}
\author{Ugo Bindini \and Giovanni Italiano \and Matteo Migliorini \and Giovanni Paolini}
\date{Autunno 2018}

\begin{document}
\maketitle

\section{Aperture}

\paragraph{Algoritmo generale.}
Si licita il colore più lungo, escludendo \s e \h se non sono almeno quinte e \d se non sono almeno quarte.
In caso di parità di lunghezza dei semi più lunghi, si dichiara il più alto se sono almeno quinti e si dichiara il più basso se sono quarti.
Nota: di conseguenza, con 4\s\ - 4\h\ - 3\d\ - 2\c, si apre di 1 \c.

\paragraph{Mani bilanciate.} Distribuzioni: 4-3-3-3, 4-4-3-2, 5-3-3-2 con quinta minore.

\begin{threecol}
 12-14 HCP & 1 a colore & Si licita seguendo l'algoritmo generale.\\
 15-17 HCP & 1 \sa\\
 18-19 HCP & 1 \c & Poi \sa a salto sulla risposta a livello di 1 (es: 1 \c\ - 1 \s\ - 2 \sa; 1 \c\ - 1 \sa\ - 3 \sa).\\
 20-21 HCP & 2 \sa\\
 22+ HCP & 2 \c & Poi \sa a livello.
\end{threecol}

\paragraph{Mani non bilanciate.} Si apre a livello di 1 con 12-20 HCP, licitando secondo l'algoritmo generale. Si aprono di 2 \c tutte le mani con 21+ HCP o forzanti manche.

\paragraph{Sottoaperture.} Si apre a livello di 2 con colore almeno sesto (non \c) e 6-10 HCP. Si può aprire a livello di 3 con colore almeno settimo e 6-10 HCP, meglio se a compagno passato (barrage).


\section{Risposte sull'apertura di 1 a colore}

Si risponde con 5+ HCP oppure con un Asso.

\paragraph{Senza fit.} Se possibile, dire un colore almeno quarto a livello di 1 (il più economico). Per salire a livello di 2 bisogna avere almeno 12 HCP (a meno che non sia per dare il fit).

\paragraph{Con fit su un colore licitato dal compagno.} Se il colore è \c o \d, ha comunque la priorità licitare un proprio seme almeno quarto a livello di 1 (c'è tempo per dare il fit successivamente).

Per dare il fit in un seme nobile \j, si licita come segue.

\begin{twocol}
	2 \j & Mano minima, 5-9 HCP.\\
	3 \j & Invito a manche, 10-11 HCP.\\
	4 \j & Signoff senza visuale di slam, 12-16 HCP.\\
	4 \sa & Richiesta d'Assi (interesse di slam), 17+ HCP.
\end{twocol}

\pagebreak

\section{Risposte sull'apertura di 1 \sa}

\paragraph{Transfer.} Con un colore almeno sesto (certezza di fit), il rispondente può licitare

\begin{twocol}
 2 \d & Obbliga l'apertore a licitare 2 \h.\\
 2 \h & Obbliga l'apertore a licitare 2 \s.\\
 2 \s & Obbliga l'apertore a licitare 3 \c.\\
 3 \c & Obbliga l'apertore a licitare 3 \d.\\
\end{twocol}

In questo modo si seleziona l'atout; il rispondente poi guida scegliendo il livello del contratto.

\paragraph{Per giocare a \sa.} Il rispondente può licitare

\begin{twocol}
	Pass & \\
	2 \sa & Invitante a 3 \sa.\\
	3 \sa & A giocare.\\
	4 \c & Richiesta d'Assi.
\end{twocol}


\paragraph{Richiesta di quarte nobili (Stayman).} Con 7/8+ HCP il rispondente può licitare 2 \c, cercando fit in un seme nobile. L'apertore licita come segue.

\begin{twocol}
 2 \d & Nessuna quarta nobile.\\
 2 \h & \h quarte, ma non \s quarte.\\
 2 \s & \s quarte, ma non \h quarte.\\
 2 SA & Entrambe le quarte nobili.
\end{twocol}

\section{Risposte sull'apertura di 2 \sa}

\paragraph{Transfer.} Con un colore almeno sesto (certezza di fit), il rispondente può licitare

\begin{twocol}
 3 \d & Obbliga l'apertore a licitare 3 \h.\\
 3 \h & Obbliga l'apertore a licitare 3 \s.\\
 3 \s & Obbliga l'apertore a licitare 4 \c.\\
 4 \c & Obbliga l'apertore a licitare 4 \d.\\
\end{twocol}

In questo modo si seleziona l'atout; il rispondente poi guida scegliendo il livello del contratto. Le ultime due possibilità vanno usate con cautela, poiché il livello sale molto e, soprattutto, vi prendete la responsabilità di saltare 3 \sa!

\paragraph{Per giocare a \sa.} Il rispondente può licitare

\begin{twocol}
	Pass & \\
	3 \sa & A giocare.\\
	4 \c & Richiesta d'Assi.
\end{twocol}

\paragraph{Richiesta di quarte nobili (Stayman).} Con 4/5+ HCP (manche garantita, almeno a \sa) il rispondente può licitare 3 \c, cercando fit in un seme nobile. L'apertore licita come segue.

\begin{twocol}
 3 \d & Nessuna quarta nobile.\\
 3 \h & \h quarte, ma non \s quarte.\\
 3 \s & \s quarte, ma non \h quarte.\\
 3 SA & Entrambe le quarte nobili.
\end{twocol}


\section{Rever}

Dopo l'apertura di 1 \j (12-20 HCP) e una risposta a livello di 1, con la propria seconda dichiarazione l'apertore deve comunicare al compagno se ha 12-15 HCP o 16-20 HCP (\emph{rever}). Per dare rever occorre superare strettamente 2 \j (colore di apertura). Fa eccezione il caso in cui l'apertore voglia dare fit al rispondente: in questo caso licitare il colore del rispondente a livello dà 12-15 HCP, a salto dà rever. Ecco alcuni esempi:

\begin{twocol}
1 \d\ - 1 \s\ - 2 \h & 16-20 HCP, con almeno 6 \d e 5 \h (perché?), e ovviamente non 4 \s. Con la stessa distribuzione ma 12-15 HCP, l'apertore licita invece 2 \d, tenendosi le \h per sé.\\
1 \s\ - 1 \sa\ - 2 \h & 12-15 HCP, con almeno 5 \s e 4 \h.\\
1 \d\ - 1 \h\ - 3 \h & 16-20 HCP, con almeno 4 \d e 4 \h (si giocherà a \h).\\
1 \h\ - 1 \s\ - 2 \s & 12-15 HCP, con almeno 5 \h e 4 \s (si giocherà a \s).
\end{twocol}

\section{Richieste d'Assi e di Re}

\paragraph{Richiesta d'Assi ad atout (4 \sa Blackwood).} Ai fini della richiesta d'Assi, il Re di atout è considerato un Asso. Risposte:

\begin{twocol}
5 \c & 0 o 3 Assi.\\
5 \d & 1 o 4 Assi.\\
5 \h & 2 o 5 Assi senza Q di atout.\\
5 \s & 2 o 5 Assi con Q di atout.
\end{twocol}

Dopo aver effettuato la richiesta d'Assi, si può proseguire con la richiesta di Re (5 \sa). Risposte:

\begin{twocol}
6 \c & 0 Re.\\
6 \d & 1 Re.\\
6 \h & 2 Re.\\
6 \s & 3 Re.
\end{twocol}

\paragraph{Richiesta d'Assi a \sa (4 \c Gerber).} Risposte:

\begin{twocol}
	4 \d & 0 o 4 Assi.\\
	4 \h & 1 Asso.\\
	4 \s & 2 Assi.\\
	4 \sa & 3 Assi.
\end{twocol}

Dopo aver effettuato la richiesta d'Assi, si può proseguire con la richiesta di Re (5 \c). Risposte:

\begin{twocol}
	5 \d & 0 o 4 Re.\\
	5 \h & 1 Re.\\
	5 \s & 2 Re.\\
	5 \sa & 3 Re.
\end{twocol}

Dopo aver ricevuto le risposte, il richiedente sceglierà il contratto da giocare, che potrebbe anche essere 4 \sa o 5 \sa.


\section{Interventi}

\begin{twocol}
	Contre & 12+ HCP, un po' di carte nei semi non dichiarati dagli avversari (con preferenza per i nobili). In alternativa, mano forte (rever) con qualsiasi distribuzione.\\
	1 \j & 10-15 HCP, \j almeno quinto.\\
	1 \sa & Come apertura (15-17 HCP, mano bilanciata), con fermo nel seme degli avversari.\\
	2 \j a livello & 12-15 HCP, \j almeno quinto.\\
	2 \sa & Come apertura (20-22 HCP, mano bilanciata), con fermo nel seme degli avversari.\\
	2 \j a salto & 6-10 HCP, \j almeno sesto. \\
	3 \j & 6-10 HCP, \j almeno settimo (mano sbilanciata, barrage).
\end{twocol}

\section{Risposte su barrage a livello di 2}

Dopo una sottoapertura a livello di 2 o un intervento in barrage a livello di 2, il rispondente con 14+ HCP può licitare 2 \sa (interrogativo). Risposte:
\begin{twocol}
	3 \c & Mano minima, seme brutto.\\
	3 \d & Mano minima, seme buono.\\
	3 \h & Mano massima, seme brutto.\\
	3 \s & Mano massima, seme buono.\\
	3 SA & Seme ottimo (AKQ).
\end{twocol}


\section{Principi generali e consigli}

\indent

Non abbiate paura ad usare il Contre! Molto più spesso di quanto pensiate è la cosa giusta da fare. Contre su una licita a livello di 1 o di 2 è forzante, a meno di casi molto eccezionali (es: 1 \c\ - X - Pass e il quarto giocatore ha 7/8 carte di fiori, o simili).

\smallspace

Non codifichiamo un modo per dare rever dopo una risposta a livello di 2. D'altronde, poiché entrambi già sanno che si hanno almeno 24 HCP in linea, è improbabile che ci si fermi prima di una manche, e non ci sono problemi a salire con il livello per continuare la descrizione della mano.

\smallspace

La licita di un nuovo colore da parte del rispondente è sempre forzante almeno un giro (\emph{i.e.}, l'apertore non può passare).

\smallspace

Ripetere un proprio colore in cui non si è ancora trovato un fit allunga il colore di una carta. Eccezione: dopo l'apertura di 1 \c e una risposta a livello di 1, licitare nuovamente le \c promette almeno 5 carte di \c.

\smallspace

Giocando ad atout è meglio avere il fit, ma non è indispensabile: talvolta può essere meglio giocare un contratto a livello di 2 con 7 carte in linea, piuttosto che intestardirsi nella ricerca di un fit e salire troppo con la dichiarazione.

\end{document}
