\documentclass[a4paper,10pt]{article}


\usepackage[utf8x]{inputenc}
\usepackage[italian]{babel}
\usepackage[pdftex]{graphicx}
\usepackage{amsfonts}
\usepackage{amsmath}
\usepackage{amssymb}
\usepackage{amsthm}
\usepackage[mathscr]{urwchancal}
\usepackage{xcolor}
\usepackage{enumerate}
\usepackage{fancyhdr}
\usepackage[colorlinks=true, linkcolor=blue, urlcolor=blue, citecolor=blue]{hyperref}
\usepackage{xspace}
\usepackage[parfill]{parskip}
\usepackage[lmargin=4.5cm, marginparwidth=3.5cm, marginparsep=0.5cm]{geometry}
\usepackage{marginnote}
\usepackage{array}
\usepackage{skak}
\usepackage{needspace}
\usepackage{epigraph}
\usepackage{tabularx}

\DeclareMathAlphabet{\mathpzc}{OT1}{pzc}{m}{it}

\topmargin -1cm
% \oddsidemargin -0.5cm
\textwidth 14.5cm

\reversemarginpar

\setlength{\parindent}{0 pt} % Default 15 pt.
\setlength{\parskip}{0.15 cm} % Default 0 cm?

\renewcommand{\c}{$\clubsuit$\xspace}
\renewcommand{\d}{$\diamondsuit$\xspace}
\newcommand{\h}{$\heartsuit$\xspace}
\newcommand{\s}{$\spadesuit$\xspace}
\renewcommand{\j}{$\bigstar$\xspace}
\newcommand{\rj}{$\blacksquare$\xspace}
\newcommand{\sa}{SA\xspace}
\newcommand{\M}{\mbox{\raisebox{-1.2pt}{$^\heartsuit\mkern-6mu$} \raisebox{1.2pt}{$\mkern-6mu_\spadesuit$}\xspace}}%{$\mathpzc{M}$\xspace}
\newcommand{\m}{\mbox{\raisebox{-1.2pt}{$^\clubsuit \mkern-4.5mu$} \raisebox{1.2pt}{$\mkern-4.5mu_\diamondsuit$}}\xspace}%{$\mathpzc{m}$\xspace}

\newcommand{\cfbox}[2]{%
  \colorlet{currentcolor}{.}%
  {\color{#1}%
    \fbox{\color{currentcolor}#2}}%
}

\newcommand{\alert}[1]{\cfbox{red}{#1}}

\newcommand{\smallspace}{\vskip0.3cm}

\renewcommand{\tabcolsep}{0.3cm}

\newcommand{\note}[1]{\textcolor{red}{#1}}


\newenvironment{twocol}
{\smallspace\noindent\tabularx{\linewidth}{ l X }}%p{0.78\textwidth}}}
{\endtabularx\smallspace}


\newenvironment{threecol}
{\smallspace\noindent\tabularx{\textwidth}{l l X}}
{\endtabularx\smallspace}

\newcommand{\biddingtable}[2][0.4cm]{
  \needspace{1cm}
  \marginnote{
    \scriptsize{
      \def\arraystretch{1.5}
      \renewcommand{\tabcolsep}{0.1cm}
      \begin{tabular}{|>{\centering\arraybackslash}p{0.6cm}>{\centering\arraybackslash}p{0.6cm}>{\centering\arraybackslash}p{0.6cm}>{\centering\arraybackslash}p{0.6cm}|}
        \hline
        #2
      \end{tabular}
    }
  }[#1]
}

\newcommand{\biddingtablesec}[1]{\biddingtable[-0.65cm]{#1}}
% \newcommand{\biddingtablepar}[1]{\biddingtable{#1}{0cm}}


% Title Page
\title{Convenzioni per il corso di bridge}
\author{Ugo Bindini \and Giovanni Italiano \and Matteo Migliorini}
\date{Autunno 2019}

\begin{document}
\maketitle


\section{Aperture}

\paragraph{Algoritmo generale.}
Si licita il colore più lungo, escludendo \s e \h se non sono almeno quinte e le \d se non sono almeno quarte.
In caso di parità di lunghezza dei semi più lunghi, si dichiara il più alto se sono almeno quinti e si dichiara il più basso se sono quarti.
NB: conseguentemente, con 4\s\ - 4\h\ - 3\d\ - 2\c, si apre di 1 \c.

\paragraph{Mani bilanciate.} Distribuzioni: 4-3-3-3, 4-4-3-2, 5-3-3-2 con quinta minore.

\begin{threecol}
 12-14 HCP & 1 a colore & Si licita seguendo l'algoritmo generale.\\
 15-17 HCP & 1 \sa\\
 18-19 HCP & 1 \c & Poi \sa a salto (ad esempio: 1 \c\ - 1 \s\ - 2 \sa, 1 \c\ - 1 \sa\ - 3 \sa).\\
 20-21 HCP & 2 \sa\\
 22+ HCP & 2 \c & Poi 2 \sa.
\end{threecol}

\paragraph{Mani non bilanciate.} Si apre a livello di 1 con 12-21 HCP, licitando secondo l'algoritmo generale. Tutte le mani di 22+ HCP (e le mani con forza di manche) si aprono 2 \c.

%\paragraph{Sottoaperture (barrage).} Si può aprire a livello di 2 con colore sesto (ma non 2 \c!), a livello di 3 con colore almeno settimo, in entrambi i casi con 6-10 HCP e meglio se a compagno passato (barrage).

\section{Risposte sull'apertura di 1 a colore}

\biddingtable{1 \j & P & *}
Si risponde con 5+ HCP oppure con un Asso.

\paragraph{Senza fit.} Se possibile, dire il colore più lungo a livello di 1 (a parità di lunghezza, il più economico). Per salire a livello di 2 bisogna avere almeno 12 HCP (a meno che non sia per dare il fit). Se non si rientra in nessuna di queste possibilità, si licita 1 \sa.

\paragraph{Con fit sul colore \j dell'apertore.} Se il colore è \c o \d, ha comunque la priorità licitare un proprio seme almeno quarto a livello di 1 (c'è tempo per dare il fit successivamente). Per dare fit direttamente:

\biddingtable{1 \j & P & *}
\begin{twocol}
 2 \j & Mano minima (5-8 HCP).\\
 3 \j & Invitante a manche (9-11 HCP).\\
 4 \j & Garantisce la manche (12-15 HCP).\\
 4 \sa & Richiesta d'Assi (16+ HCP).
\end{twocol}

Dopo 3 \j si può passare (12-13) o dire 4 \j (14+). Dopo 2 \j si può passare (12-16), invitare con 3 \j (17-18) o giocare 4 \j (19+).

\section{Risposte sull'apertura di 1 \sa}

\paragraph{Per giocare a \sa.} Il rispondente può licitare

\biddingtable{1 \sa & P & *}
\begin{twocol}
 PASS & 0-7 HCP.\\
 2 \sa & Invitante (8-9 HCP).\\
 3 \sa & Sign-off (10-14 HCP).\\
 4 \c & Richiesta d'Assi (Gerber).
\end{twocol}

\paragraph{Transfer.} Il rispondente può licitare

\biddingtable{1 \sa & P & *}
\begin{twocol}
 2 \d & Obbliga l'apertore a licitare 2 \h.\\
 2 \h & Obbliga l'apertore a licitare 2 \s.\\
 2 \s & Obbliga l'apertore a licitare 3 \c.\\
 3 \c & Obbliga l'apertore a licitare 3 \d.
\end{twocol}

In questo modo si seleziona l'atout; il rispondente poi guida scegliendo il livello del contratto (pass, invito a manche, manche, cue-bid, richiesta d'assi).

\paragraph{Richiesta di quarte nobili (Stayman).} Con 8+ HCP il rispondente può licitare 2 \c, cercando fit in un seme nobile. L'apertore licita come segue.

\biddingtable{1 \sa & P & 2 \c & P \\ *}
\begin{twocol}
 2 \d & Nessuna quarta nobile.\\
 2 \h & \h quarte, ma non \s quarte.\\
 2 \s & \s quarte, ma non \h quarte.\\
 2 SA & Entrambe le quarte nobili.
\end{twocol}

Il rispondente ora conosce talmente bene la mano dell'apertore da poter selezionare il contratto migliore, passando eventualmente da un invito.

\section{Risposte sull'apertura di 2 \sa}

Ciò che segue si applica anche nei seguenti casi:
\begin{itemize}
  \item l'apertura è stata 1 \c\ - 1 \j\ - 2 \sa (18-19 HCP). In questo caso bisogna alzare di due punti tutti i range (poiché l'apertore ha due punti in meno).
  \item l'apertura è stata 2 \c\ - 2 \j\ - 2 \sa (22+ HCP). In questo caso bisogna abbassare di due punti tutti i range (poiché l'apertore ha due punti in più).
\end{itemize}

\paragraph{Per giocare a \sa.} Il rispondente può licitare

\biddingtable{2 \sa & P & *}
\begin{twocol}
 PASS & 0-4 HCP.\\
 3 \sa & Sign-off (5-10 HCP).\\
 4 \c & Richiesta d'Assi (Gerber).
\end{twocol}

\paragraph{Transfer.} Il rispondente può licitare

\biddingtable{2 \sa & P & *}
\begin{twocol}
 3 \d & Obbliga l'apertore a licitare 3 \h.\\
 3 \h & Obbliga l'apertore a licitare 3 \s.\\
 3 \s & Obbliga l'apertore a licitare 4 \c.\\
 4 \c & Obbliga l'apertore a licitare 4 \d.\\
\end{twocol}

In questo modo si seleziona l'atout; il rispondente poi guida scegliendo il livello del contratto. Le ultime due possibilità vanno usate con cautela, poiché il livello sale molto e, soprattutto, vi prendete la responsabilità di saltare 3 \sa!

\paragraph{Richiesta di quarte nobili (Stayman).} Con 4/5+ HCP (manche garantita, almeno a \sa) il rispondente può licitare 3 \c, cercando fit in un seme nobile. L'apertore licita come segue.

\biddingtable{2 \sa & P & 2 \c & P \\ *}
\begin{twocol}
 3 \d & Nessuna quarta nobile.\\
 3 \h & \h quarte, ma non \s quarte.\\
 3 \s & \s quarte, ma non \h quarte.\\
 3 SA & Entrambe le quarte nobili.
\end{twocol}

Il rispondente ora conosce talmente bene la mano dell'apertore da poter selezionare il contratto migliore.

\section{Rever}

Dopo l'apertura di 1 \j (12-21 HCP) e una risposta a livello di 1, con la propria seconda dichiarazione l'apertore deve comunicare al compagno se ha 12-15 HCP o 16-20 HCP (rever). Per dare rever occorre superare strettamente 2 \j (colore di apertura).

Fa eccezione il caso in cui l'apertore voglia dare fit al rispondente: in questo caso licitare il colore del rispondente a livello dà 12-15 HCP, a salto dà rever. Ecco alcuni esempi:

\begin{twocol}
1 \s\ - 1 \sa\ - 2 \h & 12-15 HCP, con almeno 5 \s e 4 \h.\\
1 \d\ - 1 \h\ - 3 \h & 16-18 HCP, con almeno 4 \d e 4 \h (si giocherà a \h). Nota: con 19-20 HCP l'apertore deve licitare invece 4\h. Perché?\\
1 \h\ - 1 \s\ - 2 \s & 12-15 HCP, con almeno 5 \h e 4 \s (si giocherà a \s).\\
1 \d\ - 1 \s\ - 2 \h & 16-21 HCP, con almeno 5 \d e 4 \h, e non 4 \s (dare il fit nobile avrebbe la precedenza). Con la stessa distribuzione ma 12-15 HCP, l'apertore licita invece 1 \sa o 2 \d, tenendosi le \h per sé.
\end{twocol}

\section{Roudi}

Dopo la sequenza 1 \j\ - 1 nobile - 1 \sa, il rispondente può licitare 2 \c (Roudi) per interrogare l'apertore sul punteggio, e sull'eventuale presenza della terza nel seme nobile (al fine di scoprire i fit 5-3). Si risponde come segue.

\biddingtable{1 \j & P & 1 \M & P \\ 1 \sa & P & *}
\begin{twocol}
  2 \d & 12-13 HCP, senza terza nel nobile.\\
  2 \h & 12-13 HCP, terza nel nobile.\\
  2 \s & 14-15 HCP, terza nel nobile.\\
  2 \sa & 14-15 HCP, senza terza nel nobile.
\end{twocol}

\section{Cue-bid}

Le cue-bid possono cominciare solo dopo che è stata espressamente concordata un'atout. Licitando un seme (non atout) si promette di avere un controllo nel seme (Asso, Re, singolo o vuoto). Se si salta un seme si nega il controllo (ovvero, bisogna sempre licitare la cue-bid più economica disponibile). Per proseguire dopo che il compagno ha saltato un seme, si deve avere un controllo in quel seme. Se un giocatore licita un colore in cui aveva già promesso un controllo, sta rinforzando il proprio controllo (Asso o vuoto).

Spesso (ma non sempre) è opportuno interrompere le cue-bid per effettuare una richiesta d'Assi (4 \sa).

\section{Richieste d'Assi e di Re}

\paragraph{Con atout concordata.} Si possono richiedere gli Assi al compagno licitando 4 \sa (Roman Key Card Blackwood). Il Re di atout è considerato il quinto Asso. Risposte:

\begin{twocol}
5 \c & 0 o 3 Assi.\\
5 \d & 1 o 4 Assi.\\
5 \h & 2 o 5 Assi senza la Q di atout.\\
5 \s & 2 o 5 Assi con la Q di atout.
\end{twocol}

Dopo una risposta 5 \m, se il richiedente vuole informazioni riguardo alla Q di atout può effettuare la prima licita disponibile (diversa dall'atout). Risposte:

\begin{itemize}
 \item la prima licita ad atout disponibile: niente Q di atout;
 \item la prima licita a \j disponibile (ma necessariamente inferiore a 6 nel colore di atout): Q di atout e K di \j;
 \item 5 \sa: Q di atout, ma nessun K oppure nessun K dichiarabile secondo il punto precedente.
\end{itemize}

In alternativa (o in aggiunta), chi aveva chiesto gli Assi può chiedere anche i Re licitando 5 \sa. Risposte:

\begin{twocol}
6 \c & 0 Re.\\
6 \d & 1 Re.\\
6 \h & 2 Re.\\
6 \s & 3 Re.
\end{twocol}

\paragraph{Per giocare a \sa.} Dopo che è stato licitato un contratto a \sa in modo naturale (ad es. apertura di 1 \sa, conclusione a 3 \sa, etc.), si possono chiedere gli Assi licitando 4 \c (Gerber).  Risposte:

\begin{twocol}
  4 \d & 0 o 4 Assi.\\
  4 \h & 1 Assi.\\
  4 \s & 2 Assi.\\
  4 \sa & 3 Assi.
\end{twocol}

Chi aveva chiesto gli Assi, può in seguito licitare 5 \c (richiesta di Re). Risposte:

\begin{twocol}
  5 \d & 0 o 4 Re.\\
  5 \h & 1 Re.\\
  5 \s & 2 Re.\\
  5 \sa & 3 Re.
\end{twocol}

Dopo aver ricevuto le risposte, il richiedente sceglierà il contratto da giocare. La struttura permette di fermarsi a 4 \sa o 5 \sa (a giocare) se le risposte non sono soddisfacenti.

\section{Sottoaperture}

Con una mano debole (6-10 HCP) e monocolore, si può effettuare un'apertura di interferenza a livello alto (barrage) per disturbare la comunicazione costruttiva degli avversari. Si apre come segue:
\begin{twocol}
  2 \d, \h, \s & 6-10 HCP, \j sesto.\\
  3 \j & 6-10 HCP, \j almeno settimo (meglio se a compagno passato).
\end{twocol}

NB: poiché l'apertura di 2 \c codifica le mani fortissime, si sacrifica la possibilità di effettuare una sottoapertura con la sesta di \c.

Dopo una sottoapertura 2 \j, con una mano interessata a manche (14+ HCP) il compagno può chiedere informazioni licitando 2 \sa (interrogativo). Risposte:

\biddingtable{2 \j & P & 2 \sa & P \\ *}
\begin{twocol}
  3 \c & Mano minima, al più uno tra A,K e Q di \j (seme brutto).\\
  3 \d & Mano minima, due tra A, K e Q di \j (seme bello).\\
  3 \d & Mano massima, al più uno tra A,K e Q di \j (seme brutto).\\
  3 \d & Mano massima, due tra A, K e Q di \j (seme bello).\\
  3 \sa & AKQ di \j (seme chiuso).
\end{twocol}

\section{Interventi}

\begin{twocol}
  X (contre) & 12+ HCP, al massimo due carte nei semi licitati dagli avversari, almeno 4 carte in un seme nobile. In alternativa, mano forte (rever) con qualsiasi distribuzione.\\
  1 \j & 9+ HCP, \j almeno quinto.\\
  1 \sa & Come apertura (15-17 HCP, mano bilanciata), con fermo nel seme degli avversari.\\
  2 \j a livello & 12+ HCP, \j almeno quinto.\\
  2 \sa & Come apertura (20-21 HCP, mano bilanciata), con fermo nel seme degli avversari.\\
  2 \j a salto & 6-10 HCP, \j almeno sesto (come sottoapertura).\\
  3 \j & 6-10 HCP, \j almeno settimo (come sottoapertura).
\end{twocol}

\section{Principi generali e consigli}

\indent

Ricordate che le licite possono essere:
\begin{itemize}
  \item Forzanti: il compagno non può passare.
  \item Invitanti: il compagno deve passare con il minimo, rialzare con il massimo.
  \item Non forzanti: il compagno può passare, ma può anche licitare.
  \item Sign-off: il contratto è stabilito e il compagno deve passare.
\end{itemize}
Per ogni licita del compagno, chiedetevi a quale categoria appartenga e agite di conseguenza. Prima di ogni vostra licita, chiedetevi a quale categoria appartenga e controllate che sia coerente con quello che volete ottenere.

\smallspace

Un nuovo colore (mai licitato prima dalla coppia) a livello di 2 o più è sempre forzante, salvo se chi lo licita ha un upper-bound esplicito di forza. Inoltre, la licita di un nuovo colore da parte del rispondente è sempre forzante, anche a livello di 1.

\smallspace

Non codifichiamo un modo per dare rever dopo una risposta a livello di 2. D'altronde, poiché entrambi già sanno che si hanno almeno 24 HCP in linea, non ci si ferma prima di una manche, e non ci sono problemi a salire con il livello per continuare la descrizione della mano.

\smallspace


Ripetere un proprio colore in cui non si è ancora trovato un fit allunga il colore di una carta. Eccezione: dopo l'apertura di 1 \c e una risposta a livello di 1, licitare nuovamente le \c promette almeno 5 carte di \c.

\smallspace

Non abbiate paura ad usare il Contre! Molto più spesso di quanto pensiate è la cosa giusta da fare. Contre su una licita a livello di 1 o di 2 è forzante, a meno di casi molto eccezionali (es: 1 \c\ - X - Pass e il quarto giocatore ha 7/8 carte di fiori, o simili).

\smallspace

Giocando ad atout è meglio avere il fit, ma non è indispensabile: talvolta può essere opportuno giocare un contratto con 7 carte in linea, piuttosto che intestardirsi nella ricerca di un fit e salire troppo con la dichiarazione.

\end{document}
