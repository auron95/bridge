\documentclass[a4paper,10pt]{article}


\usepackage[utf8]{inputenc}
\usepackage[italian]{babel}
\usepackage[T1]{fontenc}
\usepackage[dvips]{graphicx}
\usepackage{amsmath}
\usepackage{amsthm}
\usepackage{fancyhdr}
\usepackage{amsfonts}
\usepackage{amssymb}
\usepackage{xspace}

\topmargin -1cm
\oddsidemargin -0.5cm
%\evensidemargin	-1cm
\textwidth 17cm


\renewcommand{\c}{$\clubsuit$\xspace}
\renewcommand{\d}{$\diamondsuit$\xspace}
\newcommand{\h}{$\heartsuit$\xspace}
\newcommand{\s}{$\spadesuit$\xspace}
\renewcommand{\j}{$\bigstar$\xspace}
\newcommand{\sa}{SA\xspace}

\newcommand{\smallspace}{\vskip0.3cm}

% Title Page
\title{Convenzioni di bridge}
\author{Giove}

\begin{document}
\maketitle


\section{Aperture}

\paragraph{Algoritmo generale.}
Si licita il colore pi\`u lungo, escludendo \s e \h se non sono almeno quinte e le \d se non sono almeno quarte.
In caso di parit\`a di lunghezza dei semi pi\`u lunghi, si dichiara il pi\`u alto se sono almeno quinti e si dichiara il pi\`u basso se sono quarti.
Nota: con 4 carte a \s e \h, 3 a \d\ e 2 a \c, si apre di 1\c.

\subsection{Mani bilanciate}
(4-3-3-3, 4-4-3-2, 5-3-3-2)
\smallspace

\begin{tabular}{p{0.05\textwidth} p{0.05\textwidth} p{0.7\textwidth}}
 12-14 & 1 \j & Si licita seguendo l'algoritmo generale.\\

 15-17 & 1 \sa & \\

 18-19 & 1 \j & 5-3-3-2, \j \`e il seme quinto. Poi \sa a salto.\\
       & 1 \c & Tutte le altre distribuzioni. Poi \sa a salto.\\

 20-22 & 2 \sa & \\

 23+ & 2 \c & 
 \end{tabular}

\subsection{Mani monocolore}
(6-3-2-2, 6-3-3-1, 7-3-2-1, \dots)\\

\begin{tabular}{p{0.05\textwidth} p{0.05\textwidth} p{0.7\textwidth}}
 12+ & 1 \j & \j \`e il colore lungo. Poi si ripete il colore, eventualmente a salto.
\end{tabular}


\subsection{Mani bicolore}
(5-4-3-1, 5-4-2-2, 6-4-2-1, 5-5-2-1, \dots)\\

\begin{tabular}{p{0.05\textwidth} p{0.05\textwidth} p{0.7\textwidth}}
 12+ & 1 \j & \j \`e il colore pi\`u lungo; in caso di parit\`a (5-5 o 6-6) si licita il color pi\`u alto (in accordo con l'algoritmo generale). Sulla ridichiarazione, per licitare il secondo colore a livello di 2 in modo ascendente (es: 1\d{} - 1\s{} - 2\h), bisogna avere 16+ HCP (rever).
\end{tabular}


\subsection{Mani tricolore}
(4-4-4-1, 5-4-4-0)\\

\begin{tabular}{p{0.05\textwidth} p{0.05\textwidth} p{0.7\textwidth}}
 12+ & 1 \j & Si licita seguendo l'algoritmo generale, con un'eccezione: se si ha una 4-4-4-1 con singolo a \s e meno di 16 HCP, si apre di 1\d (in modo che, sull'eventuale risposta di 1\s, si possa licitare 2\c).
\end{tabular}


\subsection{Sottoaperture}

\begin{tabular}{p{0.09\textwidth} p{0.7\textwidth}}
 2 \j & 6-10 HCP, con \j colore esattamente sesto diverso da \c.\\
 3 \j / 4 \j & Colore almeno settimo (meglio se a compagno passato).\\
 3 \sa & Minore almeno settimo chiuso (con A, K, Q) e non pi\`u di una dama negli altri tre colori.
\end{tabular}



\section{Risposte sull'apertura di 1 a colore}


 

\end{document}
