\documentclass[a4paper,10pt]{article}


\usepackage[utf8]{inputenc}
\usepackage[italian]{babel}
\usepackage[pdftex]{graphicx}
\usepackage{amsmath}
\usepackage{amsthm}
\usepackage{fancyhdr}
\usepackage{amsfonts}
\usepackage{amssymb}
\usepackage{xspace}
\usepackage{color}

\topmargin -1cm
\oddsidemargin -0.5cm
\textwidth 17cm


\renewcommand{\c}{$\clubsuit$\xspace}
\renewcommand{\d}{$\diamondsuit$\xspace}
\newcommand{\h}{$\heartsuit$\xspace}
\newcommand{\s}{$\spadesuit$\xspace}
\renewcommand{\j}{$\bigstar$\xspace}
\newcommand{\sa}{SA\xspace}

\newcommand{\smallspace}{\vskip0.3cm}

\renewcommand{\tabcolsep}{0.3cm}

\newcommand{\note}[1]{\textcolor{red}{#1}}


\newenvironment{twocol}
  {\smallspace\noindent\begin{tabular}{l p{0.8\textwidth}}}
  {\end{tabular}\smallspace}

\newenvironment{threecol}
  {\smallspace\noindent\begin{tabular}{l l p{0.7\textwidth}}}
  {\end{tabular}\smallspace}

  
% Title Page
\title{Convenzioni di bridge}
\author{Ugo Bindini \& Giovanni Paolini}

\begin{document}
\maketitle


\section{Aperture}

\paragraph{Algoritmo generale.}
Si licita il colore pi\`u lungo, escludendo \s e \h se non sono almeno quinte e le \d se non sono almeno quarte.
In caso di parit\`a di lunghezza dei semi pi\`u lunghi, si dichiara il pi\`u alto se sono almeno quinti e si dichiara il pi\`u basso se sono quarti.
Nota: con 4\s\ - 4\h\ - 3\d\ - 2\c, si apre di 1\c.

\subsection{Mani bilanciate}
(4-3-3-3, 4-4-3-2, 5-3-3-2)
\smallspace

\begin{threecol}
 12-14 & 1 \j & Si licita seguendo l'algoritmo generale.\\
 15-17 & 1 \sa & Solo se la quinta non \`e nobile.\\
 18-19 & 1 \j & 5-3-3-2, \j \`e il seme quinto (anche nobile). Poi \sa a salto sulla risposta di 1 a colore.\\
       & 1 \c & Tutte le altre distribuzioni. Poi \sa a salto sulla risposta di 1 a colore.\\
 20-22 & 2 \sa & Solo se la quinta non è nobile.\\
 23+ & 2 \c & \note{Da rivedere}
\end{threecol}


\subsection{Mani monocolore}
(6-3-2-2, 6-3-3-1, 7-3-2-1, \dots)

\begin{threecol}
 12+ & 1 \j & \j \`e il colore lungo. Poi si ripete il colore, eventualmente a salto.
\end{threecol}


\subsection{Mani bicolore}
(5-4-3-1, 5-4-2-2, 6-4-2-1, 5-5-2-1, \dots)

\begin{threecol}
 12+ & 1 \j & \j \`e il colore pi\`u lungo; in caso di parit\`a (5-5 o 6-6) si licita il colore pi\`u alto (in accordo con l'algoritmo generale). Sulla ridichiarazione, per licitare il secondo colore a livello di 2 in modo ascendente (es: 1\d{} - 1\s{} - 2\h), bisogna avere 16+ HCP (Rever).
\end{threecol}


\subsection{Mani tricolore}
(4-4-4-1, 5-4-4-0)

\begin{threecol}
 12+ & 1 \j & Si licita seguendo l'algoritmo generale, con un'eccezione: se si ha una 4-4-4-1 con singolo a \s e meno di 16 HCP, si apre di 1\d anzich\'e 1\c (in modo che, sull'eventuale risposta di 1\s, si possa licitare 2\c).
\end{threecol}


\subsection{Sottoaperture}

\begin{twocol}
 2 \j & 6-10 HCP, con \j colore esattamente sesto diverso da \c.\\
 3 \j / 4 \j & Colore almeno settimo (meglio se a compagno passato).\\
 3 \sa & Minore almeno settimo chiuso (con A, K, Q) e non pi\`u di una dama negli altri tre colori.
\end{twocol}



\pagebreak

\section{Risposte sull'apertura di 1 a colore}

Si risponde tendenzialmente con $\geq 5$ punti oppure con un Asso.

\subsection{Senza fit}

\begin{itemize}
 \item Se possibile, dire un colore a livello di 1 (il pi\`u lungo, e in caso di parit\`a come da algoritmo generale).
 \item Altrimenti:
  \begin{threecol}
    $\leq 10$ HCP & 1 \sa & \\
    $\geq 11$ HCP & 2 \h & Con le \h almeno quinte.\\
    & 2 \d & Con le \d almeno quarte.\\
    & 2 \c & In tutti gli altri casi.\\
  \end{threecol}
\end{itemize}


\subsection{Con fit su un colore maggiore \j}

\begin{twocol}
 2 \j  & 5-7 HCP, fit terzo.\\
 3 \c & 10+ HCP, fit terzo.\\
 3 \d & 8-9 HCP, fit terzo.\\
 3 \j  & 5-9 HCP, fit quarto.\\
 4 \j  & 5-9 HCP, fit almeno quinto.\\
 2 \sa & 10+ HCP, fit almeno quarto (Jordan).\\
 Cue-bid & 12+ HCP, fit terzo. Forzante manche.
\end{twocol}

\paragraph{Sviluppo su 3 \c.} L'apertore licita
\begin{twocol}
 3 \d & 12-13 HCP.\\
 3 \h & 14-15 HCP.\\
 3 \s & 16-17 HCP.\\
 3 \sa & 18+ HCP.
\end{twocol}

Il rispondente può ora valutare esattamente la forza della mano e scegliere se: fermarsi a 4 \j, licitare una Cue-bid o chiedere gli Assi.

\paragraph{Sviluppo su 2 \sa (Jordan).} L'apertore licita
\begin{twocol}
 3 \c & Mano minima (12-13 HCP).\\
 3 \d & 14+ HCP, singolo o vuoto a \d.\\
 3 \h & 14+ HCP, singolo o vuoto a \h (a \c se \h è l'atout).\\
 3 \s & 14+ HCP, singolo o vuoto a \s (a \c se \s è l'atout).\\
 3 \sa & 16+ HCP, senza singoli né vuoti.\\
 4 \c & 16+ HCP, \c quinte.\\
 4 \d & 16+ HCP, \d quinte.\\
 4 \h & 16+ HCP, \h quinte (atout \s); 14-15 HCP, senza singoli né vuoti (atout \h).\\
 4 \s & 16+ HCP, \s quinte (atout \h); 14-15 HCP, senza singoli né vuoti (atout \s).
\end{twocol}

\subsection{Con fit sulle \d}

\begin{twocol}
 1 \h/\s & Se possibile, come nel caso senza fit.\\
 1 \sa & ??\\
 2 \d  & Fit scarso.\\
 3 \d  & Fit migliore (11-12 HCP).\\
 2 \c  & Forzante manche (13+ HCP).\\
\end{twocol}



\pagebreak

\section{Risposte sull'apertura 1 SA}

\subsection{Mano sbilanciata (Transfer)}

Almeno sei carte in un colore (o anche cinque con singoli o vuoti con senno, o con pochi punti).

\begin{twocol}
 2 \d & Transfer per le \h.\\
 2 \h & Transfer per le \s.\\
 2 \s & Transfer per le \c.\\
 3 \c & Transfer per le \d.\\
\end{twocol}

L'apertore deve licitare il seme richiesto dal rispondente (che diventa automaticamente l'atout concordata), dopodiché il rispondente può proseguire passando, licitando manche o slam (a giocare), iniziando le Cue-bid, oppure chiedendo gli Assi (4 \sa).


\subsection{Mano bilanciata, senza quarte nobili e senza ambizioni di slam}

\begin{twocol}
 2 \sa & 6-8 HCP, invitante manche.\\
 3 \sa & 9-12 HCP, a giocare.
\end{twocol}


\subsection{1 SA - 2 \c (Stayman)}

Richiesta di quarte nobili, $\geq 7$ HCP. Seguono le risposte.

\begin{twocol}
 2 \d & Nessuna quarta nobile e minimo dell'apertura (15 HCP).\\
 2 \h & \h quarte, ma non \s quarte.\\
 2 \s & \s quarte, ma non \h quarte.\\
 2 SA & Nessuna quarta nobile e massimo dell'apertura (17 HCP).\\
 3 \c & 4\h\ - 4\s\ - 3\c\ - 2\d.\\
 3 \d & 4\h\ - 4\s\ - 3\d\ - 2\c.\\
\end{twocol}

\noindent Esaminiamo gli sviluppi della Stayman.

\paragraph{Fit in un colore nobile.}

Con il fit su un nobile \j, dopo la risposta di 2 \j:
\begin{twocol}
 3 \j & Seleziona \j come atout. Invitante manche.\\
 4 \j & A giocare.\\
 Cue-bid & A partire da 3 \s. Sottintende il nobile dell'apertore come atout.\\
\end{twocol}

\noindent Con il fit su un nobile \j, dopo la risposta di 3 \c o 3 \d: il rispondente dichiara \j, e l'apertore inizia le Cue-bid.


\paragraph{Richiesta dell'altro nobile.}
Se il richiedente aveva una quinta nobile \j, ma non c'\`e fit quarto dell'apertore, può dichiarare il primo \j disponibile (con senno). Se l'apertore ha fit (terzo) inizia le Cue-bid, altrimenti licita \sa.

\paragraph{Gioco a SA.}
Per giocare a SA il rispondente licita:
\begin{twocol}
 2 \sa & (se possibile) 7-8 HCP, invitante manche.\\
 3 \sa & A giocare.\\
 4 \sa & Richiesta d'Assi.
\end{twocol}

\paragraph{Richiesta dei minori.} Si può effettuare dopo le risposte di 2 \d, 2 \h, 2 \s o 2 \sa, dichiarando 3 \c. È forzante manche. Seguono le risposte.

\begin{itemize}
 \item Se l'apertore ha negato quarte nobili (cioè dopo 2 \d o 2 \sa):
  \begin{twocol}
    3 \d & \d almeno quarte, \c al massimo terze (5-3-3-2 oppure 4-3-3-3).\\
    3 \h & 4\c\ - 4\d\ - 3\h\ - 2\s.\\
    3 \s & 4\c\ - 4\d\ - 3\s\ - 2\h.\\
    3 \sa & \c almeno quarte, \d al massimo terze (5-3-3-2 oppure 4-3-3-3).
  \end{twocol}

 \item Se l'apertore ha un palo nobile quarto (cioè dopo 2 \h o 2 \s):
  \begin{twocol}
    3 \d & \d quarte (4-4-3-2).\\
    3 \h & Nessun minore quarto, mano minima.\\
    3 \s & Nessun minore quarto, mano massima.\\
    3 \sa & \c quarte (4-4-3-2).
  \end{twocol}
\end{itemize}

\noindent Seguono gli sviluppi dopo la richiesta dei minori.
\begin{itemize}
 \item Per giocare a \sa: 3 \sa (o passo, sulla risposta di 3 \sa), a giocare; 4 \sa richiesta d'Assi.
 \item Se c'è fit su un minore, dopo 3 \d o 3 \sa (il fit non è ambiguo): Cue-bid nel primo maggiore disponibile, se possibile; altrimenti licitare il minore (il che nega le Cue-bid nei colori maggiori, per cui l'apertore deve chiudere a manche se non ha la copertura necessaria per sperare in uno slam).
 L'apertore prosegue con la prima Cue-bid, oppure chiudendo a manche.
 \item Se c'è fit su un minore, dopo 3 \h o 3 \s nel primo caso: licitare il minore.
 L'apertore prosegue con la prima Cue-bid.
 \item Se si vuole segnalare un minore quinto \j (in cui l'apertore non ha mostrato quattro carte): dichiarare 4 \j.
 L'apertore prosegue così:
 \begin{itemize}
  \item 4 ($\bigstar+1$) senza fit terzo nel minore. Il rispondente dichiara 4 \sa (richiesta d'Assi), oppure 4 ($\bigstar+2$) per costringere l'apertore a licitare 4 \sa (a giocare).
  \item La prima Cue-bid disponibile che non sia 4 ($\bigstar+1$), ma comunque senza superare 5 \j.
  
 \end{itemize}
\end{itemize}

\pagebreak

\section{Sviluppi su apertura forte bilanciata}

\paragraph{Transfer.} Sull'apertura 2 \sa, con mano sbilanciata il rispondente può licitare una Transfer.

\paragraph{Stayman avanzata.} Su 2 \sa (20-22 HCP), oppure 1 \c\ - 1 \j\ - 2 \sa (18-19 HCP), il rispondente può giocare 3 \c (Stayman avanzata). L'apertore risponde come segue.
\begin{twocol}
 3 \d & Entrambe le quarte nobili.\\
 3 \h & \h quarte, ma non \s quarte.\\
 3 \s & \s quarte, ma non \h quarte.\\
 3 \sa & Nessuna quarta nobile.
\end{twocol}



\pagebreak

\section{Verso lo slam}

\subsection{Cue-bid}

Non si gioca se il contratto sarà a SA, e comunque occorre avere già stabilito l'atout (esplicitamente o implicitamente). A partire da 3 \s e salendo gradualmente si dichiara un seme \j (che non sia l'atout) se si possiede una delle seguenti (Cue-bid debole): Asso di \j, K di \j, singolo o vuoto a \j.

Saltare un seme disponibile è come dichiarare che non si ha la Cue-bid in quel seme. Dichiarare SA è come dichiarare l'ultimo seme detto dal compagno, ma sono esclusi 4 SA e 5 SA (richiesta d'Assi e di Re). L'unica possibilità è quindi 3 \s\ - 3 SA, che dichiara Cue-bid (debole) a \s per entrambi i giocatori. Se un giocatore ripete un seme \j su cui ha già dato una Cue-bid, possiede una delle seguenti (Cue-bid forte): Asso di \j, singolo a \j.

Quando un giocatore nega la Cue-bid in un seme, il compagno dovrebbe immediatamente chiudere a manche se a sua volta non possiede la Cue-bid in quel seme.
Come corollario, continuare le Cue-bid dopo che il compagno ne ha negate alcune, equivale a dichiarare (implicitamente) le Cue-bid negate dal compagno.

Esempio (con atout \h): 3 \s\  (Cue-bid debole a \s) - 4 \d\ (nega la Cue-bid a \s e a \c) - 4 \s (Cue-bid forte a \s e debole a \c) - 5 \h\ (nega la Cue-bid forte a \d) - 6 \c\ (Cue-bid forte a \c) - \dots

Ovviamente talvolta ha senso interrompere le Cue-bid per effettuare una richiesta d'Assi.


\subsection{Richiesta d'Assi e richiesta di Re (Roman Key Card Blackwood)}

\paragraph{Con atout.} Ai fini delle richieste d'Assi e di Re, il K di atout \`e considerato un Asso. In questo caso quindi ci sono 5 Assi e 3 Re. Seguono le risposte alla richiesta d'Assi (4 \sa).
\begin{twocol}
5 \c & 0 o 3 Assi.\\
5 \d & 1 o 4 Assi.\\
5 \h & 2 o 5 Assi, senza Q di atout.\\
5 \s & 2 o 5 Assi, con Q di atout.\\
\end{twocol}

\noindent Se si riceve una risposta che non dice nulla sulla Q di atout, si può licitare il primo colore disponibile che non sia l'atout per chiedere la Q di atout. Seguono le risposte.

\begin{itemize}
 \item La prima licita ad atout disponibile: niente Q di atout.
 \item La prima licita a \j disponibile (ma necessariamente inferiore a \note{6ATOUT}): Q di atout e K di \j.
 \item 5 \sa: Q di atout, e nessun K oppure nessun K dichiarabile secondo il punto precedente.
\end{itemize}

Se l'atout è \s, questa dichiarazione permette di dare il K di ogni seme, (nel caso in cui il rispondente possieda la Q di \s). Se l'atout è \h, è comunque sempre possibile sapere i K del rispondente mediante questa tecnica: con richiesta 5\s, la risposta 5 \sa potrebbe significare nessun K, ma anche K di \s.
Allora il richiedente domanda 6\c, e il rispondente licita 6\d se non possiede il K di \s, 6\h se possiede il K di \s.


\smallspace

\noindent In alternativa alla richiesta di Q di atout, con 5 \sa si può effettuare la richiesta di Re. Seguono le risposte.

\begin{twocol}
6 \c & Nessun Re.\\
6 \d & 1 Re.\\
6 \h & 2 Re.\\
6 \s & 3 Re. \\
\end{twocol}

\paragraph{Senza atout.} Seguono le risposte alla richiesta d'Assi (4 \sa).

\begin{twocol}
5 \c & 0 o 3 Assi.\\
5 \d & 1 o 4 Assi.\\
5 \h & 2 Assi.\\
\end{twocol}

\noindent Il richiedente può successivamente licitare 5 \s per costringere il rispondente a licitare 5 \sa (a giocare).

\smallspace

\noindent In alternativa si può effettuare la richiesta di Re, dichiarando 5 \sa. Seguono le risposte.

\begin{twocol}
6 \c & Nessun Re.\\
6 \d & 1 Re.\\
6 \h & 2 Re.\\
6 \s & 3 Re. \\
6 \sa & 4 Re.
\end{twocol}





\end{document}
