\documentclass[a4paper,10pt]{article}


\usepackage[utf8]{inputenc}
\usepackage[italian]{babel}
\usepackage[T1]{fontenc}
\usepackage[dvips]{graphicx}
\usepackage{amsmath}
\usepackage{amsthm}
\usepackage{fancyhdr}
\usepackage{amsfonts}
\usepackage{amssymb}
\usepackage{xspace}

\topmargin -1cm
\oddsidemargin -0.5cm
%\evensidemargin	-1cm
\textwidth 17cm


\renewcommand{\c}{$\clubsuit$\xspace}
\renewcommand{\d}{$\diamondsuit$\xspace}
\newcommand{\h}{$\heartsuit$\xspace}
\newcommand{\s}{$\spadesuit$\xspace}
\renewcommand{\j}{$\bigstar$\xspace}
\newcommand{\sa}{SA\xspace}

\newcommand{\smallspace}{\vskip0.3cm}

% Title Page
\title{Convenzioni di bridge}
\author{Giove}

\begin{document}
\maketitle


\section{Aperture}

\paragraph{Algoritmo generale.}
Si licita il colore pi\`u lungo, escludendo \s e \h se non sono almeno quinte e le \d se non sono almeno quarte.
In caso di parit\`a di lunghezza dei semi pi\`u lunghi, si dichiara il pi\`u alto se sono almeno quinti e si dichiara il pi\`u basso se sono quarti.
Nota: con 4 carte a \s e \h, 3 a \d\ e 2 a \c, si apre di 1\c.

\subsection{Mani bilanciate}
(4-3-3-3, 4-4-3-2, 5-3-3-2)
\smallspace

\begin{tabular}{p{0.05\textwidth} p{0.05\textwidth} p{0.7\textwidth}}
 12-14 & 1 \j & Si licita seguendo l'algoritmo generale.\\

 15-17 & 1 \sa & Solo se la quinta non \`e nobile\\

 18-19 & 1 \j & 5-3-3-2, \j \`e il seme quinto. Poi \sa a salto.\\
       & 1 \c & Tutte le altre distribuzioni. Poi \sa a salto.\\

 20-22 & 2 \sa & \\

 23+ & 2 \c & 
 \end{tabular}

\subsection{Mani monocolore}
(6-3-2-2, 6-3-3-1, 7-3-2-1, \dots)\\

\begin{tabular}{p{0.05\textwidth} p{0.05\textwidth} p{0.7\textwidth}}
 12+ & 1 \j & \j \`e il colore lungo. Poi si ripete il colore, eventualmente a salto.
\end{tabular}


\subsection{Mani bicolore}
(5-4-3-1, 5-4-2-2, 6-4-2-1, 5-5-2-1, \dots)\\

\begin{tabular}{p{0.05\textwidth} p{0.05\textwidth} p{0.7\textwidth}}
 12+ & 1 \j & \j \`e il colore pi\`u lungo; in caso di parit\`a (5-5 o 6-6) si licita il colore pi\`u alto (in accordo con l'algoritmo generale). Sulla ridichiarazione, per licitare il secondo colore a livello di 2 in modo ascendente (es: 1\d{} - 1\s{} - 2\h), bisogna avere 16+ HCP (rever).
\end{tabular}


\subsection{Mani tricolore}
(4-4-4-1, 5-4-4-0)\\

\begin{tabular}{p{0.05\textwidth} p{0.05\textwidth} p{0.7\textwidth}}
 12+ & 1 \j & Si licita seguendo l'algoritmo generale, con un'eccezione: se si ha una 4-4-4-1 con singolo a \s e meno di 16 HCP, si apre di 1\d anzich\'e 1\c (in modo che, sull'eventuale risposta di 1\s, si possa licitare 2\c).
\end{tabular}


\subsection{Sottoaperture}

\begin{tabular}{p{0.09\textwidth} p{0.7\textwidth}}
 2 \j & 6-10 HCP, con \j colore esattamente sesto diverso da \c.\\
 3 \j / 4 \j & Colore almeno settimo (meglio se a compagno passato).\\
 3 \sa & Minore almeno settimo chiuso (con A, K, Q) e non pi\`u di una dama negli altri tre colori.
\end{tabular}


\section{Risposte sull'apertura di 1 a colore}

Si risponde tendenzialmente con $\geq 5$ punti oppure con un Asso.

\subsection{Senza fit}

\begin{itemize}
 \item Se possibile, dire un colore a livello di 1 (il pi\`u lungo, e in caso di parit\`a come da algoritmo generale).
 \item Altrimenti:
\begin{center}
\begin{tabular}{p{0.1\textwidth} p{0.05\textwidth} p{0.7\textwidth}}
  $\leq 10$ HCP & 1 \sa & \\
  $\geq 11$ HCP & 2 \h & con le \h almeno quinte\\
  & 2 \d & Con le \d almeno quarte\\
  & 2 \c & In tutti gli altri casi\\
\end{tabular}
\end{center}
\end{itemize}


\subsection{Con fit su un colore maggiore \j}

\begin{tabular}{p{0.05\textwidth} p{0.7\textwidth}}
 2 \sa & 10+ HCP, fit almeno quarto.\\
 2 \j  & 7-10 HCP, fit almeno terzo (tendenzialmente proprio terzo).\\
 3 \j  & 7-9 HCP, fit almeno quarto.\\
 4 \j  & Appoggio lungo, mano scarsa (?).\\
\end{tabular}

Con un fit esattamente terzo e tanti punti, si pu\`o rispondere: 1\s (sull'apertura di 1\h, se si hanno le picche almeno quarte), 2\c, oppure con una cue-bid (a partire da 3\s?).


\subsection{Con fit sulle \d}

\begin{tabular}{p{0.09\textwidth} p{0.7\textwidth}}
 1 \h/\s & Se possibile, come nel caso senza fit.\\
 1 \sa & ??\\
 2 \d  & Fit scarso.\\
 3 \d  & Fit migliore (11-12 HCP).\\
 2 \c  & Forzante manche (13+ HCP).\\
\end{tabular}




\section{Risposte sull'apertura 1 SA}

\subsection{Mano sbilanciata - \textit{Transfer}}

Almeno sei carte in un colore (o anche cinque con singoli o vuoti con senno, o se hai pochi punti), \textit{transfer}, ovvero:

\begin{tabular}{p{0.09\textwidth} p{0.7\textwidth}}
 2 \d & Per le \h.\\
 2 \h & Per le \s.\\
 2 \s & Per le \c.\\
 3 \c & Per le \d.\\
\end{tabular}

\subsection{Mano bilanciata senza quarte nobili (e senza ambizioni di slam)}

\begin{tabular}{p{0.09\textwidth} p{0.7\textwidth}}
 2 \sa & 6-8 HCP, invitante manche.\\
 3 \sa & 9-12 HCP, a giocare.
\end{tabular}


\subsection{1 SA - 2\c (\textit{Stayman})}

Richiesta di quarte nobili, $\geq 7$ HCP. Risposte:\\
\begin{tabular}{p{0.09\textwidth} p{0.7\textwidth}}
 2 \d & Nessuna quarta nobile e minimo dell'apertura (15 HCP).\\
 2 \h & \h quarte, ma non \s quarte.\\
 2 \s & \s quarte, ma non \h quarte.\\
 2 SA & Nessuna quarta nobile e massimo dell'apertura (17 HCP).\\
 3 \c & 4\h\ - 4\s\ - 3\c\ - 2\d.\\
 3 \d & 4\h\ - 4\s\ - 3\d\ - 2\c.\\
\end{tabular}

Se il richiedente aveva una quinta nobile \j, ma non c'\`e fit quarto dell'apertore, dichiara il primo \j disponibile (con senno). Se l'apertore ha fit (terzo) inizia le \textit{cue-bid}, altrimenti licita \sa.

Con il fit su un nobile \j (dopo 2\j):

\begin{tabular}{p{0.09\textwidth} p{0.7\textwidth}}
 3 \j & Seleziona \j come atout. Invitante manche.\\
 4 \j & A giocare.\\
 \textit{Cue-bid} & A partire da 3\s; sottintende il nobile dell'apertore (2\h o 2\s) come atout, verso la slam.\\
\end{tabular}

Con il fit su un nobile \j (dopo 3\c o 3\d): il rispondente dichiara \j, e l'apertore inizia le \textit{cue-bid}.


Per giocare SA il rispondente licita:

\begin{tabular}{p{0.09\textwidth} p{0.7\textwidth}}
 2 \sa & (se possibile) 7-8 HCP, invitante manche.\\
 3 \sa & A giocare.\\
 4 \sa & Richiesta d'Assi.
\end{tabular}

Richiesta dei minori (dopo 2\d, 2\h, 2\s, 2\sa), forzante manche: 3\c. Risposte:
\begin{itemize}
 \item Se l'apertore ha negato quarte nobili (cioè dopo 2\d o 2\sa):
 
  \begin{tabular}{p{0.09\textwidth} p{0.7\textwidth}}
    3\d & \d almeno quarte, \c al massimo terze (5-3-3-2 oppure 4-3-3-3).\\
    3\h & 4\c\ - 4\d\ - 3\h\ - 2\s.\\
    3\s & 4\c\ - 4\d\ - 3\s\ - 2\h.\\
    3 \sa & \c almeno quarte, \d al massimo terze (5-3-3-2 oppure 4-3-3-3).
  \end{tabular}

 \item Se l'apertore ha un palo nobile quarto (cioè dopo 2\h o 2\s):
 
  \begin{tabular}{p{0.09\textwidth} p{0.7\textwidth}}
    3\d & \d quarte (4-4-3-2).\\
    3\h & Nessun minore quarto, mano minima.\\
    3\s & Nessun minore quarto, mano massima.\\
    3 \sa & \c quarte (4-4-3-2).
  \end{tabular}
\end{itemize}

Dopo la richiesta dei minori:
\begin{itemize}
 \item Per giocare a \sa: 3 \sa o passo, a giocare; 4 \sa richiesta d'Assi.
 \item Se c'è fit su un minore, dopo 3\d o 3 \sa (il fit non è ambiguo): \textit{cue-bid} nel primo maggiore disponibile, se possibile; altrimenti licitare il minore.
 L'apertore prosegue con la prima \textit{cue-bid}.
 \item Se c'è fit su un minore, dopo 3\h o 3\s nel primo caso: licitare il minore.
 L'apertore prosegue con la prima \textit{cue-bid}.
 \item Se si vuole segnalare un minore quinto \j (in cui l'apertore non ha mostrato quattro carte): dichiarare 4\j.
 L'apertore prosegue così:
 \begin{itemize}
  \item 4($\bigstar+1$) senza fit terzo nel minore. Il rispondente dichiara 4 \sa (richiesta d'Assi), oppure 4($\bigstar+2$) per costringere l'apertore a licitare 4 \sa (a giocare).
  \item La prima \textit{cue-bid} nobile disponibile (che non sia 4($\bigstar+1$)), senza superare 5\j.
  
 \end{itemize}
\end{itemize}



\section{Verso gli slam}

\subsection{\it{Cue-bid}}

Non si gioca se il contratto sar\`a SA, e comunque occorre avere gi\`a stabilito l'atout. A partire da 3\s e salendo gradualmente si dichiara un seme \j (che non sia l'atout) se si possiede una delle seguenti (\textit{cue-bid} debole):
\begin{itemize}
\item Asso di \j.
\item K di \j.
\item Esattamente una carta di \j.
\item Nessuna carta di \j.
\end{itemize}

Saltare un seme disponibile \`e come dichiarare che non si ha nessuna delle quattro condizioni precedenti in quel seme. Dichiarare SA \`e come dichiarare l'ultimo seme detto dal compagno, ma sono esclusi 4 SA e 5 SA. L'unica possibilit\`a \`e quindi 3\s\,- 3 SA, che dichiara \textit{cue-bid} (debole) a picche per entrambi i giocatori. Se un giocatore ripete un seme \j che egli stesso ha gi\`a dichiarato possiede una delle seguenti (\textit{cue-bid} forte):
\begin{itemize}
\item Asso di \j.
\item Nessuna carta di \j.
\end{itemize}

\subsection{Richiesta d'Assi e richiesta di Re}

\paragraph{Con atout.} Ai fini delle richieste d'Assi e di Re il K di atout \`e considerato un Asso. In questo caso quindi ci sono 5 Assi e 3 Re. Risposte alla richiesta d'Assi (4 \sa):
 
\begin{tabular}{p{0.05\textwidth} p{0.7\textwidth}}
5 \c & 0 o 3 Assi.\\
5 \d & 1 o 4 Assi.\\
5 \h & 2 o 5 Assi, senza Q di atout.\\
5 \s & 2 o 5 Assi, con Q di atout.\\
\end{tabular}

Se si riceve una risposta che non dice nulla sulla Q di atout, si può rialzare di 1 per chiedere la Q di atout (di 2 se rialzando di 1 si cade sull'atout). La risposta è la seguente: si rialza di 1 per negare la Q, di 2 per dare la Q.

Inoltre, con 5 \sa si può effettuare la richiesta di Re. Risposte:

\begin{tabular}{p{0.05\textwidth} p{0.7\textwidth}}
6 \c & nessun Re.\\
6 \d & 1 Re.\\
6 \h & 2 Re.\\
6 \s & 3 Re. \\
\end{tabular}


\paragraph{Senza atout.} Risposte alla richiesta d'Assi (4 \sa):

\begin{tabular}{p{0.05\textwidth} p{0.7\textwidth}}
5 \c & 0 o 3 Assi.\\
5 \d & 1 o 4 Assi.\\
5 \h & 2 Assi.\\
\end{tabular}

Il richiedente può successivamente licitare 5\s per costringere il rispondente a licitare 5 \sa (a giocare).

In alternativa si può effettuare la richiesta di Re, dichiarando 5 \sa. Risposte:

\begin{tabular}{p{0.05\textwidth} p{0.7\textwidth}}
6 \c & Nessun Re.\\
6 \d & 1 Re.\\
6 \h & 2 Re.\\
6 \s & 3 Re. \\
6 \sa & 4 Re.
\end{tabular}





\end{document}
