\documentclass[a4paper,10pt]{article}


\usepackage[utf8]{inputenc}
\usepackage[italian]{babel}
\usepackage[pdftex]{graphicx}
\usepackage{amsfonts}
\usepackage{amsmath}
\usepackage{amssymb}
\usepackage{amsthm}
\usepackage{color}
\usepackage{fancyhdr}
\usepackage{hyperref}
\usepackage{xspace}
% \usepackage[parfill]{parskip}

\topmargin -1cm
\oddsidemargin -0.5cm
\textwidth 17cm

\setlength{\parindent}{0 pt} % Default 15 pt.
\setlength{\parskip}{0.15 cm} % Default 0 cm?

\renewcommand{\c}{$\clubsuit$\xspace}
\renewcommand{\d}{$\diamondsuit$\xspace}
\newcommand{\h}{$\heartsuit$\xspace}
\newcommand{\s}{$\spadesuit$\xspace}
\renewcommand{\j}{$\bigstar$\xspace}
\newcommand{\sa}{SA\xspace}


\newcommand{\smallspace}{\vskip0.3cm}

\renewcommand{\tabcolsep}{0.3cm}

\newcommand{\note}[1]{\textcolor{red}{#1}}


\newenvironment{twocol}
  {\smallspace\noindent\begin{tabular}{l p{0.78\textwidth}}}
  {\end{tabular}\smallspace}

\newenvironment{twocolind}
  {\smallspace\noindent\begin{tabular}{l p{0.68\textwidth}}}
  {\end{tabular}\smallspace}

\newenvironment{threecol}
  {\smallspace\noindent\begin{tabular}{l l p{0.78\textwidth}}}
  {\end{tabular}\smallspace}

  
% Title Page
\title{Convenzioni di bridge}
\author{Ugo Bindini, Alessandro Iraci, Giovanni Paolini}

\begin{document}
\maketitle

\tableofcontents

\pagebreak
\section{Aperture}

\paragraph{Algoritmo generale.}
Si licita il colore più lungo, escludendo \s e \h se non sono almeno quinte e le \d se non sono almeno quarte.
In caso di parità di lunghezza dei semi più lunghi, si dichiara il più alto se sono almeno quinti e si dichiara il più basso se sono quarti.
Nota: con 4\s\ - 4\h\ - 3\d\ - 2\c, si apre di 1 \c.

\noindent Le aperture standard (con almeno 12 HCP) si possono effettuare anche con 11 HCP se si posseggono due Assi e un Re.

\subsection{Mani bilanciate}
(4-3-3-3, 4-4-3-2, 5-3-3-2 con quinta minore)
\smallspace

\begin{threecol}
 12-14 & 1 \j & Si licita seguendo l'algoritmo generale. \\
 15-17 & 1 \sa & \\
 18-19 & 1 \c & Poi \sa a salto sulla risposta a livello di 1 (ma dare un fit nobile ha precedenza). \\
 20-22 & 2 \sa & \\
 23+ & 2 \c & Forzante manche. \\
\end{threecol}


\subsection{Mani monocolore}
(5-3-3-2 con quinta nobile, 6-3-2-2, 6-3-3-1, 7-3-2-1, \dots)

\begin{threecol}
 12+ & 1 \j & \j è il colore lungo. Poi si ripete il colore, eventualmente a salto (per dare Rever).\\
 & 2 \c & Tanti punti, forzante manche.
\end{threecol}


\subsection{Mani bicolore}
(5-4-3-1, 5-4-2-2, 6-4-2-1, 5-5-2-1, \dots)

\begin{threecol}
 12+ & 1 \j & \j è il colore più lungo; in caso di parità (5-5 o 6-6) si licita il colore più alto (in accordo con l'algoritmo generale). Sulla ridichiarazione, per licitare il secondo colore a livello di 2 in modo ascendente (es: 1 \d\ - 1 \s\ - 2 \h), bisogna avere 16+ HCP (Rever).\\
 & 2 \c & Tanti punti, forzante manche.
\end{threecol}


\subsection{Mani tricolore}
(4-4-4-1, 5-4-4-0)

\begin{threecol}
 12+ & 1 \j & Si licita seguendo l'algoritmo generale, con un'eccezione: se si ha una 4-4-4-1 con singolo a \s e meno di 16 HCP, si apre di 1 \d anziché 1 \c (in modo che, sull'eventuale risposta di 1 \s, si possa licitare 2 \c).\\
 & 2 \c & Tanti punti, forzante manche.
\end{threecol}


\subsection{Sottoaperture}

\begin{twocol}
 2 \j & 6-10 HCP, con \j colore esattamente sesto diverso da \c.\\
 3/4 \j & Colore almeno settimo con 6-8 HCP. Con 9-11 HCP solo a compagno passato.\\
\end{twocol}



\pagebreak

\section{Risposte sull'apertura di 1 a colore}

Si risponde tendenzialmente con 5+ HCP oppure con un Asso.

\paragraph{Sviluppi ``naturali''.}
Nel caso in cui la dichiarazione che si sta effettuando sia naturale, valgono i seguenti princ\^ipi.
\begin{itemize}
 \item Licitare un colore nuovo garantisce almeno 4 carte (almeno 5 se il compagno ha dimostrato di non averne più di 3). Osservazione importante: spesso licitando un nuovo colore si allunga il colore di apertura (vedi esempi).
 
 Per esempio: 1 \d\ - 1 \sa\ - 2 \h dà almeno 6 carte di \d e 5 di \h (e Rever), perché la risposta di 1 \sa nega quarte nobili;
 1 \d\ - 1 \s\ - 2 \c dà almeno 4 carte di \d e di \c (non almeno 5 di \d perché si potrebbe essere nel caso della 4-4-4-1 con 12-15 HCP e singolo a \s, vedi apertura tricolore);
 1 \d\ - 1 \h\ - 2 \c dà almeno 5 carte di \d e 4 di \c (non si può essere nel caso della 4-4-4-1 con 12-15 HCP e singolo a \s, perché bisognerebbe dare il fit a \h licitando 2 \h);
 1 \d\ - 1 \h\ - 1 \s dà almeno 4 carte di \d e di \s;
 1 \d\ - 1 \s\ - \hbox{2 \h} dà Rever e almeno 4 carte di \d e di \h (le \h possono essere anche solo quarte, perché il rispondente può avere 4 carte di \h\ -- benché ciò sia improbabile);
 1 \c\ - 2 \c\ - 2 \h dà almeno 5 carte di \h (perché il rispondente ha negato colori quarti diversi dalle \c) e quindi almeno 6 carte di \c.
 
 \item Ripetere un colore allunga di 1 il numero di carte.
 
 Esempio: 1 \h\ - 1 \sa\ - 2 \s\ - 2 \sa\ - 3 \h dà Rever e almeno 7 carte a \h e 5 a \s (infatti 2 \s garantisce almeno 5 carte di \s perché il rispondente ne ha al massimo 3, per cui l'apertura di 1 \h diventa garanzia di almeno 6 carte di \h).
 
 {\bf Attenzione:} il caso in cui l'apertore apre di 1 \c e poi ripete le \c è un po' particolare, perché la prima dichiarazione garantisce solo 2 carte a \c. In ogni caso, la ripetizione delle \c garantisce almeno 5 carte, perché non vale la pena di ripetere le \c se si hanno solo 4 carte (piuttosto si licita \sa, che tendenzialmente fa capire che si hanno 4 carte a \c dal momento che non se ne hanno negli altri semi).
 
 Per esempio, 1 \c\ - 1 \h\ - 2 \c dà meno di 16 HCP e almeno 5 carte di \c (se fossero solo 4, ad esempio nella distribuzione 4\c\ - 4\d\ - 3\h\ - 2\s, si liciterebbe invece 1 \sa).
 Invece 1 \c\ - 1 \h\ - 3 \c dà Rever e almeno 6 carte di \c (infatti, se le carte di \c fossero solo 5, o si avrebbe un altro palo da licitare oppure la mano sarebbe bilanciata).
 
%  \item Appoggiare un colore licitato dal compagno dà il fit (almeno $8-k$ carte, dove $k$ è il numero di carte che sei sicuro che il tuo compagno abbia in quel colore).
 \item Licitare un colore nobile ha la priorità sul dare il fit in un colore minore. Dare il fit su un colore nobile ha la priorità su qualsiasi altra cosa.
 
 Per esempio, se l'apertore ha 4\c\ - 4\d\ - 4\h\ - 1\s e ha aperto di 1 \c (quindi ha 16+ HCP, vedi apertura tricolore), sulla risposta di 1 \d dichiara 2 \h (Rever, \h quarte) e non 3 \d (Rever, fit a \d).
 Il fit a \d può essere dato dopo, nel caso in cui non si trovi fit a \h.
 
 \item {\bf Rever (1).} Solo per l'apertore: dopo una risposta a livello di 1, superare strettamente 2 \j (dove \j è il colore di apertura) dà Rever, e va fatto se e solo se si hanno 16+ HCP.
 In questo modo, dopo la seconda licita dell'apertore, il rispondente sa se l'apertore ha 12-15 HCP oppure 16+ HCP.
 
 {\bf Eccezione:} se l'apertore vuole dare il fit nel colore del rispondente, lo fa a livello per non dare Rever, e lo fa a salto per dare Rever. In particolare, 1 \h\ - 1 \s\ - 2 \s non è Rever.
 
 {\bf In che colore dare Rever.} Se si vuole dare Rever (16+ HCP), allora ci si dovrebbe trovare in uno dei seguenti casi: mano bilanciata (allora si è seguito l'algoritmo per le aperture con mano bilanciata); mano monocolore (si hanno almeno 6 carte nel colore, quindi si ripete quello); mano bicolore o tricolore (si licita uno degli altri colori). Purtroppo non sempre ci si trova in questi casi.
 Ad esempio si può avere una 5-3-3-2 con quinta nobile e 16-17 HCP (in questo caso si dà Rever nel migliore dei minori, anche se è solo terzo);
 oppure si può essere nel caso 1 \h\ - 1 \sa con cinque carte di \h e quattro di \s, in cui non si possono dire le \s perché il rispondente ne ha meno di 4 (anche in questo caso si dà Rever nel migliore dei minori -- anche se è solo secondo -- oppure con non tanti punti si rinuncia al Rever dichiarando 2 \sa, che è invitante manche).
 
 {\bf Morale della favola:} il Rever in un colore minore talvolta non garantisce 4 carte nel colore. È compito del rispondente ragionare e capire se ci si può trovare in uno di questi casi.
 
 Esempi di Rever: 1 \d\ - 1 \h\ - 2 \s (con \d e \s almeno quarte); 1 \h\ - 1 \sa\ - 3 \h (con \h almeno seste); \hbox{1 \h}\ - 1 \s\ - 3 \s (con \h almeno quinte e \s almeno quarte).
 
 Altro esempio: se l'apertore ha meno di 16 HCP, 6 carte di \h e 5 carte di \s, su 1 \h\ - 1 \sa deve rispondere 2 \h e non 2 \s. Questo è un po' triste, però non ci si può fare nulla.
 
 Ulteriore esempio: su 1 \d\ - 1 \s, con la distribuzione 4\d\ - 4\h\ - 3\s\ - 2\c e meno di 16 HCP, l'apertore licita 1 \sa e non 2 \h.
 
 \item {\bf Rever (2).} Anche dopo una risposta di 2 a colore che nega il fit nel colore di apertura, l'apertore deve decidere se: dare o meno il fit; dare o meno Rever.
 \begin{itemize}
  \item Per dare il fit senza dare Rever, si ripete a livello il colore del compagno. Nel caso eccezionale 1 \s\ - 2 \h, l'apertore con un fit almeno terzo può dire 3 \h con il minimo (circa 12 HCP) e 4 \h negli altri casi (circa 13-15 HCP, oppure 12 HCP con una mano sbilanciata e/o fit almeno quarto). In questo modo il rispondente può passare 3 \h nel caso in cui abbia a sua volta il minimo.
  
  Attenzione: prima di dare il fit in un minore, è bene essere ragionevolmente sicuri che non si stia perdendo il fit in un colore nobile.
  
  Esempi: 1 \s\ - 2 \d\ - 3 \d (l'apertore ha 12-15 HCP, fit quarto a \d, e tendenzialmente esattamente 5 carte di \s e al massimo 3 carte di \h);
  1 \h\ - 2 \d\ - 3 \d (l'apertore ha 12-15 HCP, fit quarto a \d, e tendenzialmente esattamente 5 carte di \h e al massimo 4 di \s\ -- infatti il rispondente ha negato 4 carte di \s, per cui l'apertore licita 2 \s solo se ne ha almeno 5).
  
  \item Per non dare il fit ma dare Rever: distinguiamo i vari casi.
  
  Nel caso di una distribuzione monocolore, si ripete il proprio colore a salto, cioè a livello di 3 (es: \mbox{1 \s}\ - 2 \d\ - 3 \s). Nel caso particolare 1 \c\ - 2 \c, l'apertore licita 4 \c (ha almeno 6 carte di fiori,\footnote{Con meno di 6 carte di \c, l'apertore avrebbe una mano bilanciata oppure bicolore.} il rispondente ne ha almeno 4 con 12 HCP; si giocherà a fiori).
  
  Nel caso di una distribuzione bicolore o tricolore, si dice il secondo colore a livello di 3 (es: 1 \s\ - \mbox{2 \c}\ - 3 \d, oppure 1 \s\ - 2 \d\ - 3 \c).
  
  Rimane il caso di una distribuzione bilanciata, che (data l'apertura di 1 a colore) deve rientrare in uno dei seguenti due casi: 5-3-3-2 con quinta nobile e 15-17 HCP (in realtà 16-17, se si vuole dare Rever) oppure una qualsiasi bilanciata con 18-19 HCP. Nel primo caso, si dà Rever licitando un minore a livello di 3 (es: 1 \s\ - 2 \d\ - 3 \c), scegliendo il migliore tra i minori non licitati dal compagno (in questo modo si lasciano libere le dichiarazioni di 2 \sa, invitante manche, e 3 \sa, con circa 14-15 HCP o forse anche con 13 HCP).
  Nel secondo caso la licita è stata 1 \c\ - 2 \c, e l'apertore licita 2 \sa per segnalare la forza della mano (non è un invito, perché entrambi sanno che la manche è assicurata!).
  
  \item Per non dare né fit né Rever: si licita un colore a livello di 2 (quello di apertura o un altro), oppure 2 \sa (12 HCP oppure 13 brutti) o 3 \sa.
  
  \item Per dare sia fit che Rever: si licita la prima cue-bid, a partire da 4 \c.
  
  Attenzione: non è detto che sia ragionevole dare il fit in un minore in questo modo. Infatti, le cue-bid rendono più complicato giocare a \sa (manche o slam). A seconda dei casi, può essere preferibile dare Rever fingendo di non avere il fit (soprattutto se la mano è bilanciata e il fit è corto).
 \end{itemize}

\end{itemize}


\subsection{Senza fit}

Se possibile, dire un colore almeno quarto a livello di 1 (il più lungo, e in caso di parità come da algoritmo generale). Altrimenti, si licita come segue.

Su 1 \d, 1 \h, 1 \s:
\begin{twocol}
	1 \sa & $\leq 10$ HCP.\\
	2 \h & 11+ HCP, con le \h almeno quinte (solo dopo 1 \s).\\
	2 \d & 11+ HCP, con le \d almeno quarte, se non si è nel caso precedente.\\
	2 \c & 11+ HCP, in tutti gli altri casi.\\
\end{twocol}

Su 1 \c:
\begin{twocol}
	1 \d & 5-6 HCP, anche con zero carte di \d! \\
	1 \sa & 7-11 HCP. \\
	2 \c & 12+ HCP. \\
\end{twocol}

\paragraph{Dopo 1 \c\ - 2 \c.} Dopo la risposta di 2 \c, l'apertore licita come segue.
%TODO: Questa cosa è provvisoria, va discussa.

\begin{twocol}
	2 \d & 16-18 HCP, mano non bilanciata (e quindi fit a \c). \\
	2 \h & 6-5 \c/\h, senza rever. \\
	2 \s & 6-5 \c/\s, senza rever. \\
	2 SA & 12-14 HCP, mano bilanciata. \\
	3 \c & 19+ HCP, \c almeno quinte (oppure 4-4-4-1 con singolo a \h). \\
	3 \d & Mano non bilanciata, senza rever. Richiesta di fermo a \d. \\
	3 \h & Mano non bilanciata, senza rever. Richiesta di fermo a \h. Garantisce il fermo a \d. \\
	3 \s & Mano non bilanciata, senza rever. Richiesta di fermo a \s. Garantisce i fermi a \d e \h. \\
	3 \sa & 18-19 HCP, mano bilanciata. \\
\end{twocol}

Dopo 3 \d/\h/\s, il rispondente licita come segue.

\begin{twocol}
	3 \h & Solo su 3 \d. Fermo a \d, richiesta di fermo a \h. \\
	3 \s & Solo su 3 \d/\h. Fermo a \d/\h, richiesta di fermo a \s. \\
	3 \sa & A giocare. Garantisce tutti i fermi. \\
	4 \c & Nega il fermo, 12-13 HCP. \\
	4 \d/\h/\s & Cue-bid, nega il fermo (?), interesse di slam a \c. \\
	4 \sa & Richiesta d'Assi (con atout \c). \\
	5 \c & Nega il fermo, 14+ HCP, senza ambizioni di slam a \c. \\
\end{twocol}

In generale, ad una richiesta di fermo si risponde affermativamente con un'altra richiesta di fermo (la più economica) o con 3 \sa, negativamente con 4 \c (mano minima) o 5 \c (mano massima). 


\paragraph{Dopo il terzo colore licitato a livello di 1.} Per esempio: 1 \c\ - 1 \d\ - 1 \s.
L'apertore non ha rever, quindi ha 12-15 HCP.
Se il rispondente vuole dare il fit nel secondo colore \j licitato dall'apertore (che è nobile!), dichiara:
\begin{twocol}
 Pass & $\leq$ 8 HCP.\\
 2 \j  & 9-10 HCP.\\
 3 \j & 11 HCP.\\
 Cue-bid  & Forzante manche.\\
\end{twocol}



\subsection{Con fit su un colore maggiore \j}
\begin{twocol}
 2 \j  & 5-9 HCP, fit terzo.\\
 3 \c  & 8-9 HCP, fit almeno quarto.\\
 3 \d  & 6-7 HCP, fit almeno quarto.\\
 3 \j  & 0-5 HCP, fit almeno quarto (barrage, non se gli avversari hanno passato).\\
%  4 \j  & 5-7 HCP, fit almeno quinto.\\
 2 \sa & 10+ HCP, fit almeno terzo (Jacoby).\\
 3 \sa & Richiesta d'Assi, forzante manche. \\
 Cue-bid & Fit almeno terzo, forzante manche. Le Cue-bid partono da 3 \h/\s.
\end{twocol}
% \note{Splinter?}
Dopo la risposta di 2 \j, l'apertore può licitare 2 \sa per chiedere ulteriore descrizione della mano. Seguono le risposte.

\begin{twocol}
	3 \c & 8-9 HCP. \\
	3 \d & 6-7 HCP. \\
	3 \j & 4-5 HCP.
\end{twocol}

\paragraph{Sviluppo su 2 \sa (Jacoby).} Su 1 \s\ - 2 \sa, l'apertore licita
\begin{twocol}
 3 \c & 14-15 HCP, singolo o vuoto a \c.\\
 3 \d & 14-15 HCP, singolo o vuoto a \d.\\
 3 \h & 14-15 HCP, singolo o vuoto a \h.\\
 3 \s & Mano minima (12-13 HCP).\\
 3 \sa & 16+ HCP, senza singoli né vuoti.\\
 4 \c & 16+ HCP, singolo o vuoto a \c.\\
 4 \d & 16+ HCP, singolo o vuoto a \d.\\
 4 \h & 16+ HCP, singolo o vuoto a \h.\\
 4 \s & 14-15 HCP, senza singoli né vuoti.
\end{twocol}

\noindent Su 1 \h\ - 2 \sa semplicemente si scambiano \h e \s nelle risposte precedenti, escludendo però 4 \s: in caso di singolo o vuoto a \s con 16+ HCP si licita 3 \sa.

Dopo la risposta su 2 \sa, il rispondente può: chiudere a 3 \j o 4 \j (a giocare), iniziare le Cue-bid (direttamente dalla licita più economica disponibile, diversa da \j e da 4 \sa) oppure chiedere gli assi licitando 4 \sa. La licita di 3 \sa in questo caso conta come Cue-bid nell'ultimo seme licitato dall'apertore, che nega le Cue-bid saltate (es: 1 \s\ - \mbox{2 \sa}\ - 3 \c\ - 3 \sa dà Cue-bid a \c e la nega a \d e \h).

Attenzione: le risposte che garantiscono un singolo o un vuoto danno implicitamente una Cue-bid debole nel colore (questo è importante nel caso in cui, successivamente, si inizino le Cue-bid).

\subsection{Con fit sulle \d}

\begin{twocol}
 1 \h/\s & Se possibile, come nel caso senza fit.\\
 2 \c  & 11+ HCP, come nel caso senza fit.\\
 2 \d  & 5-8 HCP.\\
 3 \d  & 9-10 HCP.\\
\end{twocol}


\paragraph{Sviluppo su 2 \d.}
Su 1 \d\ - 2 \d, l'apertore licita
\begin{twocol}
  Pass & Senza rever (12-15 HCP).\\
  2 \h/\s & Richiesta di fermo a \h/\s. Il rispondente licita 2/3 \sa con fermo e mano minima/massima, 3 \c/\d senza fermo e con mano massima/minima.\\
  2 \sa & 17-18 HCP. Invitante manche a \sa.\\
  3 \c & Richiesta di fermo a \c, 19+ HCP. Il rispondente licita 3 \sa con fermo e 3 \d senza fermo.\\
  3 \d & Invitante manche a \d.\\
  3 \h/\s & Cue-bid.\\
  3 \sa & Passabile, ma il rispondente può correggere a 4 \d con mano massima e sbilanciata (forzante manche a \d).\\
  4 \c & Cue-bid.\\
  4 \sa & Richiesta d'assi.
\end{twocol}



\subsection{Con fit sulle \c (almeno seste)}

\begin{twocol}
 1 \d/\h/\s & Se possibile, come nel caso senza fit.\\
 2 \c & 12+ HCP, come nel caso senza fit.\\
 2 \sa & 5-8 HCP (barrage).\\
 3 \c & 9-11 HCP.\\
\end{twocol}



\subsection{Sviluppi dopo la risposta di 1 a colore maggiore}

\paragraph{Senza rever.} Licitando 2 nel colore del rispondente si dà il fit. Altrimenti 1 \sa oppure (se possibile) un nuovo colore.

Dopo 1 colore - 1 colore - 1 colore, la risposta di 1 \sa esclude la manche ($\leq$ 7-8 HCP), mentre la risposta di 2 \sa è invitante manche a \sa.

Dopo 1 colore - 1 \h/\s\ - 1 \sa, la licita di 2 \c (Roudi) chiede ulteriore descrizione della mano. L'apertore licita come segue.
\begin{twocol}
  2 \d & 12-13 HCP, nega la terza di \h/\s.\\
  2 \h & 12-13 HCP, terza di \h/\s.\\
  2 \s & 14-15 HCP, terza di \h/\s.\\
  2 \sa & 14-15 HCP, nega la terza di \h/\s.
\end{twocol}


\paragraph{Con rever.} Dopo 1 \c\ - 1 \h si licita
\begin{twocol}
 2 \d & \d quarte, senza fit.\\
 2 \s & \s quarte, senza fit.\\
 2 \sa & 18-19 HCP, mano bilanciata \emph{senza fit} - vedi aperture bilanciate.\\
 3 \c & 16-18 HCP, \c seste, senza fit.\\
 3 \d & 19+ HCP, \c seste, senza fit.\\
 3 \h & Fit a \h, 16-18 HCP, invitante.\\
 3 \s\ / 4 \c / 4 \d & Cue-bid. 19+ HCP, fit a \h, forzante manche.\\
\end{twocol}

\noindent Se il rispondente aveva licitato 1 \s, semplicemente si scambiano i ruoli di \h e \s.
\smallspace
\noindent Dopo 1 \d\ - 1 \h si licita
\begin{twocol}
 2 \s & \s quarte, senza fit.\\
 2 \sa & 16-18 HCP, senza fit, senza singoli né vuoti, \d al massimo seste (5-4-2-2, 6-3-2-2).\\
 3 \c & \c quarte (bicolore \d/\c), senza fit, con almeno un singolo o vuoto.\\
 3 \d & \d seste con almeno un singolo o vuoto, oppure \d settime, senza fit.\\
 3 \h & Fit a \h, 16-18 HCP, invitante.\\
 3 \s\ / 4 \c\ / 4 \d & Cue-bid. 19+ HCP, fit a \h, forzante manche.\\
 3 \sa & 19+ HCP, senza fit, senza singoli né vuoti, \d al massimo seste (5-4-2-2, 6-3-2-2).
\end{twocol}

\noindent Se il rispondente aveva licitato 1 \s, semplicemente si scambiano i ruoli di \h e \s.
\smallspace
\noindent Dopo 1 \h\ - 1 \s si licita
\begin{twocol}
 2 \sa & 16-18 HCP, mano quasi bilanciata, invito a giocare a \sa (senza fit).\\
 3 \c & \c quarte, senza fit (19+ oppure mano sbilanciata).\\
 3 \d & \d quinte, senza fit (19+ oppure mano sbilanciata).\\
 3 \h & \h seste, senza fit (19+ oppure mano sbilanciata).\\
 3 \s & Fit a \s, 16-18 HCP, invitante.\\
 3 \sa & 19+ HCP, mano quasi bilanciata (ovvero 5-3-3-2).\\
 4 \c\ / 4 \d\ / 4 \h & Cue-bid. 19+ HCP, fit a \s, forzante manche.\\
\end{twocol}

\paragraph{Sviluppi dopo un rever a livello di 2 (a colore).} Il rispondente può licitare 2 \sa con mano minima (Ingberman), obbligando l'apertore a licitare 3 \c con mano minima (16-18 HCP). Dopo questa licita di 3 \c, il rispondente può scegliere l'atout (eventualmente passando) e si gioca un contratto parziale.



\subsection{Sviluppi dopo la risposta di 1 \sa}

Sull'apertura di 1 \d, 1 \h o 1 \s:
\begin{twocol}
 2 \c & Rever generico con 16-18 HCP.\\
 2 \sa & 14-15 HCP, invitante manche a \sa.\\
 Altro a livello di 2 & Come da algoritmo generale.\\
 Qualcosa a livello di 3 & Come da algoritmo generale, ma con 19+ HCP.
\end{twocol}

\noindent Sull'apertura di 1 \c:
\begin{twocol}
  2 \c & 13-14 HCP, invitante manche a \sa.\\
  2 \sa & 15-16 HCP, invitante manche a \sa.\\
  3 \sa & 17+ HCP, da passare.\\
  Altro & Come da algoritmo generale.
\end{twocol}




\pagebreak

\section{Risposte sull'apertura 1 SA}

\subsection{Mano sbilanciata (Transfer)}

Almeno sei carte in un colore (o anche cinque con singoli o vuoti con senno, o con pochi punti).

\begin{twocol}
 2 \d & Transfer per le \h.\\
 2 \h & Transfer per le \s.\\
 2 \s & Transfer per le \c.\\
 3 \c & Transfer per le \d.\\
\end{twocol}

L'apertore deve licitare il seme richiesto dal rispondente (che diventa automaticamente l'atout concordata), dopodiché il rispondente può proseguire passando, licitando manche o slam (a giocare), iniziando le Cue-bid, oppure chiedendo gli Assi (4 \sa).


\subsection{Mano bilanciata, senza quarte nobili e senza ambizioni di slam}

\begin{twocol}
 2 \sa & 7-8 HCP, invitante manche.\\
 3 \sa & 9-12 HCP, a giocare.
\end{twocol}


\subsection{1 SA - 2 \c (Stayman)}

Richiesta di quarte nobili, 8+ HCP (oppure 7 con senno). Seguono le risposte.

\begin{twocol}
 2 \d & Nessuna quarta nobile. \\
 2 \h & \h quarte, ma non \s quarte. \\
 2 \s & \s quarte, ma non \h quarte. \\
 2 \sa & Entrambe le quarte nobili. \\
 3 \c & 5-3-3-2 con la quinta di \c, mano massima. \\
 3 \d & 5-3-3-2 con la quinta di \d, mano massima. \\
\end{twocol}

\noindent Esaminiamo gli sviluppi della Stayman.

\paragraph{Fit in un colore nobile.}

Con il fit su un nobile \j, dopo la risposta di 2 \j.
\begin{twocol}
 3 \j & Seleziona \j come atout. Invitante manche. \\
 Cue-bid & A partire da 3 \c. Sottintende il nobile dell'apertore come atout. \\
 4 \j & A giocare. \\
\end{twocol}

Con il fit su un nobile dopo la risposta di 2 \sa.

\begin{twocol}
	3 \c & Interesse di slam. \\
	3 \d & Transfer per le \h, 8 HCP. \\
	3 \h & Transfer per le \s, 8 HCP. \\
	4 \d & Transfer per le \h, 9+ HCP senza interesse di slam. \\
	4 \h & Transfer per le \s, 9+ HCP senza interesse di slam. \\
\end{twocol}

Dopo 3 \c, l'apertore è obbligato a licitare 3 \d, il rispondente fissa l'atout con 3 \h/\s, quindi iniziano le cue-bid.
Dopo 3 \d/\h, l'apertore licita 3 \h/\s (da passare) con mano minima, e 4 \h/\s con mano massima (a giocare). Dopo 4 \d/\h, l'apertore licita 4 \h/\s (a giocare).

\paragraph{Richiesta dell'altro nobile.}
Sulla risposta di 2 \h, se il rispondente ha cinque carte di \s ma non quattro carte di \h, licita 2 \s. L'apertore licita come segue.

\begin{twocol}
	2 \sa & Nega il fit a \s, mano minima. \\
	3 \c/\d/\h & Cue-bid, fit a \s, mano massima. \\
	3 \s & Fit a \s, mano minima. \\
	3 \sa & Nega il fit a \s, mano massima.
\end{twocol}

Dopo 1 \sa \xspace- 2 \c \xspace- 2 \s, se il rispondente ha cinque carte di \h ma non quattro carte di \s, licita 2 \sa con mano minima, e 3 \h altrimenti (forzante manche).

Su 1 \sa \xspace- 2 \c \xspace- 2 \s \xspace- 2 \sa, l'apertore licita come segue.

\begin{twocol}
	Pass & Mano minima. \\
	3 \h & Mano massima, terza di \h. Invita a scegliere fra 3 \sa e 4 \h. \\
	3 \sa & Mano massima, nega la terza di \h. \\
\end{twocol}

Dopo la licita di 3 \h, il rispondente corregge a 3 \sa senza la quinta di \h, e a 4 \h con la quinta di \h.

\paragraph{Gioco a SA.}
Per giocare a SA il rispondente licita:
\begin{twocol}
 2 \sa & (se possibile) 7-8 HCP, invitante manche.\\
 3 \sa & A giocare.\\
 4 \c & Richiesta d'Assi (Gerber).
\end{twocol}

\paragraph{Fit miracoloso nel minore quinto.}
Dopo la risposta di 3 \c/\d, il rispondente può confermare tale atout licitando una Cue-bid (a partire da 3 \d\ / 4 \h, con senno).


\paragraph{Ricerca del fit in un minore.}
Dopo le risposte di 2 \d, 2 \h o 2 \s, il rispondente può cercare una quarta minore licitando il minore. L'apertore dà la quarta licitando una Cue-bid e la nega licitando 3 \sa.



\pagebreak

\section{Sviluppi sulle aperture forti}


\subsection{Aperture forti bilanciate}

\paragraph{Transfer.} Sull'apertura 2 \sa, con mano sbilanciata (seme almeno sesto) il rispondente può licitare una Transfer (anche 4 \c, che in questo caso non è una Gerber).

\paragraph{Stayman avanzata.} Su 2 \sa (20-22 HCP), oppure 1 \c\ - 1 \j\ - 2 \sa (18-19 HCP), il rispondente può giocare 3 \c (Stayman avanzata). L'apertore risponde come segue.
\begin{twocol}
 3 \d & Nessuna quarta nobile.\\
 3 \h & \h quarte, ma non \s quarte.\\
 3 \s & \s quarte, ma non \h quarte.\\
 3 \sa & Entrambe le quarte nobili.\\
\end{twocol}

\noindent Distinguiamo ora i vari casi.
\begin{itemize}
  \item Risposta di 3 \sa.
  \begin{twocol}
    Pass & \\
    4 \c & Gerber (per giocare a \sa).\\
    4 \d & Obbliga a dichiarare 4 \h (per giocare a \h).\\
    4 \h & Obbliga a dichiarare 4 \s (per giocare a \s).\\
  \end{twocol}
  
  \item Risposta di 3 \h/\s.
  \begin{twocol}
    3 \sa & A giocare.\\
    4 \c & Gerber.\\
%     4 \d & Ho una quinta nell'altro maggiore \j! L'apertore risponde 4 \j se ha la terza, altrimenti risponde licitando l'altro maggiore. In quest'ultimo caso, 4 \sa è a giocare e 5 \c è una 5-Gerber.\\
    4 \h/\s & A giocare.\\
    4 \sa & RKCB (sottintendendo il fit).
  \end{twocol}
  
  \item Risposta di 3 \d.
  \begin{twocol}
    3 \h & Quinta di \h. L'apertore risponde 3 \sa se non ha il fit, altrimenti con la prima Cue-bid.\\
    3 \s & Quinta di \s. L'apertore risponde 3 \sa se non ha il fit, altrimenti con la prima Cue-bid.\\
    3 \sa & A giocare.\\
    4 \c & Gerber.\\
    4 \d & Entrambe le quinte maggiori. L'apertore risponde con il suo maggiore più lungo (sarà un fit).\\
  \end{twocol}
\end{itemize}



\subsection{Apertura di 2 \c}

Ricordiamo che l'apertura di 2 \c si effettua con 23+ HCP e mano bilanciata, oppure con una mano (anche sbilanciata) con cui si può giocare la manche anche se il compagno ha delle pessime carte.
È l'unica apertura sulla quale il compagno è obbligato a rispondere, anche con 0 HCP.
Come filosofia generale, è il compagno dell'apertore a condurre la dichiarazione.
In ogni caso, dopo l'apertura di 2 \c ogni licita sotto la manche è automaticamente forzante (con unica eccezione 2 \c\ - 2 \d\ - 2 \sa, vedi oltre).

Seguono le risposte.
\begin{twocol}
  2 \d & 0-6 HCP.\\
  2 \h/\s & 7+ HCP, nobile almeno quarto.\\
  2 \sa & 7+ HCP, mano bilanciata (nobili al più terzi, minori al più quarti).\\
  3 \c/\d & 7+ HCP, minore almeno quinto, nessun nobile quarto.\\
\end{twocol}

Sulla risposta di 2 \d, se l'apertore ha mano bilanciata licita 2 \sa (passabile) nel caso in cui non può garantire la manche a \sa, altrimenti licita 3 \sa.

Sulle risposte di 2 \d/\h/\s, se l'apertore ha mano bilanciata licita 2 \sa (oppure 3 \sa nel caso descritto sopra); il rispondente può allora licitare 3 \c (Stayman avanzata).

In tutti gli altri casi si prosegue in modo naturale.


\pagebreak

\section{Sviluppi sulle sottoaperture}

Sulle sottoaperture le Cue-bid si iniziano solo a salto. Dichiarare 4 \sa sottintende il seme dell'apertore come atout. Ogni altra dichiarazione a colore è naturale (con seme almeno quinto e con 14+ HCP) e forzante 1 giro.

\paragraph{Su 2 \j.} Se il rispondente ha almeno 14 HCP può dichiarare 2 \sa. Seguono le risposte.
\begin{twocol}
 3 \c & Mano minima, carte deboli a \j.\\
 3 \d & Mano minima, carte buone a \j.\\
 3 \h & Mano massima, carte deboli a \j.\\
 3 \s & Mano massima, carte buone a \j.\\
 3 \sa & A, K, Q di \j.
\end{twocol}

\noindent Se dopo 2 \sa c'è un intervento, Pass dà mano minima, X o XX mano massima.




\pagebreak

\section{Verso lo slam}

\subsection{Cue-bid}

Non si gioca se il contratto sarà a SA, e comunque occorre avere già stabilito l'atout (esplicitamente o implicitamente). A partire da 3 \s e salendo gradualmente si dichiara un seme \j (che non sia l'atout) se si possiede una delle seguenti (Cue-bid debole): Asso di \j, K di \j, singolo o vuoto a \j.

Saltare un seme disponibile è come dichiarare che non si ha la Cue-bid in quel seme. Dichiarare SA è come dichiarare l'ultimo seme detto dal compagno, ma sono esclusi 4 SA e 5 SA (richiesta d'Assi e di Re). L'unica possibilità è quindi 3 \s\ - 3 SA, che dichiara Cue-bid (debole) a \s per entrambi i giocatori. Se un giocatore ripete un seme \j su cui ha già dato una Cue-bid, possiede una delle seguenti (Cue-bid forte): Asso di \j, singolo a \j.

Quando un giocatore nega la Cue-bid in un seme, il compagno dovrebbe immediatamente chiudere a manche se a sua volta non possiede la Cue-bid in quel seme.
Come corollario, continuare le Cue-bid dopo che il compagno ne ha negate alcune, equivale a dichiarare (implicitamente) le Cue-bid negate dal compagno.

Esempio (con atout \h): 3 \s\  (Cue-bid debole a \s) - 4 \d\ (nega la Cue-bid a \s e a \c) - 4 \s (Cue-bid forte a \s e debole a \c) - 5 \h\ (nega la Cue-bid forte a \d) - 6 \c\ (Cue-bid forte a \c) - \dots \note{Questo esempio ha un bug!}

\paragraph{Cue-bid avanzate} Se \j è un seme diverso dall'atout, un giocatore A si dice \j-completo se ha dato Cue-bid forte a \j oppure ha già negato una Cue-bid a \j. Se A è \j-completo, \j diventa Cue-bid nel colore in cui ha dato meno informazioni (in caso di parità, il più basso in rango). Come corollario, se A salta \j sta negando una Cue-bid.
\vspace{4mm}

Ovviamente talvolta ha senso interrompere le Cue-bid per effettuare una richiesta d'Assi.


\subsection{Richieste d'Assi e di Re a colore (Roman Key Card Blackwood)}

Ai fini delle richieste d'Assi e di Re, il K di atout è considerato un Asso. In questo caso quindi ci sono 5 Assi e 3 Re. Seguono le risposte alla richiesta d'Assi (4 \sa) nel caso in cui l'atout sia un colore nobile.
\begin{twocol}
5 \c & 1 o 4 Assi.\\
5 \d & 0 o 3 Assi.\\
5 \h & 2 o 5 Assi, senza Q di atout.\\
5 \s & 2 o 5 Assi, con Q di atout.\\
\end{twocol}
\noindent Seguono le risposte nel caso in cui l'atout sia un colore minore.
\begin{twocol}
5 \c & 0 o 3 Assi.\\
5 \d & 1 o 4 Assi.\\
5 \h & 2 o 5 Assi, senza Q di atout.\\
5 \s & 2 o 5 Assi, con Q di atout.\\
\end{twocol}

\noindent Se si riceve una risposta che non dice nulla sulla Q di atout, si può licitare il primo colore disponibile che non sia l'atout per chiedere la Q di atout. Seguono le risposte.

\begin{itemize}
 \item La prima licita ad atout disponibile: niente Q di atout.
 \item La prima licita a \j disponibile (ma necessariamente inferiore a 6 nel colore di atout): Q di atout e K di \j.
 \item 5 \sa: Q di atout, e nessun K oppure nessun K dichiarabile secondo il punto precedente.
\end{itemize}

Se l'atout è \s, questa dichiarazione permette di dare il K di ogni seme, (nel caso in cui il rispondente possieda la Q di \s). Se l'atout è \h, è comunque sempre possibile sapere i K del rispondente mediante questa tecnica: con richiesta 5 \s, la risposta 5 \sa potrebbe significare nessun K, ma anche K di \s.
Allora il richiedente domanda 6 \c, e il rispondente licita 6 \d se non possiede il K di \s, 6 \h se possiede il K di \s.


\smallspace

\noindent In alternativa alla richiesta di Q di atout, con 5 \sa si può effettuare la richiesta di Re. Seguono le risposte.

\begin{twocol}
6 \c & Nessun Re.\\
6 \d & 1 Re.\\
6 \h & 2 Re.\\
6 \s & 3 Re. \\
\end{twocol}
% 
% \paragraph{Senza atout.} Seguono le risposte alla richiesta d'Assi (4 \sa).
% 
% \begin{twocol}
% 5 \c & 0 o 3 Assi.\\
% 5 \d & 1 o 4 Assi.\\
% 5 \h & 2 Assi.\\
% \end{twocol}
% 
% \noindent Il richiedente può successivamente licitare 5 \s per costringere il rispondente a licitare 5 \sa (a giocare).
% 
% \smallspace
% 
% \noindent In alternativa si può effettuare la richiesta di Re, dichiarando 5 \sa. Seguono le risposte.
% 
% \begin{twocol}
% 6 \c & Nessun Re.\\
% 6 \d & 1 Re.\\
% 6 \h & 2 Re.\\
% 6 \s & 3 Re. \\
% 6 \sa & 4 Re.
% \end{twocol}

\subsection{Richieste d'Assi e di Re a \sa (Gerber)}

La richiesta d'Assi si effettua licitando 4 \c, dopo che è stato concordato di giocare a \sa. Seguono le risposte.
\begin{twocol}
  4 \d & 0 o 4 Assi.\\
  4 \h & 1 Asso.\\
  4 \s & 2 Assi.\\
  4 \sa & 3 Assi.
\end{twocol}

\noindent Dopo la richiesta d'Assi, si può effettuare la richiesta di Re licitando 5 \c. Seguono le risposte.
\begin{twocol}
  5 \d & 0 o 4 Re.\\
  5 \h & 1 Re.\\
  5 \s & 2 Re.\\
  5 \sa & 3 Re.
\end{twocol}


\subsection{5 \sa ``Pick A Small Slam''}

In una situazione in cui non vuole già dire qualcos'altro (richiesta di Re, a giocare dopo 5 \c, Cue-bid, ...), la licita di 5 \sa invita a dichiarare in modo naturale proponendo un piccolo slam. Nega il grande slam. Si prosegue la licita in modo naturale.



\pagebreak
\section{Interventi}

Dopo un'apertura a livello di 1 degli avversari, sono previsti i seguenti interventi.

\begin{twocol}
  Dbl & 11+ HCP, (indicativamente) al massimo 2 carte nei colori licitati dagli avversari, almeno 3 negli altri; se gli avversari hanno licitato un nobile, quattro carte nell'altro. \\
  & In alternativa, 16+ HCP e distribuzione qualsiasi.\\
  1 \j & 8+ HCP, \j almeno quinto.\\
  1 \sa & Come apertura, con fermo nel palo degli avversari.\\
  2 \j a livello & 11+ HCP, \j almeno quinto.\\
  2 \j a salto & Come apertura.\\
  3 \j, 4 \j & Come apertura. È escluso l'intervento di 3 \c immediatamente dopo l'apertura di 1 a colore (vedi oltre).
\end{twocol}

\subsection{Ghestem}
Sull'apertura di 1 \j degli avversari, senza che abbiano licitato entrambi, con 12+ HCP e una bicolore (almeno 5-5) si licita come segue.
\begin{twocol}
  2 \j & Con il palo più alto e quello più basso in rango, tra quelli diversi da \j.\\
  2 \sa & Con i due pali di rango più basso.\\
  3 \c & Con i due pali di rango più alto.
\end{twocol}

\subsection{Sviluppi dopo il contre informativo}

\paragraph{A livello di 1, se il RHO passa.} Dopo 1 \j\ - Dbl - Pass, il compagno del contrante licita come segue.

\begin{twocol}
	Pass & Mano forte sbilanciata, buone carte di \j, nessun colore licitabile. \\
	Colore a livello & 0-8 HCP, colore quarto (se possibile, altrimenti il colore più lungo). \\
	Colore a salto & 9+ HCP, colore quarto, forzante. \\
	Colore a doppio salto & 12+ HCP, colore quinto, forzante. Mai a livello di 4. \\
	1 \sa & 8-11 HCP, mano bilanciata, fermo nel colore dell'avversario. \\
	2 \j & 10+ HCP, mano tendenzialmente sbilanciata, senza licite migliori. \\
	2 \sa & 12+ HCP, mano bilanciata, fermo nel colore avversario. \\
\end{twocol}

\paragraph{A livello di 1, se il RHO surcontra.} Dopo 1 \j\ - Dbl - Rbl, il compagno del contrante licita come segue.

\begin{twocol}
	Pass & 0-7 HCP. \\
	Colore a livello & 8-11 HCP, colore quarto. \\
	Colore a salto & 0-7 HCP, colore quinto (barrage), non a sfavore di zona. \\
	2 \j & 12+ HCP, qualsiasi distribuzione, forzante. \\
\end{twocol}

\paragraph{A livello di 2. (Lebensohl)} Dopo 2 \j\ - Dbl - Pass, il compagno del contrante licita come segue.

\begin{twocol}
	Colore a livello di 2 & 0-7 HCP, colore quarto. \\
	2 \sa & Artificiale, forzante, vedi oltre. \\
	Colore a livello di 3, a livello & 8-11 HCP, colore quarto. \\
	Colore a livello di 3, a salto & 12+ HCP, colore quarto, forzante manche. \\
	3 \sa & 12+ HCP, mano bilanciata, senza fermo nel colore avversario.
\end{twocol}

Sulla risposta di 2 \sa, il contrante è obbligato a licitare 3 \c senza Rever, ma può licitare naturalmente un colore a livello di 3 con Rever. Se il contrante licita un nuovo colore, gli sviluppi sono naturali. Dopo 3 \c, invece, il compagno del contrante risponde come segue.

\begin{twocol}
	Pass & 0-7 HCP, almeno quattro carte di \c. \\
	Colore a livello di 3, rango minore di \j & 0-7 HCP, colore quarto. \\
	Colore a livello di 3, rango maggiore di \j & 8-11 HCP, colore quarto. \\
	3 \sa & 12+ HCP, mano bilanciata, fermo nel colore avversario.
\end{twocol}
\end{document}
