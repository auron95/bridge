\documentclass[a4paper,10pt]{article}


\usepackage[utf8x]{inputenc}
\usepackage[italian]{babel}
\usepackage[pdftex]{graphicx}
\usepackage{amsfonts}
\usepackage{amsmath}
\usepackage{amssymb}
\usepackage{amsthm}
\usepackage[mathscr]{urwchancal}
\usepackage{xcolor}
\usepackage{enumerate}
\usepackage{fancyhdr}
\usepackage[colorlinks=true, linkcolor=blue, urlcolor=blue, citecolor=blue]{hyperref}
\usepackage{xspace}
\usepackage[parfill]{parskip}
\usepackage[lmargin=4.5cm, marginparwidth=3.5cm, marginparsep=0.5cm]{geometry}
\usepackage{marginnote}
\usepackage{array}
\usepackage{skak}
\usepackage{needspace}
\usepackage{epigraph}
\usepackage{tabularx}

\DeclareMathAlphabet{\mathpzc}{OT1}{pzc}{m}{it}

\topmargin -1cm
% \oddsidemargin -0.5cm
\textwidth 14.5cm

\reversemarginpar

\setlength{\parindent}{0 pt} % Default 15 pt.
\setlength{\parskip}{0.15 cm} % Default 0 cm?

\renewcommand{\c}{$\clubsuit$\xspace}
\renewcommand{\d}{$\diamondsuit$\xspace}
\newcommand{\h}{$\heartsuit$\xspace}
\newcommand{\s}{$\spadesuit$\xspace}
\renewcommand{\j}{$\bigstar$\xspace}
\newcommand{\rj}{$\blacksquare$\xspace}
\newcommand{\sa}{SA\xspace}
\newcommand{\nt}{SA\xspace}
\newcommand{\M}{\mbox{\raisebox{-1.2pt}{$^\heartsuit\mkern-6mu$} \raisebox{1.2pt}{$\mkern-6mu_\spadesuit$}\xspace}}%{$\mathpzc{M}$\xspace}
\newcommand{\m}{\mbox{\raisebox{-1.2pt}{$^\clubsuit \mkern-4.5mu$} \raisebox{1.2pt}{$\mkern-4.5mu_\diamondsuit$}}\xspace}%{$\mathpzc{m}$\xspace}

\newcommand{\cfbox}[2]{%
    \colorlet{currentcolor}{.}%
    {\color{#1}%
    \fbox{\color{currentcolor}#2}}%
}

\newcommand{\alert}[1]{\cfbox{red}{#1}}

\newcommand{\smallspace}{\vskip0.3cm}

\renewcommand{\tabcolsep}{0.3cm}

\newcommand{\note}[1]{\textcolor{red}{#1}}


\newenvironment{twocol}
  {\smallspace\noindent\tabularx{\linewidth}{ l X }}%p{0.78\textwidth}}}
  {\endtabularx\smallspace}


\newenvironment{threecol}
  {\smallspace\noindent\tabularx{\textwidth}{l l X}}
  {\endtabularx\smallspace}


\newcommand{\biddingtable}[2][0.4cm]{
  \needspace{1cm}
  \marginnote{
    \scriptsize{
    \def\arraystretch{1.5}
    \renewcommand{\tabcolsep}{0.1cm}
    \begin{tabular}{|>{\centering\arraybackslash}p{0.6cm}>{\centering\arraybackslash}p{0.6cm}>{\centering\arraybackslash}p{0.6cm}>{\centering\arraybackslash}p{0.6cm}|}
      \hline
      #2
    \end{tabular}
    }
  }[#1]
}

\newcommand{\biddingtablesec}[1]{\biddingtable[-0.65cm]{#1}}
% \newcommand{\biddingtablepar}[1]{\biddingtable{#1}{0cm}}

% Title Page
\title{Convenzioni di bridge}
\author{Ugo Bindini \and Cristoforo Caffi \and Alice Cortinovis \and Alessandro Iraci \and Giovanni Paolini \and Francesco Veneziano}

\begin{document}
\maketitle


\tableofcontents

\pagebreak

\emph{``Una singola convenzione non capita mai, ma se si giocano 100 convenzioni possono capitare!''} --- Norberto Bocchi

\section{Cose da implementare}
\begin{itemize}
	\item Non-serious 3\sa.
	\item Minorwood rip risposte autistiche (forse)
	\item 1 \nt forcing.
	\item Sistemare la Gazzilli e la Mirtilli.
	\item Risposte su 2 \c.
	\item Secondo giro del rispondente.
\end{itemize}

\subsection{Apertura minore}

Dopo l'apertura di 1 \d:
\begin{itemize}
	\item 1 \nt per mani semi-bilanciate deboli (anche con 4 \d). 
	\item 2 \nt invita (senza fit)
	\item 2 \c: forzante manche, artificiale. Forza 2 \d senza rever.
	\item 2 \d: mano invitante, fit.
	\item 3 \d: barrage.
\end{itemize}

Simile su 1 \c:

Attenzione che dopo 2 \c tutte le risposte tranne 2 \d danno rever.

\section{Aperture}

\paragraph{Algoritmo generale.}
Si licita il colore più lungo, escludendo \s e \h se non sono almeno quinte e le \d se non sono almeno quarte.
In caso di parità di lunghezza dei semi più lunghi, si dichiara il più alto se sono almeno quinti e si dichiara il più basso se sono quarti.
Nota: con 4\s\ - 4\h\ - 3\d\ - 2\c, si apre di 1 \c.

\subsection{Mani fortissime}

Ogni mano con forza di manche o 22+ HCP bilanciata si apre convenzionalmente di 2 \c (forzante 1 giro).

\subsection{Mani bilanciate}
(4-3-3-3, 4-4-3-2, 5-3-3-2 con quinta minore)
\smallspace

\begin{threecol}
 1 \j & 12--14 HCP & Si licita seguendo l'algoritmo generale. \\
 1 \sa & 15--17 HCP & \\
 1 \c & 18--19 HCP & Poi 2 \sa su una risposta a livello di 1 (ma dare un fit nobile con \alert{3 \d} ha precedenza). \\
 2 \sa & 20--21 HCP & Anche 5-3-3-2 con quinta nobile.
\end{threecol}


\subsection{Mani monocolore}
(5-3-3-2 con quinta nobile, 6-3-2-2, 6-3-3-1, 7-3-2-1, \dots)

\begin{twocol}
 1 \j & 12+ HCP, \j è il colore lungo.
\end{twocol}


\subsection{Mani bicolore}
(5-4-3-1, 5-4-2-2, 6-4-2-1, 5-5-2-1, \dots)

\begin{twocol}
 1 \j & 12+ HCP, \j è il colore più lungo; in caso di parità (5-5 o 6-6) si licita il colore più alto (in accordo con l'algoritmo generale).
\end{twocol}


\subsection{Mani tricolore}
(4-4-4-1, 5-4-4-0)

\begin{twocol}
 1 \j & 12--15 HCP oppure 5-4-4-0, si licita seguendo l'algoritmo generale. Unica eccezione: una mano 4-4-4-1 con singolo a picche si apre di 1 \d.
\end{twocol}


\subsection{Sottoaperture}

Per la sottoapertura di 2 \d si propongono due soluzioni alternative: 2 \d bicolore nobile almeno 5-4 (Pisana) o naturale con sesta di quadri debole.

\begin{twocol}
 \alert{2 \d} & 6--10 HCP, bicolore nobile almeno 4-4 (\hyperref[miniera]{Miniera}).\\
 2 \j & 6--10 HCP, \j almeno sesto (anche \d se non si usa la \hyperref[miniera]{Miniera}).\\
 3 \j & Colore almeno settimo (barrage), meglio se a compagno passato.\\
 \alert{3 \sa} & Minore almeno settimo chiuso (AKQ), al più una Q laterale.\\
 4 \j & Barrage (4 perdenti).
\end{twocol}

\pagebreak

\section{Risposte sull'apertura di 1 a colore}

Si risponde con 5+ HCP oppure con un Asso.

\subsection{Con fit su un colore maggiore \M}

% Per valutare la forza della mano si calcolano i punti totali (HD), dati dalla somma degli HCP e i punti distribuzione, così intesi.
% \begin{itemize}
%  \item Per ogni atout oltre l'ottava: 1 punto;
%  \item per ogni doubleton: 1 punto;
%  \item per ogni singolo: 2 punti se si hanno tre atout, 3 punti se si hanno almeno quattro atout;
%  \item per ogni vuoto: 3 punti se si hanno tre atout, 5 punti se si hanno almeno quattro atout
% \end{itemize}

\biddingtable{1 \M & P & * & }
\begin{twocol}
  \alert{2 \c} & Fit almeno terzo, 10--11 HCP, rispondente passato di mano (\hyperref[drury]{Drury}).\\
  2 \M  & Fit terzo, mano non invitante.\\
  2 \sa & Fit almeno terzo (\hyperref[jensen]{Jensen}), mano almeno invitante.\\
  3 \c & 8--9 HCP, fit almeno quarto.\\
  3 \d & 5-7 HCP, fit almeno quarto.\\
  3 \M  & 0--4 HCP, fit almeno quarto (barrage). \\
  3 \sa & Splinter nell'altro nobile.\\
  4 \c/\d & Splinter a \c/\d.\\
  4 \M  & Fit almeno quinto (barrage).\\
  %Cue-bid & Fit almeno terzo, forzante manche. Le Cue-bid partono da 3 \h (con atout \s), da 3 \s (con atout \h).
\end{twocol}

\paragraph{Sviluppi dopo 1 \M\ - 2 \c (Drury).} \label{drury} \hspace{1cm}

\biddingtable{& & P & P \\ 1 \M & P & 2 \c & P \\ * &&&}
\begin{twocol}
  2 \d & 12+ HCP, forzante fino a 2 \M.\\
  2 \M & Apertura leggera, sign-off.\\
  3 \sa & 5-3-3-2 ``passa o correggi'', senza visuale di slam.\\
  4 \M & A giocare.\\
  Altro & Cue-bid, forzante manche.
\end{twocol}

\paragraph{Sviluppi dopo 1 \M\ - 2 \M.}
L'apertore passa senza rever. Con rever può licitare un nuovo colore a salto (Cue-bid) con interesse di slam, oppure come segue.

\biddingtable{1 \M & P & 2 \M & P \\ * &&&}
\begin{twocol}
  2 \sa & Interrogativo.\\
  Nuovo seme a livello & \emph{Trial-bid}, seme debole almeno terzo in cerca di supporto (singolo/vuoto od onori).\\
  3 \M & Invitante in cerca di supporto in atout.\\
  3 \sa & Passa o correggi a 4 \M.\\
  4 \M & A giocare.\\
\end{twocol}

Dopo una \emph{trial-bid}, il rispondente rivaluta la mano e sceglie fra 3 \M e 4 \M.

Dopo 1 \M\ - 2 \M\ - 2 \sa, il rispondente licita come segue.

\biddingtable{1 \M & P & 2 \M & P \\ 2 \sa & P & * &}
\begin{twocol}
  3 \M & Mano minima.\\
  Altro & Mano massima non bilanciata, il colore più economico nel quale possiede dei valori (senza superare 3 \M).\\
  3 \sa & Mano massima bilanciata (preferisce 3 \sa a 4 \M).\\
  4 \M & Mano massima (preferisce 4 \M a 3 \sa).\\
\end{twocol}

\paragraph{Sviluppo dopo 1 \M\ - 2 \sa (Jensen).} \label{jensen} Dopo 1 \s\ - 2 \sa, l'apertore licita come segue.

\biddingtable{1 \s & P & 2 \sa & P \\ * &&&}
\begin{twocol}
  3 \c & 12--15 HCP.\\
  3 \d & 16+ HCP, singolo o vuoto a \d, senza quinte laterali.\\
  3 \h & 16+ HCP, singolo o vuoto a \h, senza quinte laterali.\\
  3 \s & 16+ HCP singolo o vuoto a \c, senza quinte laterali.\\
  3 \sa & 18+ HCP, senza quinte laterali né singoli né vuoti.\\
  4 \c/\d/\h & 16+ HCP, seme almeno quinto.\\
  4 \s & 16--17 HCP, senza singoli né vuoti.
\end{twocol}

Se l'apertore aveva aperto di 1 \h, le risposte sono analoghe (è esclusa la risposta di 4 \s).

Dopo 1 \M\ - 2 \sa\ - 3 \c, il rispondente licita come segue.

\biddingtable{1 \M & P & 2 \sa & P \\ 3 \c & P & * &}
\begin{twocol}
  3 \d & Forzante manche, richiesta di singoli o vuoti.\\
  3 \M & Invitante a 4 \M.\\
  Altro & Cue-bid.
\end{twocol}

Dopo 1 \M\ - 2 \sa\ - 3 \c\ - 3 \d, l'apertore licita come segue.

\biddingtable{1 \M & P & 2 \sa & P \\ 3 \c & P & 3 \d & P  \\ * &&&}
\begin{twocol}
  3 \M & Senza singoli né vuoti, mano minima.\\
  3 \sa & Senza singoli né vuoti, mano massima.\\
  Altro & Singolo o vuoto (il più economico).
\end{twocol}

\subsection{Dopo un'apertura minore}

Se possibile, dire un colore almeno quarto a livello di 1 --- il più lungo, e in caso di parità il più economico. Attenzione al caso di una mano bicolore \s/\h (vedi oltre). Altrimenti, si licita come segue.

Dopo l'apertura di 1 \c:

\biddingtable{1 \c & P & * &}
\begin{twocol}
  %1 \d/\h/\s & \d/\h/\s almeno quarte.\\
  1 \sa & 5--9 HCP, senza quarte licitabili a livello di 1. \\
  2 \c & Forzante manche, senza quarte licitabili a livello di 1, quindi 4+ \c. \\
  \alert{2 \d} & 0--4 HCP e un seme nobile almeno sesto. L'apertore può licitare 2 \h (``passa o correggi''). \\
  2 \h & 5--8 HCP, bicolore \s/\h almeno 5-4. \\
  2 \s & 9--11 HCP, bicolore \s/\h almeno 5-4. \\
  2 \sa & 10--11 HCP, mano bilanciata.\\
  3 \c & 10--11 HCP, buon fit a \c.
\end{twocol}

Dopo l'apertura di 1 \d:
\biddingtable{1 \d & P & * &}
\begin{twocol}
  %1 \d/\h/\s & \d/\h/\s almeno quarte.\\
  1 \sa & 5--9 HCP, senza quarte licitabili a livello di 1. \\
  2 \c & Forzante manche, senza quarte licitabili a livello di 1. \\
  2 \d & 5-9 HCP, fit a \d. \\
  2 \h & 5--8 HCP, bicolore \s/\h almeno 5-4 (\hyperref[revflanneryhearts]{Reverse Flannery}). \\
  2 \s & 9--11 HCP, bicolore \s/\h almeno 5-4 (\hyperref[revflanneryspades]{Reverse Flannery}). \\
  2 \sa & 10--11 HCP, mano bilanciata.\\
  3 \c & 10+ HCP, buon fit a \d. L'apertore dice 3 \d con mano minima, dà il fermo con 3 \M (negandolo nell'altro nobile), o dice 3 \sa se ferma i nobili. \\
  3 \d & Barrage.
\end{twocol}



\subsection{Dopo l'apertura 1 \h (senza fit)}

Se possibile, dare la quarta di \s. Le risposte 2/1 da parte di un rispondente precedentemente passato di mano sono solo invitanti (10--11 HCP); in questo caso la risposta di 2 \c è artificiale (\hyperref[drury]{Drury}) e dà il fit.

\biddingtable{1 \h & P & * &}
\begin{twocol}
  1 \s & \s almeno quarte.\\
  1 \sa & 5--11 HCP. \\
  2 \c & Forzante manche. \\
  2 \d & Forzante manche, \d almeno quinte.\\
  2 \s & 0--5 HCP, \s almeno seste.\\
  3 \s & 5--7 HCP, \s belle almeno settime.\\
  4 \s & Barrage.

\end{twocol}


\subsection{Dopo l'apertura di 1 \s (senza fit)}

Le risposte 2/1 da parte di un rispondente precedentemente passato di mano sono solo invitanti; in questo caso la risposta di 2\c è artificiale (\hyperref[drury]{Drury}) e dà il fit.

\biddingtable{1 \s & P & * &}
\begin{twocol}
  1 \sa & 5--11 HCP. \\
  2 \c & Forzante manche. \\
  2 \d & Forzante manche, \d almeno quinte.\\
  2 \h & Forzante manche, \h almeno quinte.\\
  3 \h & 9--11 HCP, \h belle almeno seste, invitante a 3 \sa o 4 \h.\\
  4 \h & Barrage.
\end{twocol}

\pagebreak

\section{Sviluppi dopo una risposta 1/1}

La seconda dichiarazione dell'apertore descrive la distribuzione della mano secondo principi naturali.

\subsection{Dopo una risposta a colore}

\paragraph{Dopo 1 \c\ - 1 \d.} In ordine di priorità:

\biddingtable{1 \c & P & 1 \d & P \\ * &&&}
\begin{twocol}
 2 \sa & 18--19 HCP, mano bilanciata.\\
 1 \M & 12--15 HCP, \M almeno quarto.\\
 2 \M & 16+ HCP, \M almeno quarto.\\
 3 \M & Cue-bid con fit a \d, mano molto forte e sbilanciata.\\
 2 \d & 12--15 HCP, fit a \d.\\
 3 \d & 16+ HCP, fit a \d.\\
 1 \sa & 12--14 HCP, mano bilanciata.\\
 3 \sa & 19+ HCP, \c almeno seste, fermi nei semi nobili.\\
 2 \c & 12--15 HCP, \c almeno seste.\\
 3 \c & 16+ HCP, \c almeno seste.
\end{twocol}

\paragraph{Dopo 1 \c\ - 1 \h.} In ordine di priorità:

\biddingtable{1 \c & P & 1 \h & P \\ * &&&}
\begin{twocol}
 2 \sa & 18--19 HCP, mano bilanciata, senza fit a \h.\\
 \alert{3 \d} & 18--19 HCP, mano bilanciata, con fit a \h.\\
 3 \s & Cue-bid con fit a \h, mano molto forte e sbilanciata.\\
 4 \m & Cue-bid con fit a \h, mano molto forte e sbilanciata.\\
 2 \h & 12--15 HCP, fit a \h.\\
 3 \h & 16+ HCP, fit a \h (implica mano non bilanciata).\\
 1 \s & 12--15 HCP, \s almeno quarte.\\
 2 \s & 16+ HCP, \s almeno quarte.\\
 2 \d & 16+ HCP, \d almeno quarte.\\
 3 \sa & 19+ HCP, \c almeno seste, fermi a \d e \s.\\
 2 \c & 12--15 HCP, \c almeno quinte.\\
 3 \c & 16+ HCP, \c almeno seste.\\
 1 \sa & 12--15 HCP.\\
\end{twocol}


\paragraph{Dopo 1 \c\ - 1 \s.} In ordine di priorità:

\biddingtable{1 \c & P & 1 \s & P \\ * &&&}
\begin{twocol}
 2 \sa & 18--19 HCP, mano bilanciata, senza fit a \s.\\
 \alert{3 \d} & 18--19 HCP, mano bilanciata, con fit a \s.\\
 3 \h & Cue-bid con fit a \s, mano molto forte e sbilanciata.\\
 4 \m & Cue-bid con fit a \s, mano molto forte e sbilanciata.\\
 2 \s & 12--15 HCP, fit a \s.\\
 3 \s & 16+ HCP, fit a \s (implica mano non bilanciata).\\
 2 \d & 16+ HCP, \d almeno quarte.\\
 2 \h & 16+ HCP, \h almeno quarte.\\
 3 \sa & 19+ HCP, \c almeno seste, fermi a \d e \h.\\
 2 \c & 12--15 HCP, \c almeno quinte.\\
 3 \c & 16+ HCP, \c almeno seste.\\
 1 \sa & 12--15 HCP.\\
\end{twocol}

\paragraph{Dopo 1 \d\ - 1 \h.} In ordine di priorità:
\biddingtable{1 \d & P & 1 \h & P \\ * &&&}

\begin{twocol}
 2 \h & 12--15 HCP, fit a \h.\\
 3 \h & 16--18 HCP, fit a \h.\\
 3 \s & Cue-bid con fit a \h.\\
 4 \m & Cue-bid con fit a \h.\\
 1 \s & 12--15 HCP, \s almeno quarte.\\
 2 \s & 16+ HCP, \s quarte, senza fit (implica \d almeno quinte).\\
 2 \c & 12--15 HCP, \c almeno quarte.\\
 2 \d & 12--15 HCP, \d almeno seste.\\
 1 \sa & 12--14 HCP, mano bilanciata.\\
 2 \sa & 16--18 HCP, senza fit, senza singoli né vuoti, \d al massimo seste (5-4-2-2, 6-3-2-2).\\
 3 \c & 16+ HCP, \c almeno quarte (bicolore \d/\c), senza fit, con almeno un singolo o vuoto.\\
 3 \d & 16+ HCP, \d seste con almeno un singolo o vuoto, oppure \d settime, senza fit.\\
 3 \sa & 19+ HCP, senza fit, senza singoli né vuoti, \d al massimo seste (5-4-2-2, 6-3-2-2).
\end{twocol}

\paragraph{Dopo 1 \d\ - 1 \s.} In ordine di priorità:

\biddingtable{1 \d & P & 1 \s & P \\ * &&&}
\begin{twocol}
 2 \s & 12-15 HCP, fit a \s.\\
 3 \s & 16-18 HCP, fit a \s.\\
 3 \h & Cue-bid con fit a \s.\\
 4 \m & Cue-bid con fit a \s.\\
 2 \h & 16+ HCP, \h quarte, senza fit (implica \d almeno quinte).\\
 2 \c & 12-15 HCP, \c almeno quarte.\\
 2 \d & 12-15 HCP, \d almeno quinte.\\
 1 \sa & 12-15 HCP.\\
 2 \sa & 16-18 HCP, senza fit, senza singoli né vuoti, \d al massimo seste (5-4-2-2, 6-3-2-2).\\
 3 \c & 16+ HCP, \c almeno quarte (bicolore \d/\c), senza fit, con almeno un singolo o vuoto.\\
 3 \d & 16+ HCP, \d seste con almeno un singolo o vuoto, oppure \d settime, senza fit.\\
 3 \sa & 19+ HCP, senza fit, senza singoli né vuoti, \d al massimo seste (5-4-2-2, 6-3-2-2).
\end{twocol}


\paragraph{Dopo 1 \h\ - 1 \s.} In ordine di priorità:

\biddingtable{1 \h & P & 1 \s & P \\ * &&&}
\begin{twocol}
 \alert{2 \sa} & 16+ HCP, \h almeno seste, una quarta laterale minore (\hyperref[mirtilli]{Mirtilli}).\\
 \alert{2 \c} & 12--15 HCP e \c almeno quarte, oppure 16+ HCP (\hyperref[gazzilli]{Gazzilli}).\\
 3 \s & Mano invitante e sbilanciata, buone \s quarte.\\
 2 \s & 12--15 HCP, fit quarto.\\
 3 \m & Mano invitante, \m buon seme almeno quinto.\\
 3 \h & Mano invitante, \h belle almeno seste.\\
 2 \d & 12--15 HCP, \d almeno quarte.\\
 2 \h & 12--15 HCP, \h almeno seste.\\
 1 \sa & 12--15 HCP (mano semi-bilanciata).
\end{twocol}


\subsection{Dopo la risposta di 1 \sa}

Poiché la mano del rispondente è limitata superiormente, l'apertore può sempre decidere di passare.

\paragraph{Dopo 1 \c\ - 1 \sa.} In ordine di priorità:

\biddingtable{1 \c & P & 1 \sa & P \\ * &&&}
\begin{twocol}
 2 \d/\h/\s & 16+ HCP, seme almeno quarto.\\
 2 \sa & 18--19 HCP, mano bilanciata.\\
 3 \c & 16+ HCP, \c almeno seste.\\
 2 \c & A giocare.
\end{twocol}

\paragraph{Dopo 1 \d\ - 1 \sa.} In ordine di priorità:

\biddingtable{1 \d & P & 1 \sa & P \\ * &&&}
\begin{twocol}
 2 \c & 14--15 HCP, \c almeno quarte.\\
 2 \d & 12--15 HCP, \d almeno seste.\\
 2 \M & 16+ HCP, \M almeno quarto (nel caso delle \s, implica \d almeno quinte).\\
 2 \sa & Invitante a 3 \sa.\\
 3 \c & 16+ HCP, \c almeno quarte (mano sbilanciata).\\
 3 \d & 16+ HCP, \d almeno seste (mano sbilanciata).
\end{twocol}

\paragraph{Dopo 1 \h\ - 1 \sa.} In ordine di priorità:

\biddingtable{1 \h & P & 1 \sa & P \\ * &&&}
\begin{twocol}
 \alert{2 \sa} & 16--17 HCP, \h almeno seste, una quarta laterale minore (\hyperref[mirtilli]{Mirtilli}).\\
 \alert{2 \c} & 12--15 HCP e \c almeno quarte, oppure 16+ HCP (\hyperref[gazzilli]{Gazzilli}).\\
 2 \d & 12--15 HCP, \d almeno terze.\\
 2 \h & 12--15 HCP, \h almeno seste.\\
 2 \s & Invitante, buone \s almeno quinte.\\
 3 \m & Invitante, buon seme almeno quinto.\\
 3 \h & Invitante, \h belle almeno seste.
\end{twocol}

\paragraph{Dopo 1 \s\ - 1 \sa.} In ordine di priorità:

\biddingtable{1 \s & P & 1 \sa & P \\ * &&&}
\begin{twocol}
 \alert{2 \sa} & 16--17 HCP, \s almeno seste, una quarta laterale qualsiasi (\hyperref[mirtilli]{Mirtilli}).\\
 \alert{2 \c} & 12--15 HCP e \c almeno quarte, oppure 16+ HCP (\hyperref[gazzilli]{Gazzilli}).\\
 2 \d & 12--15 HCP, \d almeno terze.\\
 2 \h & 12--15 HCP, \h almeno quarte.\\
 2 \s & 12--15 HCP, \s almeno seste.\\
 3 \c/\d/\h & Invitante, buon seme almeno quinto.\\
 3 \s & Invitante, \s belle almeno seste.
\end{twocol}


\subsection{2 \c Gazzilli} \label{gazzilli}

La 2 \c Gazzilli si applica in tre situazioni: 1 \h\ - 1 \s\ - 2 \c, 1 \h\ - 1 \sa\ - 2 \c, 1 \s\ - 1 \sa\ - 2 \c.
Il rispondente licita come segue.

\biddingtable{1 \M & P & 1 \j & P \\ 2 \c & P & * &}
\begin{twocol}
  2 \d & Relais positivo, 8+ HCP. \\
  Riporto nel nobile & 5--7 HCP, doubleton. \\
  Altro nobile & 5--7 HCP, nobile almeno quinto. \\
  3 \m & 5--7 HCP, minore almeno sesto. \\
  2 \sa & Tutti gli altri casi.
\end{twocol}

In aggiunta, sulla sequenza 1 \h\ - 1 \sa\ - 2 \c, il rispondente ha a disposizione l'ulteriore risposta illogica di 2 \s per indicare una bicolore minore debole.

\paragraph{Sviluppi dopo il relais positivo di 2 \d.} Sull'apertura di 1 \h, l'apertore licita come segue.

\biddingtable{1 \h & P & 1 \j & P \\ 2 \c & P & 2 \d & P \\ * &&&}
\begin{twocol}
  2 \h & 12--15 HCP, \c almeno quarte (anche terze in casi eccezionali). \\
  2 \s & Esattamente 3 carte di \s, rever. (*)\\
  2 \sa & 19+ HCP, 5-3-3-2 senza terza di \s. \\
  3 \m & Minore almeno quarto, senza terza di \s, rever. \\
  3 \h & \h almeno seste, senza terza di \s, rever. \\
  3 \s & \s quarte, rever. \\
  3 \sa & 16-18 HCP, 5-3-3-2 senza terza di \s.
\end{twocol}

(*) Sviluppi dopo 2 \s. Il rispondente licita 2 \sa (relais) e l'apertore licita come segue.
\biddingtable{1 \h & P & 1 \j & P \\ 2 \c & P & 2 \d & P \\ 2 \s & P & 2 \sa & P \\ *}
\begin{twocol}
  3 \m & Minore almeno quarto. \\
  3 \h & \h almeno seste. \\
  3 \s & 19+ HCP, 5-3-3-2. \\
  3 \sa & 16--18 HCP, 5-3-3-2.
\end{twocol}


Sull'apertura di 1 \s, l'apertore licita come segue.

\biddingtable{1 \s & P & 1 \sa & P \\ 2 \c & P & 2 \d & P \\ * &&&}
\begin{twocol}
  2 \h & \h terze o quarte, rever. (*)\\
  2 \s & 12--15 HCP, \c almeno quarte (anche terze in casi eccezionali). \\
  2 \sa & 19+ HCP, 5-3-3-2 senza terza di \h. \\
  3 \m & Minore almeno quarto, senza terza di \h, rever. \\
  3 \h & \h quinte, rever. \\
  3 \s & \s almeno seste, senza terza di \h, rever. \\
  3 \sa & 16--18 HCP, 5-3-3-2 senza terza di \h.
\end{twocol}

(*) Sviluppi dopo 2 \h. Il rispondente licita 2 \s (relais) e l'apertore licita come segue.
\biddingtable{1 \s & P & 1 \sa & P \\ 2 \c & P & 2 \d & P \\ 2 \h & P & 2 \s & P \\ *}
\begin{twocol}
  2 \sa & 19+ HCP, 5-3-3-2. \\
  3 \m & Minore almeno quarto. \\
  3 \h & \h quarte. \\
  3 \s & \s almeno seste. \\
  3 \sa & 16--18 HCP, 5-3-3-2.
\end{twocol}

\subsection{2 \sa Mirtilli} \label{mirtilli}

La 2 \sa Mirtilli si applica in tre situazioni: 1 \h\ - 1 \s\ - 2 \sa, 1 \h\ - 1 \sa\ - 2 \sa, 1 \s\ - 1 \sa\ - 2 \sa.
Dopo l'apertura di 1 \M, il rispondente licita come segue.

\biddingtable{1 \M & P & 1 \j & P \\ 2 \sa & P & * &}
\begin{twocol}
 Pass & \\
 3 \c & Interrogativo, chiede la quarta laterale.\\
 3 \M & 5--7 HCP, riporto nel nobile dell'apertore.\\
 4 \m & Cue bid con fit a \M.\\
 Altro & Naturale, forzante.
\end{twocol}

\biddingtable{1 \M & P & 1 \j & P \\ 2 \sa & P & 3 \c & P \\ * &&&}
Sull'interrogativa 3 \c, l’apertore licita la quarta in maniera naturale al livello di tre e ripete il nobile di apertura per indicare convenzionalmente le fiori.

\pagebreak

\section{Sviluppi dopo una risposta 2/1}

\paragraph{Dopo 1 \m\ - 2 \d.} Il rispondente possiede una mano monocolore nobile debolissima (0-4 HCP). L'apertore licita come segue.

\biddingtable{1 \m & P & 2 \d & P \\ * &&&}
\begin{twocol}
  2 \h & ``Passa o correggi''.\\
  2 \sa & Relais interrogativo, chiede di licitare il nobile.\\
  3 \m & Naturale, non forzante.
\end{twocol}

\paragraph{Dopo 1 \m\ - 2 \h.} Il rispondente possiede una mano bicolore nobile debole con le \s almeno quinte (5--8 HCP). L'apertore licita come segue.

\biddingtable{1 \m & P & 2 \h & P \\ * &&&}
\begin{twocol}
  Pass & \\
  2 \s & A giocare.\\
  2 \sa & Relais interrogativo (\hyperref[zurli]{Zurlì}).\\
  3 \m & Naturale, non forzante.\\
  3 \M & Invitante.
\end{twocol}

\paragraph{Dopo 1 \m\ - 2 \s.} Il rispondente possiede una mano bicolore nobile invitante con le \s almeno quinte (9--11 HCP). L'apertore licita come segue.

\biddingtable{1 \m & P & 2 \s & P \\ * &&&}
\begin{twocol}
  Pass & \\
  2 \sa & Relais interrogativo (\hyperref[zurli]{Zurlì}).\\
  3 \m & Naturale, non forzante.\\
  3 \h & A giocare.\\
  3 \s & Invitante.
\end{twocol}

\paragraph{2 \sa interrogativo (Zurlì).} \label{zurli} Dopo una risposta bicolore nobile del rispondente, l'apertore può licitare 2 \sa (interrogativo). Seguono le risposte.

\biddingtable{1 \m & P & 2 \M & P \\ 2 \sa & P & * &}
\begin{twocol}
  3 \c & Mano minima, \h quarte.\\
  3 \d & Mano minima, \h almeno quinte.\\
  3 \h & Mano massima, \h almeno quinte.\\
  3 \s & Mano massima, \h quarte.
\end{twocol}

\paragraph{Dopo 1 \c\ - 2 \c.} L'apertore licita come segue.

\biddingtable{1 \c & P & 2 \c & P \\ * &&&}
\begin{twocol}
	2 \d/\h/\s & Buon seme almeno quarto (per giocare a \sa).\\
	2 SA & 18--19 HCP, mano bilanciata.\\
	3 \c & 14+ HCP, \c almeno seste, senza singoli né vuoti.\\
	3 \d & 14+ HCP, \c almeno quinte, singolo o vuoto a \d.\\
	3 \h & 14+ HCP, \c almeno quinte, singolo o vuoto a \h.\\
	3 \s & 14+ HCP, \c almeno quinte, singolo o vuoto a \s.\\
	3 \sa & 12--14 HCP mano bilanciata, a giocare.\\
	4 \c & Gerber.
\end{twocol}

\paragraph{Dopo 1 \d\ - 2 \c.} L'apertore licita come segue.

\biddingtable{1 \d & P & 2 \c & P \\ * &&&}
\begin{twocol}
	2 \d & 12--15 HCP, \d almeno seste.\\
	2 \M & Buon seme almeno quarto (per giocare a \sa).\\
	2 SA & 12--14 HCP mano (semi-)bilanciata.\\
	3 \c & Buone \c almeno quarte (per giocare a \sa).\\
	3 \d & 16+ HCP, ottime \d almeno seste.\\
\end{twocol}

\paragraph{Dopo 1 \h\ - 2 \c.} Nota: il rispondente possiede almeno 4 carte di \c. L'apertore licita come segue.

\biddingtable{1 \h & P & 2 \c & P \\ * &&&}
\begin{twocol}
 2 \d & \d almeno quarte.\\
 2 \h & \h almeno seste.\\
 2 \s & \s almeno quarte.\\
 2 \sa & 12--16 HCP, mano semi-bilanciata.\\
 3 \c & Fit almeno quarto, buona mano per il gioco ad atout.\\
 3 \d & Cue-bid, interesse di slam a \c.\\
 3 \h & 16+ HCP, ottime \h almeno seste.\\
 3 \s & Cue-bid, interesse di slam a \c.\\
 3 \sa & 17--19 HCP, mano semi-bilanciata.
\end{twocol}

\paragraph{Dopo 1 \h\ - 2 \d.} L'apertore licita come segue.

\biddingtable{1 \h & P & 2 \d & P \\ * &&&}
\begin{twocol}
 2 \h & \h almeno seste.\\
 2 \s & \s almeno quarte.\\
 2 \sa & 12--16 HCP, mano semi-bilanciata.\\
 3 \c & \c almeno quarte.\\
 3 \d & Fit almeno terzo, buona mano per il gioco ad atout.\\
 3 \h & 16+ HCP, ottime \h almeno seste.\\
 3 \s & Cue-bid, interesse di slam a \d.\\
 3 \sa & 17--19 HCP, mano semi-bilanciata.\\
 4 \c & Cue-bid, interesse di slam a \d.
\end{twocol}

\paragraph{Dopo 1 \s\ - 2 \c.} L'apertore licita come segue.

\biddingtable{1 \s & P & 2 \d & P \\ * &&&}
\begin{twocol}
 2 \d & \d almeno quarte.\\
 2 \h & \h almeno quarte.\\
 2 \s & \s almeno seste.\\
 2 \sa & 12--16 HCP, mano semi-bilanciata.\\
 3 \c & \c almeno quarte.\\
 3 \d/\h & 14+ HCP, buon seme almeno quinto.\\
 3 \s & 16+ HCP, ottime \s almeno seste.\\
 3 \sa & 17--19 HCP, mano semi-bilanciata.\\
\end{twocol}

\paragraph{Dopo 1 \s\ - 2 \d.} L'apertore licita come segue.

\biddingtable{1 \s & P & 2 \d & P \\ * &&&}
\begin{twocol}
 2 \h & \h almeno quarte.\\
 2 \s & \s almeno seste.\\
 2 \sa & 12--16 HCP, mano semi-bilanciata.\\
 3 \c & \c almeno quarte.\\
 3 \d & Fit almeno terzo, buona mano per il gioco ad atout.\\
 3 \h & Cue-bid, interesse di slam a \d.\\
 3 \s & 16+ HCP, ottime \s almeno seste.\\
 3 \sa & 17--19 HCP, mano semi-bilanciata.\\
 4 \c & Cue-bid, interesse di slam a \d.
\end{twocol}

\paragraph{Dopo 1 \s\ - 2 \h.} L'apertore licita come segue.

\biddingtable{1 \s & P & 2 \h & P \\ * &&&}
\begin{twocol}
 2 \s & \s almeno seste, senza fit a \h.\\
 2 \sa & 12--16 HCP, mano semi-bilanciata, senza fit a \h.\\
 3 \m & \m almeno quarto, senza fit a \h.\\
 3 \h & 14+ HCP, fit a \h.\\
 3 \s & 16+ HCP, ottime \s almeno seste.\\
 3 \sa & 17--19 HCP, mano semi-bilanciata, senza fit a \h.\\
 4 \m & Splinter.\\
 4 \h & Fit a \h, mano minima.
\end{twocol}

\pagebreak

\section{Dopo la ridichiarazione dell'apertore}

\subsection{Quarto colore del rispondente}

Nel caso in cui le prime tre licite siano state tutte a colore, e abbiano coinvolto 3 diversi colori, se l'apertore non ha dato rever il rispondente può licitare a livello* il quarto colore (forzante, interrogativo) anche senza possederlo. L'apertore licita come segue, in ordine di priorità.

\begin{itemize}
 \item \sa a livello: fermo nel quarto colore, disponibile a giocare a \sa.
 \item Primo colore del rispondente: misfit (colore terzo), senza fermo nel quarto colore.
 \item Altro: naturale, descrittivo (potrebbe anche non allungare i propri semi), senza fermo nel quarto colore e al più due carte nel primo colore del rispondente.
\end{itemize}

Questo si applica nei seguenti casi:

1 \c\ - 1 \d\ - 1 \h\ - 2 \s. *Eccezione: 1 \s è forzante con quarta di \s, e si interroga con 2 \s.\\
1 \c\ - 1 \d\ - 1 \s\ - 2 \h.\\
1 \c\ - 1 \h\ - 1 \s\ - 2 \d.\\
1 \d\ - 1 \h\ - 1 \s\ - 2 \c.\\
1 \d\ - 1 \h\ - 2 \c\ - 2 \s. Cercare un fit a \s è illogico: il rispondente non può avere la quinta.\\
1 \d\ - 1 \s\ - 2 \c\ - 2 \h. Cercare un fit a \h è illogico: il rispondente non può avere la quarta.\\
1 \h\ - 1 \s\ - 2 \d\ - 3 \c.

\subsection{Terzo colore del rispondente}

Nel caso in cui, dopo una risposta a livello di 1 a colore, l'apertore abbia ripetuto il proprio colore di apertura, il rispondente può licitare a livello un terzo colore (forzante, interrogativo) che può essere tanto di quattro o più carte, quanto una buona terza onorata. L'apertore licita come segue, in ordine di priorità.

\begin{itemize}
 \item \sa a livello: fermo nel quarto colore non licitato, disponibile a giocare a \sa.
 \item Un colore del rispondente: misfit (colore terzo), senza fermo nel quarto colore.
 \item Il proprio colore: allunga ulteriormente il seme, al più due carte nei semi del rispondente.
 \item Quarto colore: niente di meglio da licitare, senza fermo.
\end{itemize}

Questo si applica nei seguenti casi:

1 \c\ - 1 \d\ - 2 \c\ - 2 \M.\\
1 \c\ - 1 \h\ - 2 \c\ - 2 \d/\s.\\
1 \c\ - 1 \s\ - 2 \c\ - 2 \d/\h.\\
1 \d\ - 1 \h\ - 2 \d\ - 2 \c/\s.\\
1 \d\ - 1 \s\ - 2 \d\ - 2 \c/\h.\\
1 \h\ - 1 \s\ - 2 \h\ - 3 \m


\subsection{2 \c Roudi}

Dopo la sequenza 1 colore - 1 colore - 1 \sa, la licita di 2 \c (Roudi) chiede ulteriore descrizione della mano. Se il rispondente ha licitato 1 \d, l'apertore licita come segue.

\biddingtable{1 \c & P & 1 \d & P \\ 1 \sa & P & 2 \c & P \\ * &&&}
\begin{twocol}
	2 \d & 12--13 HCP, terza di \d.\\
	2 \h & 12--13 HCP, nega la terza di \d.\\
	2 \s & 14--15 HCP, nega la terza di \d.\\
	2 \sa & 14--15 HCP, terza di \d.
\end{twocol}

Se il rispondente ha licitato un seme nobile, l'apertore licita come segue.

\biddingtable{1 \j & P & 1 \M & P \\ 1 \sa & P & 2 \c & P \\ * &&&}
\begin{twocol}
	2 \d & 12--13 HCP, nega la terza di \M.\\
	2 \h & 12--13 HCP, terza di \M.\\
	2 \s & 14--15 HCP, terza di \M.\\
	2 \sa & 14--15 HCP, nega la terza di \M.
\end{twocol}

\subsection{2 \sa Ingberman}

Dopo un rever dell'apertore a livello di 2, il rispondente può licitare 2 \sa (Ingberman). L'apertore licita 3 \c con 16--17 HCP (mano minima), ogni altra dichiarazione è naturale con 18+ HCP. Se l'apertore licita 3 \c, il rispondente può fissare un contratto parziale a livello di 3 (a giocare).

\subsection{Stayman su 18-19 bilanciata}

Dopo 1 \c\ - 1 \j\ - 2 \sa, la licita 3 \c del rispondente è Stayman, con significati differenti a seconda del colore \j. Se il rispondente aveva licitato 1 \d, l'apertore risponde come segue.

\biddingtable{1 \c & P & 1 \d & P \\ 2 \sa & P & 3 \c & P \\ * &&&}
\begin{twocol}
 3 \d & Nessuna quarta nobile.\\
 3 \h & Quarta di \h, nega la quarta di \s.\\
 3 \s & Quarta di \s, nega la quarta di \h.\\
 3 \sa & Entrambe le quarte nobili.
\end{twocol}

Se il rispondente aveva licitato 1 \h, l'apertore risponde come segue.

\biddingtable{1 \c & P & 1 \h & P \\ 2 \sa & P & 3 \c & P \\ * &&&}
\begin{twocol}
 3 \d & Al più due carte di \h, al più tre carte di \s.\\
 3 \h & Terza di \h, nega la quarta di \s.\\
 3 \s & Quarta di \s, nega la terza di \h.\\
 3 \sa & Terza di \h e quarta di \s.
\end{twocol}

Se il rispondente aveva licitato 1 \s, l'apertore risponde come segue.

\biddingtable{1 \c & P & 1 \s & P \\ 2 \sa & P & 3 \c & P \\ * &&&}
\begin{twocol}
 3 \d & Al più due carte di \s, al più tre carte di \h.\\
 3 \h & Quarta di \h, nega la terza di \s.\\
 3 \s & Terza di \s, nega la quarta di \h.\\
 3 \sa & Terza di \s e quarta di \h.
\end{twocol}

\pagebreak

\section{Sviluppi sull'apertura di 1 \sa}

\subsection{Transfer}

Con un minore almeno sesto (o anche quinto con singoli o vuoti con senno, o con pochi punti), oppure con un nobile almeno quinto, si licita come segue.

\biddingtable{1 \sa & P & * &}
\begin{twocol}
 2 \d & Transfer per le \h.\\
 2 \h & Transfer per le \s.\\
 2 \s & Transfer per le \c.\\
 3 \c & Transfer per le \d.\\
 4 \d & \emph{Texas}, transfer per le \h.\\
 4 \h & \emph{Texas}, transfer per le \s.
\end{twocol}

L'apertore deve licitare il seme richiesto.

Dopo che l'apertore ha accettato una transfer nobile licitando 2 \M, il rispondente licita come segue.

\biddingtable{1 \sa & P & 2 \j & P \\ 2 \M & P & * &}
\begin{twocol}
	Colore a livello & Forzante manche, buon seme almeno quarto.\\
	Colore a salto & Forzante manche, nobile almeno sesto, Cue-bid.\\
	2 \sa & Invitante manche, nobile esattamente quinto.\\
	3 \M & Invitante manche, nobile almeno sesto.\\
	3 \sa & Passa o correggi a 4 \M.\\
	4 \M & Leggero interesse di slam (altrimenti \emph{Texas}), nobile almeno sesto.\\
	4 \sa & Quantitativo, nobile quinto. Invita a scegliere fra Pass, 5 \M, 6 \M, 6 \sa.\\
	5 \sa & Forzante, ``Pick a Small Slam''.

\end{twocol}


\subsection{Mano bilanciata, senza quarte nobili}

\biddingtable{1 \sa & P & * &}
\begin{twocol}
 2 \sa & 7--8 HCP, invitante manche.\\
 3 \sa & 9--12 HCP, a giocare.\\
 4 \c & Gerber.\\
 4 \sa & Quantitativo: l'apertore passa con mano minima, licita 6 \sa con mano massima.
\end{twocol}


\subsection{1 \sa\ - 2 \c (Stayman)}

Richiesta di quarte nobili, 8+ HCP (anche 7 con senno). Seguono le risposte.

\biddingtable{1 \sa & P & 2 \c & P \\ * &&&}
\begin{twocol}
	2 \d & Nessuna quarta nobile. \\
	2 \h & \h quarte, ma non \s quarte. \\
	2 \s & \s quarte, ma non \h quarte. \\
	2 \sa & Entrambe le quarte nobili. \\
	3 \c & 5-3-3-2 con la quinta di \c, mano massima. \\
	3 \d & 5-3-3-2 con la quinta di \d, mano massima. \\
\end{twocol}

Esaminiamo gli sviluppi della Stayman.

\paragraph{Fit in un colore nobile.}

Con il fit su un nobile \M, dopo la risposta di 2 \M.

\biddingtable{1 \sa & P & 2 \c & P \\ 2 \M & P & * &}
\begin{twocol}
	3 \M & Seleziona \M come atout. Invitante manche. \\
	Cue-bid & A partire da 3 \c. Sottintende il nobile dell'apertore come atout. \\
	4 \M & A giocare. \\
\end{twocol}

Con il fit su un nobile dopo la risposta di 2 \sa.

\biddingtable{1 \sa & P & 2 \c & P \\ 2 \sa & P & * &}
\begin{twocol}
	3 \c & Interesse di slam. \\
	3 \d & Transfer per le \h, senza interesse di slam. \\
	3 \h & Transfer per le \s, senza interesse di slam. \\
\end{twocol}

Dopo 3 \c, l'apertore è obbligato a licitare 3 \d, il rispondente fissa l'atout con 3 \h/\s, quindi iniziano le Cue-bid.

Dopo 3 \d/\h, l'apertore licita 3 \h/\s con mano minima (il rispondente può correggere), e 4 \h/\s con mano massima (a giocare).

\paragraph{Richiesta dell'altro nobile.}
Se il rispondente ha una mano bicolore nobile 5-4, dopo la risposta di 2 \d può cercare un fit nel proprio nobile quinto \M licitando \underline{l'altro nobile}. L'apertore risponde come segue.

\biddingtable{1 \sa & P & 2 \c & P \\ 2 \d & P & 2 \M & P \\ * &&&}
\begin{twocol}
	2 \sa & Nega il fit a \M, mano minima. \\
	3 \M & Fit a \M, mano minima. \\
	3 \sa & Nega il fit a \M, mano massima.\\
	Cue-bid & Fit a \M, mano massima. \\
\end{twocol}


\paragraph{Gioco a SA.}
Per giocare a SA il rispondente licita:

\biddingtable{1 \sa & P & 2 \c & P \\ 2 \j & P & * &}
\begin{twocol}
	2 \sa & 7--8 HCP, invitante manche. \\
	3 \sa & A giocare. \\
	4 \c & Richiesta d'Assi (Gerber).
\end{twocol}

\paragraph{Fit miracoloso nel minore quinto.}
Dopo la risposta di 3 \c/\d, il rispondente può confermare tale atout licitando una Cue-bid (a partire da 4 \d/\h, con senno). Le licite a livello di 3 a colore sono richieste di fermo, 3 \sa è giocare.

\subsection{1 SA - 3 \d (Spiderman)}

Bicolore nobile almeno 5-5, 8+ HCP (richiesta di terze nobili). Seguono le risposte.

\biddingtable{1 \sa & P & 3 \d & P \\ * &&&}
\begin{twocol}
	3 \M & Mano minima, \M almeno terzo.\\
	3 \sa & Mano massima, nobili entrambi terzi o entrambi quarti.\\
	4 \c & Mano massima, \h quarte, tre carte di \s.\\
	4 \d & Mano massima, \s quarte, tre carte di \h.\\
	4 \h & Mano massima, \h terze o quarte, due carte di \s.\\
	4 \s & Mano massima, \s terze o quarte, due carte di \h.
\end{twocol}

Dopo la risposta di 3 \sa, il rispondente licita come segue.

\biddingtable{1 \sa & P & 3 \d & P \\ 3 \sa & P & * &}
\begin{twocol}
 4 \m & Cue-bid, interesse di slam. L'apertore licita il nobile migliore fissando l'atout.\\
 4 \M & A giocare
\end{twocol}


\pagebreak

\section{Sviluppi sulle aperture forti bilanciate}

\subsection{Transfer} Sull'apertura di 2 \sa, con un minore almeno sesto (o anche quinto con singoli o vuoti con senno, o con pochi punti), oppure con un nobile almeno quinto, si licita come segue.

\biddingtable{2 \sa & P & * &}
\begin{twocol}
 3 \d & Transfer per le \h.\\
 3 \h & Transfer per le \s.\\
 3 \s & Transfer per le \c.\\
 4 \c & Transfer per le \d.\\
\end{twocol}

L'apertore deve licitare il seme richiesto.

In caso di nobile esattamente quinto, il rispondente licita 3 \sa: l'apertore sceglie se giocare a \sa o ad atout.

In alternativa, il rispondente può proseguire passando, licitando manche o slam (a giocare), iniziando le Cue-bid, oppure chiedendo gli Assi (4 \sa).

\subsection{2 \sa\ - 3 \c (Puppet Stayman)} La Puppet Stayman si applica sia dopo l'apertura di 2 \sa, sia dopo la sequenza 2 \c\ - 2 \d\ - 2 \sa. L'apertore risponde come segue.

\biddingtable{2 \sa & P & 3 \c & P \\ * &&&}
\begin{twocol}
	3 \d & Almeno una quarta nobile.\\
	3 \h & 5-3-3-2 con la quinta di \h.\\
	3 \s & 5-3-3-2 con la quinta di \s.\\
	3 \sa & Nessuna quarta né quinta nobile.\\
\end{twocol}

Dopo qualsiasi risposta dell'apertore, 4 \c è sempre Gerber.

Dopo la risposta 3 \d il rispondente licita come segue.

\biddingtable{2 \sa & P & 3 \c & P \\ 3 \d & P & * &}
\begin{twocol}
  3 \h & Quarta di \s.\\
  3 \s & Quarta di \h.\\
  3 \sa & A giocare (cercava una quinta nobile).\\
  4 \d & Entrambe le quarte nobili.\\
\end{twocol}

\pagebreak

\section{Sviluppi sull'apertura di 2 \c}

Ricordiamo che l'apertura di 2 \c si effettua con 22+ HCP e mano bilanciata, oppure con una mano (anche sbilanciata) con cui si può giocare la manche anche se il compagno ha delle pessime carte. È l'unica apertura sulla quale il compagno è obbligato a rispondere, anche con 0 HCP.

Dopo l'apertura di 2 \c ogni licita sotto la manche è automaticamente forzante (con unica eccezione 2 \c\ - 2 \d\ - 2 \sa, vedi oltre).

Il rispondente licita come segue.

\biddingtable{2 \c & P & * &}
\begin{twocol}
  2 \d & 0--6 HCP o mano inadatta ad altre licite.\\
  2 \M & 7+ HCP, buon nobile almeno quinto.\\
  2 \sa & 7+ HCP, mano bilanciata.\\
  3 \m & 7+ HCP, buon minore almeno sesto.\\
  3 \M & Interesse di slam, seme almeno sesto con due o più onori.
\end{twocol}

Sulla risposta di 2 \d, l'apertore licita come segue.

\biddingtable{2 \c & P & 2 \d & P \\ * &&&}
\begin{twocol}
  2 \M & Mano sbilanciata forzante manche, seme almeno quinto.\\
  2 \sa & Mano bilanciata, non garantisce manche.\\
  3 \m & Mano sbilanciata forzante manche, seme almeno quinto.\\
  3 \M & Interesse di slam, stabilisce l'atout e chiede di partire con le Cue-bid.\\
  3 \sa & Mano bilanciata.\\
  4 \m & Interesse di slam, stabilisce l'atout e chiede di partire con le Cue-bid.\\
  4 \M & Mano minima, a giocare.\\
  5 \m & Mano minima, a giocare.
\end{twocol}

Dopo la sequenza 2 \c\ - 2 \d\ - 2 \sa (22+ HCP, mano bilanciata), si applicano le convenzioni presenti sull'apertura diretta di 2 \sa (Transfer, Puppet Stayman).

\pagebreak

%\section{Sviluppi sull'apertura di 2 \d (Multicolor)}
%
%Ricordiamo che l'apertura di 2 \d multicolor si effettua con 6--10 HCP e una sesta nobile, oppure con 16+ HCP e una mano 4-4-4-1.
%
%Dopo l'apertura di 2 \d, il rispondente licita come segue.
%
%\biddingtable{2 \d & P & * &}
%\begin{twocol}
%	2 \h & Mano debole, ``passa o correggi''.\\
%	2 \s & Mano invitante con supporto a \h.\\
%	2 \sa & 14+ HCP, interrogativo.\\
%	3 \m & Minore almeno sesto, a giocare.\\
%	3 \h & Mano invitante con supporto a \s.
%	%Non è chiaro come proseguire dopo 4 \m & Buon minore almeno quinto, supporto in entrambi i nobili.
%\end{twocol}
%
%Dopo la risposta di 2 \h, l'apertore licita come segue.
%\biddingtable{2 \d & P & 2 \h & P \\ * &&&}
%\begin{twocol}
%	Pass & Sesta di \h.\\
%	2 \s & Sesta di \s.\\
%	2 \sa & Tricolore forte con singolo a \s.\\
%	3 \c & Tricolore forte con singolo a \c.\\
%	3 \d & Tricolore forte con singolo a \d.\\
%	3 \h & Tricolore forte con singolo a \h.\\
%\end{twocol}
%
%Dopo la risposta di 2 \s, l'apertore licita come segue.
%\biddingtable{2 \d & P & 2 \s & P \\ * &&&}
%\begin{twocol}
%	Pass & Sesta di \s.\\
%	2 \sa & Tricolore forte.\\
%	3 \h & Sesta di \h, mano minima.\\
%	4 \h & Sesta di \h, mano massima.\\
%\end{twocol}
%
%Dopo la risposta di 2 \sa, l'apertore licita come segue.
%\biddingtable{2 \d & P & 2 \sa & P \\ * &&&}
%\begin{twocol}
%	3 \c & Sesta di \h, mano massima.\\
%	3 \d & Sesta di \s, mano massima.\\
%	3 \h & Sesta di \h, mano minima.\\
%	3 \s & Sesta di \s, mano minima.\\
%	3 \sa & Tricolore forte con singolo a \h o a \s.\\
%	4 \c & Tricolore forte con singolo a \c.\\
%	4 \d & Tricolore forte con singolo a \d.\\
%\end{twocol}
%
%Dopo una risposta forte, il rispondente licita come segue. Dopo 3 \sa:
%\biddingtable{2 \d & P & 2 \sa & P \\ 3 \sa & P & * &}
%\begin{twocol}
%	4 \m & Richiesta d'Assi con atout \m.\\
%	4 \M & Richiesta d'Assi con atout \M (risposte RKCB). L'apertore licita 4 \sa con singolo a \M, su cui il rispondente può licitare 5 \c (Gerber).\\
%\end{twocol}
%
%Dopo 4 \c:
%\biddingtable{2 \d & P & 2 \sa & P \\ 4 \c & P & * &}
%\begin{twocol}
%	4 \d & Richiesta d'Assi con atout \d.\\
%	4 \M & Richiesta d'Assi con atout \M.\\
%	4 \sa & \note{TODO: Decidere come usare questo slot e altro a livello di 5.}\\ %TODO
%	5 \c & Gerber.
%\end{twocol}
%
%Dopo 4 \d:
%\biddingtable{2 \d & P & 2 \sa & P \\ 4 \d & P & * &}
%\begin{twocol}
%	4 \M & Richiesta d'Assi con atout \M.\\
%	4 \sa & Richiesta d'Assi con atout \c.\\
%	5 \c & Gerber.\\
%	& \note{TODO: decidere come usare altri slot a livello di 5.} %TODO
%\end{twocol}

% TODO: Valutare questa proposta alternativa.
%Dopo la risposta di 2 \sa, l'apertore licita come segue.
%\biddingtable{2 \d & P & 2 \sa & P \\ * &&&}
%\begin{twocol}
%	3 \c & Sesta di \h, mano massima.\\
%	3 \d & Sesta di \s, mano massima.\\
%	3 \h & Sesta di \h, mano minima.\\
%	3 \s & Sesta di \s, mano minima.\\
%	3 \sa & Tricolore forte con singolo a \c.\\
%	4 \c & Tricolore forte con singolo a \d.\\
%	4 \d & Tricolore forte con singolo a \h.\\
%	4 \h & Tricolore forte con singolo a \s.
%\end{twocol}
%
%Dopo una risposta forte, il rispondente licita il gradino successivo (il vuoto dell'apertore a livello di 4) con forte interesse di slam. Ogni altra licita mostra mano minima: un altro colore è naturale e fissa l'atout, 4 \sa è naturale a giocare.
%
%Con mano superiore al minimo l'apertore può licitare a sua volta il gradino successivo come richiesta d'assi (5 \c Gerber su 4 \sa, kickback RKCB negli altri casi).
% TODO: Dopo la licita ``interesse di slam'' del rispondente, l'apertore cosa fa? Mostra gli Assi? Fa una Cue-bid?

%\section{Sviluppi sull'apertura di 2 \h o 2 \s (Muiderberg)}
%
%Ricordiamo che l'apertura di 2 \M Muiderberg si effettua con 6--10 HCP, la quinta nel nobile licitato, ed una quarta minore laterale.
%
%Dopo l'apertura di 2 \M, il rispondente licita come segue.
%
%\biddingtable{2 \M & P & * &}
%\begin{twocol}
%	2 \s & Naturale, non forzante (solo dopo 2 \h).\\
%	2 \sa & 14+ HCP, interrogativo.\\
%	3 \c & Supporto in entrambi i minori, ``passa o correggi''.\\
%	3 \d & Fit a \M, invitante.\\
%	3 \M & Fit a \M, barrage.\\
%	3 \sa & A giocare.\\
%	4 \M & A giocare.
%\end{twocol}
%
%Dopo la risposta di 2 \sa, l'apertore licita come segue.
%\biddingtable{2 \M & P & 2 \sa & P \\ * &&&}
%\begin{twocol}
%	3 \c & \c almeno quarte, mano minima.\\
%	3 \d & \d almeno quarte, mano minima.\\
%	3 \h & \c almeno quarte, mano massima.\\
%	3 \s & \d almeno quarte, mano massima.
%\end{twocol}
%
%\pagebreak

\section{Sviluppi sulle sottoaperture}

\subsection{Sviluppi sull'apertura di 2 \M}

Su una sottoapertura 2 \M, le Cue-bid si iniziano solo a salto. Dichiarare 4 \sa sottintende il seme dell'apertore come atout. Ogni altra dichiarazione a colore è naturale (con 14+ HCP e seme almeno quinto) e forzante 1 giro.

Se il rispondente ha 14+ HCP può dichiarare 2 \sa (Ogust). Seguono le risposte.
\biddingtable{2 \M & P & 2 \sa & P \\ * &&&}
\begin{twocol}
 3 \c & Mano minima, carte deboli a \M.\\
 3 \d & Mano minima, carte buone a \M.\\
 3 \h & Mano massima, carte deboli a \M.\\
 3 \s & Mano massima, carte buone a \M.\\
 3 \sa & A, K, Q di \M.
\end{twocol}

Se dopo 2 \sa c'è un intervento, Pass dà mano minima, X o XX mano massima.

\subsection{Sviluppi sull'apertura di 2 \d Miniera} \label{miniera} % Questa, amico mio... è una miniera!

Ricordiamo che la sottoapertura di 2 \d Miniera si effettua con una mano bicolore nobile almeno 5-4 con 6--10 HCP. Il rispondente licita come segue.
\biddingtable{2 \d & P & * &}
\begin{twocol}
	Pass & Misfit nei nobili e \d almeno seste.\\
	2 \M & Sign-off.\\
	2 \sa & Interrogativo (vedi oltre).\\
	3 \c & Naturale, \c almeno seste, non forzante.\\
	3 \d & Invitante manche con i nobili 3-3.\\
	3 \M & Debole (barrage).\\
	Altro & Naturale.
\end{twocol}

Se dopo l'apertura di 2 \d l'avversario licita Contre, il Surcontre chiede il nobile più lungo (o migliore). Le altre risposte sono invariate.

Dopo il 2 \sa interrogativo l'apertore risponde come segue.
\biddingtable{2 \d & P & 2 \sa & P \\ * &&&}
\begin{twocol}
	3 \c & Mano minima, nobili 5-4.\\
	3 \d & Mano minima, nobili 5-5.\\
	3 \h & Mano massima, \h almeno quinte, \s quarte.\\
	3 \s & Mano massima, \s almeno quinte, \h quarte.\\
	3 \sa & Mano massima, nobili 5-5.
\end{twocol}

Dopo la risposta di 3 \c (mano minima), il rispondente può interrogare ulteriormente licitando 3 \d (forzante manche); ogni altra licita è non forzante. Seguono le risposte.
\biddingtable{2 \d & P & 2 \sa & P \\ 3 \c & P & 3 \d & P \\ * &&&}
\begin{twocol}
	3 \h & \h almeno quinte, \s quarte.\\
	3 \s & \s almeno quinte, \h quarte.
\end{twocol}

Dopo un'altra risposta fra 3 \d e 3 \sa, il rispondente può licitare 3 \sa (se disponibile) o 4 \M per chiudere a manche. Per fissare l'atout in modo forzante, sulla risposta di 3 \h si licita 3 \s oppure una cue-bid. In tutti gli altri casi, dove non è possibile fissare in modo naturale, il rispondente licita \alert{4 \c} (fit a \h) e \alert{4 \d} (fit a \s), chiedendo al rispondente di iniziare le cue-bid.

\subsection{Sviluppi sull'apertura di 3 \sa (Gambling)}

Il rispondente licita come segue.

\biddingtable{3 \sa & P & * &}
\begin{twocol}
 Pass & Fermi negli altri tre semi, a giocare.\\
 4/5/6 \c & ``Passa o correggi''.\\
 \alert{4 \d} & Richiesta di singoli o vuoti: il rispondente ha visuale di slam, ma non controlla un seme laterale.\\
 4 \M & A giocare.\\
 4 \sa & Richiesta di plusvalore (8\textsuperscript{a} carta o dama laterale) per il 6 \sa.
\end{twocol}

Dopo la dichiarazione di 4 \d, l'apertore licita come segue.

\biddingtable{3 \sa & P & 4 \d & P \\ * &&&}
\begin{twocol}
 4 \M & Singolo o vuoto a \M.\\
 4 \sa & Singolo o vuoto nell'altro minore.\\
 5 \m & 7-2-2-2, dichiara il proprio minore.
\end{twocol}

Dopo una risposta 4 \j che dà il singolo/vuoto, l'apertore licita come segue.

\biddingtable{3 \sa & P & 4 \d & P \\ 4 \j & P & * &}
\begin{twocol}
 5 \c & ``Passa o correggi''.\\
 5 \j & Invita l'apertore a dichiarare piccolo slam con il singolo, grande slam con il vuoto.
\end{twocol}

Sulla richiesta di plus valore (4 \sa), l'apertore licita come segue.

\biddingtable{3 \sa & P & 4 \sa & P \\ * &&&}
\begin{twocol}
 Pass & Colore settimo e nessuna Q laterale.\\
 5 \j & Q di \j. Il rispondente può licitare 5 \sa (a giocare) se la Q non è utile.\\
 6 \m & Colore almeno ottavo.
\end{twocol}



\pagebreak

\section{Verso lo slam}

\subsection{Cue-bid}

Non si gioca se il contratto sarà a SA, e comunque occorre avere già stabilito l'atout (esplicitamente o implicitamente). A partire da 3 \s e salendo gradualmente si dichiara un seme \j (che non sia l'atout) se si possiede una delle seguenti (Cue-bid debole): Asso di \j, K di \j, singolo o vuoto a \j.

Saltare un seme disponibile è come dichiarare che non si ha la Cue-bid in quel seme. Dichiarare SA è come dichiarare l'ultimo seme detto dal compagno, ma sono esclusi 4 SA e 5 SA (richiesta d'Assi e di Re). L'unica possibilità è quindi 3 \j\ - 3 SA, che dichiara Cue-bid (debole) a \j per entrambi i giocatori. Se un giocatore ripete un seme \j su cui ha già dato una Cue-bid, possiede una delle seguenti (Cue-bid forte): Asso di \j, vuoto a \j.

Quando un giocatore nega la Cue-bid in un seme, il compagno dovrebbe immediatamente chiudere a manche se a sua volta non possiede la Cue-bid in quel seme.
Come corollario, continuare le Cue-bid dopo che il compagno ne ha negate alcune, equivale a dichiarare (implicitamente) le Cue-bid negate dal compagno.

Esempio (con atout \h): 3 \s\  (Cue-bid debole a \s) - 4 \d\ (nega la Cue-bid a \s e a \c) - 4 \s (Cue-bid forte a \s e debole a \c) - 5 \h\ (nega la Cue-bid forte a \d) - 6 \c\ (Cue-bid forte a \c) - \dots \note{Questo esempio ha un bug!}

\paragraph{Cue-bid avanzate} Se \j è un seme diverso dall'atout, un giocatore A si dice \j-completo se ha dato Cue-bid forte a \j oppure ha già negato una Cue-bid a \j. Se A è \j-completo, \j diventa Cue-bid nel colore in cui ha dato meno informazioni (in caso di parità, il più basso in rango). Come corollario, se A salta \j sta negando una Cue-bid.
\vspace{4mm}

Talvolta è opportuno interrompere le Cue-bid per effettuare una richiesta d'Assi.


\subsection{Richieste d'Assi e di Re a colore (RKCB)}

Ai fini delle richieste d'Assi e di Re, il K di atout è considerato un Asso. In questo caso quindi ci sono 5 Assi e 3 Re. Seguono le risposte alla richiesta d'Assi (4 \sa).

\begin{twocol}
5 \c & 0 o 3 Assi.\\
5 \d & 1 o 4 Assi.\\
5 \h & 2 o 5 Assi, senza Q di atout.\\
5 \s & 2 o 5 Assi, con Q di atout.\\
\end{twocol}

\note{Se la coppia è d'accordo, è possibile invertire i significati di 5 \c e 5 \d, solo sui nobili oppure con atout qualsiasi.}

Se si riceve una risposta che non dice nulla sulla Q di atout, si può licitare il primo colore disponibile che non sia l'atout per chiedere la Q di atout. Seguono le risposte.

\begin{itemize}
 \item La prima licita ad atout disponibile: niente Q di atout.
 \item La prima licita a \j disponibile (ma necessariamente inferiore a 6 nel colore di atout): Q di atout e K di \j.
 \item 5 \sa: Q di atout, e nessun K oppure nessun K dichiarabile secondo il punto precedente.
\end{itemize}

In alternativa alla richiesta di Q di atout, con 5 \sa si può effettuare la richiesta di Re. Seguono le risposte.

\begin{twocol}
6 \c & Nessun Re.\\
6 \d & 1 Re.\\
6 \h & 2 Re.\\
6 \s & 3 Re.
\end{twocol}

%\subsection{Kickback RKCB}
%
%\note{Questa convenzione non è attiva di default, mettersi d'accordo col partner.}
%
%In alternativa alla normale RKCB, è possibile adottare questa convezione per risparmiare livelli (soprattutto sui minori). Ad atout \j chiaramente fissato (perché è stato dato un fit, oppure dopo una transfer su un'apertura forte bilanciata), è possibile utilizzare la licita immediatamente successiva a 4 \j come RKCB. Le risposte sono le stesse che sulla normale RKCB (14-03), a gradini.
%
%Se si usa la Kickback, la licita di 4 \sa assume il significato prima occupato dalla nuova richiesta d'Assi (tipicamente Cue-bid in quel colore).

\subsection{Exclusion KCB}

Dopo aver stabilito un fit a \j, una licita a salto che superi 4 \j è l'Exclusion KCB, e chiede le carte chiave ad eccezione dell'Asso del colore licitato. Le risposte sono:

\begin{twocol}
Primo step & 0 o 3 Assi.\\
Secondo step & 1 o 4 Assi.\\
Terzo step & 2 Assi senza Q di atout.\\
Quarto step & 2 Assi con Q di atout.
\end{twocol}

La richesta di Q di atout successiva funziona identica all'RKCB normale.

L'XKCB si può occasionalmente fare dando il fit, a patto che sia chiaro che non possa essere naturale. Ad esempio, 1 \s\ - 2 \h\ - 5 \c è XKCB a \h.

\subsection{Su intervento avversario (DOPI/ROPI)}

DOPI (risp. ROPI) è una sigla che significa ``Double (risp. Redouble) 0, Pass 1''. Se gli avversari intervengono dopo una richiesta d'Assi (per esempio con un Contre per chiedere l'attacco, oppure con un colore lungo in barrage), è possibile rispondere come segue.

\begin{twocol}
	X (o XX) & 0 o 3 Assi.\\
	Pass & 1 o 4 Assi.\\
	1\textsuperscript{o} gradino & 2 o 5 Assi, senza Q di atout.\\
	2\textsuperscript{o} gradino & 2 o 5 Assi, con Q di atout.
\end{twocol}

Se si riceve una risposta che non dice nulla sulla Q di atout, si può licitare Contre (o Surcontre), se disponibile, per chiedere la Q di atout. Altrimenti, si effettua la richiesta licitando il colore più economico che non sia l'atout. Le risposte sono le stesse del caso senza intervento avversario.

\subsection{Richieste d'Assi e di Re a \sa (Gerber)}

La richiesta d'Assi si effettua licitando 4 \c, dopo che è stato concordato di giocare a \sa. Seguono le risposte.
\begin{twocol}
  4 \d & 0 o 4 Assi.\\
  4 \h & 1 Asso.\\
  4 \s & 2 Assi.\\
  4 \sa & 3 Assi.
\end{twocol}

Dopo la richiesta d'Assi, si può effettuare la richiesta di Re licitando 5 \c. Seguono le risposte.
\begin{twocol}
  5 \d & 0 o 4 Re.\\
  5 \h & 1 Re.\\
  5 \s & 2 Re.\\
  5 \sa & 3 Re.
\end{twocol}


\subsection{5 \sa ``Pick A Small Slam''}

In una situazione in cui non vuole già dire qualcos'altro (richiesta di Re, a giocare dopo 5 \c, Cue-bid, ...), la licita di 5 \sa (preferibilmente a salto) invita a dichiarare in modo naturale proponendo un piccolo slam. Si prosegue la licita in modo naturale.



\pagebreak

\section{Interventi}

Dopo un'apertura a livello di 1 degli avversari, sono previsti i seguenti interventi.

\begin{twocol}
  Contre & 12+ HCP, (indicativamente) al massimo 2 carte nei colori licitati dagli avversari, almeno 3 negli altri; se gli avversari hanno licitato un nobile, quattro carte nell'altro. \\
  & In alternativa, 16+ HCP e distribuzione qualsiasi.\\
  1 \j & 8+ HCP, \j almeno quinto.\\
  1 \sa & Come apertura, con fermo nel palo degli avversari.\\
  2 \j a livello & 11+ HCP, \j almeno quinto.\\
  2 \j a salto & Come apertura.\\
  3 \j, 4 \j & Come apertura.
\end{twocol}

\subsection{Mani bicolore}

I seguenti interventi si applicano con una mano bicolore (almeno 5-5) e 12+ HCP. Tenere in considerazione le vulnerabilità.

Sull'apertura di 1 \c degli avversari si licita come segue.

\biddingtable{1 \c & * &&}
\begin{twocol}
  2 \c & Naturale: 11+ HCP, \c almeno quinti.\\
  2 \sa & Bicolore \d e un seme nobile. Il compagno può licitare 3 \c (richiesta del seme nobile), 3 \d (a giocare).\\
  3 \c & Bicolore \h -\s. Il compagno può licitare 3 \h/\s (a giocare), ogni altra licita è forzante manche.
\end{twocol}

Sull'apertura di 1 \d degli avversari si licita come segue.

\biddingtable{1 \d & * &&}
\begin{twocol}
  2 \d & Bicolore \h -\s.\\
  2 \sa & Bicolore \c e un seme nobile. Il compagno può licitare 3 \c (a giocare), 3 \d (richiesta del nobile).
\end{twocol}

Sull'apertura di 1 \M degli avversari si licita come segue.

\biddingtable{1 \M & * &&}
\begin{twocol}
  2 \M & Bicolore altro nobile e minore. Il compagno può licitare 2 \sa (richiesta del minore).\\
  2 \sa & Bicolore minore.
\end{twocol}

\subsection{Multi landy}

I seguenti interventi si applicano dopo l'apertura di 1 \sa degli avversari. Tenere in considerazione le vulnerabilità.
\begin{twocol}
 Contre & 15--17 HCP, mano bilanciata (punitivo).\\
 2 \c & 12+ HCP, bicolore nobile.\\
 2 \d & 6--10 HCP, seme nobile almeno sesto.\\
 2 \M & 12+ HCP, \M almeno quinto, una quarta minore laterale. Il compagno può licitare 2 \sa (richiesta del minore).\\
 2 \sa & 12+ HCP, bicolore minore.\\
\end{twocol}

Dopo il Contre, il compagno del contrante licita come dopo apertura di 1 \sa (Transfer, Stayman).


\subsection{Sviluppi dopo il Contre informativo}

\paragraph{A livello di 1.} Dopo 1 \j\ - X - Pass, il compagno del contrante licita come segue.

\biddingtable{1 \j & X & P & *}
\begin{twocol}
	Colore a livello & 0--7 HCP, colore almeno quarto (anche terzo a livello di 1 con mano molto debole).\\
	Colore a salto & 8--11 HCP, colore almeno quarto, invitante.\\
	1 \sa & 8--10 HCP, mano bilanciata, fermo nel colore avversario.\\
	2 \sa & 11--12 HCP, mano bilanciata, buon fermo nel colore avversario.\\
	2 \j & 12+ HCP, forzante due giri.
\end{twocol}

Dopo 1 \j\ - X - XX, il compagno del contrante licita come segue.

\biddingtable{1 \j & X & XX & *}
\begin{twocol}
	Pass & 0--7 HCP. \\
	Colore a livello & 8--11 HCP, colore almeno quarto.\\
	Colore a salto & Barrage, colore almeno sesto (anche quinto con senno).\\
	1 \sa & 8--10 HCP, mano bilanciata, buon fermo nel colore avversario.\\
	2 \j & Mano tricolore con singolo o vuoto a \j.
\end{twocol}

\paragraph{A livello di 2 (Lebensohl)} Dopo 2 \j\ - X - Pass, il compagno del contrante licita come segue.

\biddingtable{2 \j & X & P & *}
\begin{twocol}
	Colore a livello di 2 & 0--7 HCP, colore almeno quarto. \\
	2 \sa & Artificiale forzante, vedi oltre. \\
	Colore a livello di 3, a livello & 8--11 HCP, colore almeno quarto. \\
	Colore a livello di 3, a salto & 12+ HCP, colore almeno quarto, forzante manche. \\
	3 \sa & 12+ HCP, mano bilanciata, \underline{senza fermo} nel colore avversario.
\end{twocol}

Sulla risposta di 2 \sa, il contrante è obbligato a licitare 3 \c senza Rever, ma può licitare naturalmente un colore a livello di 3 con Rever. Se il contrante licita un nuovo colore, gli sviluppi sono naturali. Dopo 3 \c, invece, il compagno del contrante risponde come segue.

\biddingtable{2 \j & X & P & 2 \sa \\ P & 3 \c & P & *}
\begin{twocol}
	Pass & 0--7 HCP, almeno quattro carte di \c. \\
	Colore a livello di 3, rango minore di \j & 0--7 HCP, colore quarto (non poteva dire il colore a livello di 2). \\
	Colore a livello di 3, rango maggiore di \j & 8--11 HCP, colore quarto. \\
	3 \sa & 12+ HCP, mano bilanciata con fermo nel colore avversario.
\end{twocol}

\subsection{Riapertura del secondo di mano}

Dopo la sequenza 1 \M\ - P - 2 \M\ - P - P, il secondo di mano può licitare come segue.

\biddingtable{1 \M & P & 2 \M & P \\ P & * &&}
\begin{twocol}
	Contre & Altro nobile almeno quarto, supporto nei minori. Il compagno può dare la bicolore minore licitando 2 \sa. \\
	2 \sa & Mano bicolore minore.\\
	3 \m & Buon minore almeno sesto (anche quinto con senno).\\
\end{twocol}

\section{Interferenze avversarie}

\subsection{Interferenza avversaria a livello di 1}

Dopo un'apertura a livello di 1 e un intervento di 1 \M, il compagno dell'apertore licita come segue.
\biddingtable{1 \j & 1 \M & * &}
\begin{twocol}
	Pass & 0-7 HCP oppure niente da dichiarare.\\
	Nuovo colore a livello & 8-11 HCP, seme almeno quinto.\\
	1 \sa & 8-11 HCP, fermo nel nobile avversario.\\
	Contre & 8-11 HCP, altro nobile esattamente quarto; in alternativa: forzante manche, altro nobile almeno quarto.\\
	Surlicita & 12+ HCP senza l'altro nobile.\\
	2 \sa & Solo sulla sequenza 1 \h\ - 1 \s: \hyperref[jensen]{Jensen}.
\end{twocol}

Se l'apertore aveva la 18-19 bilanciata, licitare 2 \sa dà il fermo nel seme avversario; senza fermo si surlicita.

\subsection{Interferenza avversaria su 1 \sa}

Dopo la sequenza 1 \sa\ - X, oppure 1 \sa\ - P - P - X - P - P, il compagno dell'apertore licita come segue.

\biddingtable{1 \sa & X & * &}
\vspace{.8cm}
\biddingtable{1 \sa & P & P & X \\ P & P & * &}
\vspace{-.8cm}
\begin{twocol}
	Pass & Invita il compagno a surcontrare (nel primo caso).\\
	Surcontre & Transfer per le \c.\\
	2 \c & Transfer per le \d.\\
	2 \d & Transfer per le \h.\\
	2 \h & Tranfer per le \s.\\
\end{twocol}

\biddingtable{1 \sa & X & P & P \\ XX & P & * &}
Dopo il Surcontre dell'apertore, il rispondente può ancora salvarsi con una 4-4 licitando il seme più economico. L'apertore passa o licita il seme successivo (passa o correggi).

Dopo la sequenza 1 \sa\ - 2 \c, il compagno dell'apertore licita Contre per la Stayman, ogni altra licita è ``system on''.

\paragraph{In generale.}

Dopo la sequenza 1 \sa\ - 2 \c (che non sia bicolore nobile, vedi sotto), il compagno dell'apertore licita Contre per la Stayman, ogni altra licita è ``system on''.

Dopo un intervento di almeno 2 \d, che mostra un colore \rj, 2 \sa è un relay per 3 \c. Il rispondente licita come segue:

\biddingtable{1 \sa & 2 \rj & * &}
\begin{twocol}
  Contre & Punitivo. \\
	2 \j & A giocare.\\
	3 \j & Transfer per il seme successivo diverso da \rj, almeno invitante.\\
  2 \sa + Pass & Per giocare 3 \c.\\
	2 \sa + 3 \j & Per giocare 3 \j (\j di rango minore di \rj).\\
  3 \rj & Almeno una quarta nobile, Stayman. Nega il fermo a \rj.\\
	2 \sa + 3 \rj & Almeno una quarta nobile, Stayman. Dà il fermo a \rj.\\
	3 \sa & Senza i nobili, senza fermo.\\
  2 \sa + 3 \sa & Senza i nobili, con fermo.\\
  4 \j & Systems on (Gerber, Texas)
\end{twocol}

\subsection{Interferenza avversaria bicolore}

Se gli avversari intervengono con una bicolore in cui entrambi i colori sono noti:
\begin{twocol}
	Contre & Possesso del Board (10+), intenzione di punire uno o entrambi i semi. Un successivo passo del rispondente sottintende almeno quattro carte nel quarto seme. \\
	\alert{Surlicita alta} & Fit almeno invitante.\\
	\alert{Surlicita bassa} & Mano quasi forzante manche, cinque carte nel quarto seme.\\
	Fit a livello & Debole, competitivo, NF.\\
	Quarto seme & Seme almeno sesto, NF (come sottoapertura).
\end{twocol}

Se gli avversari intervengono con una bicolore in cui un solo colore è noto:
\begin{twocol}
	Contre & Possesso del Board (10+), intenzione di punire almeno due dei tre semi mancanti. Un successivo passo del rispondente sottintende almeno quattro carte negli altri due semi. \\
	\alert{Surlicita} & Fit almeno invitante.\\
	Fit a livello & Debole, competitivo, NF.\\
	Nuovo colore & Come da sottoapertura a livello di 2, NF.
\end{twocol}

\subsection{Situazioni competitive ad alto livello}

Quando si arriva in una situazione competitiva ad alto livello e una linea ha chiaramente la maggioranza dei punti, oppure ha dichiarato spontaneamente manche a sfavore di zona, se l'avversario difende, il primo che deve dichiarare licita come segue.

\begin{twocol}
	Pass & Mano minima oppure interesse di slam; chiede al compagno di contrare. Se, dopo il Contre del compagno, si continua a dichiarare, si sta	mostrando una mano da slam, normalmente con Cue-bid forte nel colore avversario.\\
	Contre & Cue-bid debole nel colore avversario, buona mano senza interesse di slam. Chiede al compagno di rialzare, a meno che non abbia mano minima o punti sprecati nel colore avversario.\\
	Rialzo & Mano debole sbilanciata, senza interesse di slam.
\end{twocol}


\pagebreak

\section{Gioco della carta}

\subsection{Attacco}

\paragraph{Principi generali} Un attacco di Asso promette almeno il K, un attacco di K promette almeno la Q, un attacco di Q o meno promette almeno tre onori in sequenza.

Se si attacca con scartina e il seme viene rigiocato al secondo giro, scarto ascendente promette un numero dispari di carte nel seme, scarto discendente promette un numero pari di carte nel seme.


\paragraph{Senza atout} Si attacca preferibilmente in sequenza di onori in un seme almeno quarto (anche terzo in mancanza di attacchi migliori).

In alternativa si attacca con l'ultima scartina in posizione dispari del seme più lungo (terza carta se il seme è terzo o quarto, quinta carta se il seme è quinto o sesto, etc.), evitando di attaccare sotto Asso o Re se il seme è al più quarto. Questo permette di rispettare la direzione ascendente/discendente descritta nei principi generali in caso di ritorno nello stesso seme.

\paragraph{Con atout} La preferenza dipende dalla licita e dalla distribuzione della mano. Si può attaccare con singolo o doubleton, con sequenza di onori. Un attacco ad atout promette al più due carte nel seme di atout, senza onori maggiori.

\subsection{Scarto}

Sull'attacco di un onore del compagno, una carta bassa segnala valori nel seme, una carta alta rifiuta il seme.

In caso di scarto su seme diverso, una carta bassa dispari segnala preferenza per il seme scartato, una carta pari disdegna il seme scartato. Uno scarto particolarmente alto (J, 10, 9) segnala un pessimo seme. Inoltre, uno scarto pari basso indica preferenza per un seme di rango più basso, uno scarto pari alto indica preferenza per un seme di rango più alto. Questo è tanto più indicativo quanto più è noto che il giocatore ha grande possibilità di scelta nel seme scartato.

\end{document}
