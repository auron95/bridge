\documentclass[a4paper,10pt]{article}


\usepackage[utf8]{inputenc}
\usepackage[italian]{babel}
\usepackage[pdftex]{graphicx}
\usepackage{amsmath}
\usepackage{amsthm}
\usepackage{fancyhdr}
\usepackage{amsfonts}
\usepackage{amssymb}
\usepackage{xspace}
\usepackage{color}

\topmargin -1cm
\oddsidemargin -0.5cm
\textwidth 17cm


\renewcommand{\c}{$\clubsuit$\xspace}
\renewcommand{\d}{$\diamondsuit$\xspace}
\newcommand{\h}{$\heartsuit$\xspace}
\newcommand{\s}{$\spadesuit$\xspace}
\renewcommand{\j}{$\bigstar$\xspace}
\newcommand{\sa}{SA\xspace}

\newcommand{\smallspace}{\vskip0.3cm}

\renewcommand{\tabcolsep}{0.3cm}

\newcommand{\note}[1]{\textcolor{red}{#1}}


\newenvironment{twocol}
  {\smallspace\noindent\begin{tabular}{l p{0.8\textwidth}}}
  {\end{tabular}\smallspace}

\newenvironment{threecol}
  {\smallspace\noindent\begin{tabular}{l l p{0.7\textwidth}}}
  {\end{tabular}\smallspace}

  
% Title Page
\title{Convenzioni di bridge}
\author{Ugo Bindini \& Giovanni Paolini}

\begin{document}
\maketitle

\tableofcontents

\pagebreak
\section{Aperture}

\paragraph{Algoritmo generale.}
Si licita il colore pi\`u lungo, escludendo \s e \h se non sono almeno quinte e le \d se non sono almeno quarte.
In caso di parit\`a di lunghezza dei semi pi\`u lunghi, si dichiara il pi\`u alto se sono almeno quinti e si dichiara il pi\`u basso se sono quarti.
Nota: con 4\s\ - 4\h\ - 3\d\ - 2\c, si apre di 1 \c.

\noindent Le aperture standard (con almeno 12 HCP) si possono effettuare anche con 11 HCP se si posseggono due Assi e un Re.

\subsection{Mani bilanciate}
(4-3-3-3, 4-4-3-2, 5-3-3-2)
\smallspace

\begin{threecol}
 12-14 & 1 \j & Si licita seguendo l'algoritmo generale.\\
 15-17 & 1 \sa & Solo se la quinta non \`e nobile.\\
 18-19 & 1 \j & 5-3-3-2, \j \`e il seme quinto (anche nobile). Poi \sa a salto sulla risposta di 1 a colore.\\
       & 1 \c & Tutte le altre distribuzioni. Poi \sa a salto sulla risposta di 1 a colore. \note{Come ci si comporta su una risposta che non è di 1 a colore?}\\
 20-22 & 2 \sa & Solo se la quinta non è nobile.\\
 23+ & 2 \c & \note{Da rivedere}
\end{threecol}


\subsection{Mani monocolore}
(6-3-2-2, 6-3-3-1, 7-3-2-1, \dots)

\begin{threecol}
 12+ & 1 \j & \j \`e il colore lungo. Poi si ripete il colore, eventualmente a salto.
\end{threecol}


\subsection{Mani bicolore}
(5-4-3-1, 5-4-2-2, 6-4-2-1, 5-5-2-1, \dots)

\begin{threecol}
 12+ & 1 \j & \j \`e il colore pi\`u lungo; in caso di parit\`a (5-5 o 6-6) si licita il colore pi\`u alto (in accordo con l'algoritmo generale). Sulla ridichiarazione, per licitare il secondo colore a livello di 2 in modo ascendente (es: 1\d{} - 1\s{} - 2\h), bisogna avere 16+ HCP (Rever).
\end{threecol}


\subsection{Mani tricolore}
(4-4-4-1, 5-4-4-0)

\begin{threecol}
 12+ & 1 \j & Si licita seguendo l'algoritmo generale, con un'eccezione: se si ha una 4-4-4-1 con singolo a \s e meno di 16 HCP, si apre di 1\d anzich\'e 1\c (in modo che, sull'eventuale risposta di 1\s, si possa licitare 2\c).
\end{threecol}


\subsection{Sottoaperture}

\begin{twocol}
 2 \j & 6-10 HCP, con \j colore esattamente sesto diverso da \c.\\
 3/4 \j & Colore almeno settimo (meglio se a compagno passato).\\
 3 \sa & Minore almeno settimo chiuso (con A, K, Q) e non pi\`u di una dama negli altri tre colori.
\end{twocol}



\pagebreak

\section{Risposte sull'apertura di 1 a colore}

Si risponde tendenzialmente con $\geq 5$ punti oppure con un Asso.

\paragraph{Sviluppi ``naturali''.}
Nel caso in cui la dichiarazione che si sta effettuando sia naturale, valgono i seguenti princ\^ipi.
\begin{itemize}
 \item Licitare un colore nuovo garantisce almeno 4 carte (almeno 5 se il compagno ha dimostrato di non averne più di 3). Osservazione importante: spesso licitando un nuovo colore si allunga il colore di apertura (vedi esempi).
 
 Per esempio: 1 \d\ - 1 \sa\ - 2 \h dà almeno 6 carte di \d e 5 di \h (e Rever), perché la risposta di 1 \sa nega quarte nobili;
 1 \d\ - 1 \s\ - 2 \c dà almeno 4 carte di \d e di \c (non almeno 5 di \d perché si potrebbe essere nel caso della 4-4-4-1 con 12-15 HCP e singolo a \s, vedi apertura tricolore);
 1 \d\ - 1 \h\ - 2 \c dà almeno 5 carte di \d e 4 di \c (non si può essere nel caso della 4-4-4-1 con 12-15 HCP e singolo a \s, perché bisognerebbe dare il fit a \h licitando 2 \h);
 1 \d\ - 1 \h\ - 1 \s dà almeno 4 carte di \d e di \s;
 1 \d\ - 1 \s\ - \hbox{2 \h} dà Rever e almeno 4 carte di \d e di \h (le \h possono essere anche solo quarte, perché il rispondente può avere 4 carte di \h\ -- benché ciò sia improbabile);
 1 \c\ - 2 \c\ - 2 \h dà almeno 5 carte di \h (perché il rispondente ha negato colori quarti diversi dalle \c) e quindi almeno 6 carte di \c.
 
 \item Ripetere un colore allunga di 1 il numero di carte.
 
 Esempio: 1 \h\ - 1 \sa\ - 2 \s\ - 2 \sa\ - 3 \h dà Rever e almeno 7 carte a \h e 5 a \s (infatti 2 \s garantisce almeno 5 carte di \s perché il rispondente ne ha al massimo 3, per cui l'apertura di 1 \h diventa garanzia di almeno 6 carte di \h).
 
 {\bf Attenzione:} il caso in cui l'apertore apre di 1 \c e poi ripete le \c è un po' particolare, perché la prima dichiarazione garantisce solo 2 carte a \c. In ogni caso, la ripetizione delle \c garantisce almeno 5 carte, perché non vale la pena di ripetere le \c se si hanno solo 4 carte (piuttosto si licita \sa, che tendenzialmente fa capire che si hanno 4 carte a \c dal momento che non se ne hanno negli altri semi).
 
 Per esempio, 1 \c\ - 1 \h\ - 2 \c dà meno di 16 HCP e almeno 5 carte di \c (se fossero solo 4, ad esempio nella distribuzione 4\c\ - 4\d\ - 3\h\ - 2\s, si liciterebbe invece 1 \sa).
 Invece 1 \c\ - 1 \h\ - 3 \c dà Rever e almeno 6 carte di \c (infatti, se le carte di \c fossero solo 5, o si avrebbe un altro palo da licitare oppure la mano sarebbe bilanciata).
 
 \item Appoggiare un colore licitato dal compagno dà il fit (almeno $8-k$ carte, dove $k$ è il numero di carte che sei sicuro che il tuo compagno abbia in quel colore).
 \item Licitare un colore nobile ha la priorità sul dare il fit in un colore minore. Dare il fit su un colore nobile ha la priorità su qualsiasi altra cosa.
 
 Per esempio, se l'apertore ha 4\c\ - 4\d\ - 4\h\ - 1\s e ha aperto di 1 \c (quindi ha 16+ HCP, vedi apertura tricolore), sulla risposta di 1 \d dichiara 2 \h (Rever, \h quarte) e non 3 \d (Rever, fit a \d).
 Il fit a \d può essere dato dopo, nel caso in cui non si trovi fit a \h.
 
 \item {\bf Rever (1).} Solo per l'apertore: dopo una risposta a livello di 1, superare strettamente 2 \j (dove \j è il colore di apertura) dà Rever, e va fatto se e solo se si hanno 16+ HCP.
 In questo modo, dopo la seconda licita dell'apertore, il rispondente sa se l'apertore ha 12-15 HCP oppure 16+ HCP.
 
 {\bf Eccezione:} se l'apertore vuole dare il fit nel colore del rispondente, lo fa a livello per non dare Rever, e lo fa a salto per dare Rever. In particolare, 1 \h\ - 1 \s\ - 2 \s non è Rever.
 
 {\bf In che colore dare Rever.} Se si vuole dare Rever (16+ HCP), allora ci si dovrebbe trovare in uno dei seguenti casi: mano bilanciata (allora si è seguito l'algoritmo per le aperture con mano bilanciata); mano monocolore (si hanno almeno 6 carte nel colore, quindi si ripete quello); mano bicolore o tricolore (si licita uno degli altri colori). Purtroppo non sempre ci si trova in questi casi.
 Ad esempio si può avere una 5-3-3-2 con quinta nobile e 16-17 HCP (in questo caso si dà Rever nel migliore dei minori, anche se è solo terzo);
 oppure si può essere nel caso 1 \h\ - 1 \sa con cinque carte di \h e quattro di \s, in cui non si possono dire le \s perché il rispondente ne ha meno di 4 (anche in questo caso si dà Rever nel migliore dei minori -- anche se è solo secondo -- oppure con non tanti punti si rinuncia al Rever dichiarando 2 \sa, che è invitante manche).
 
 {\bf Morale della favola:} il Rever in un colore minore talvolta non garantisce 4 carte nel colore. È compito del rispondente ragionare e capire se ci si può trovare in uno di questi casi.
 
 Esempi di Rever: 1 \d\ - 1 \h\ - 2 \s (con \d e \s almeno quarte); 1 \h\ - 1 \sa\ - 3 \h (con \h almeno seste); \hbox{1 \h}\ - 1 \s\ - 3 \s (con \h almeno quinte e \s almeno quarte).
 
 Altro esempio: se l'apertore ha meno di 16 HCP, 6 carte di \h e 5 carte di \s, su 1 \h\ - 1 \sa deve rispondere 2 \h e non 2 \s. Questo è un po' triste, però non ci si può fare nulla.
 
 Ulteriore esempio: su 1 \d\ - 1 \s, con la distribuzione 4\d\ - 4\h\ - 3\s\ - 2\c e meno di 16 HCP, l'apertore licita 1 \sa e non 2 \h.
 
 \item {\bf Rever (2).} Anche dopo una risposta di 2 a colore che nega il fit nel colore di apertura, l'apertore deve decidere se: dare o meno il fit; dare o meno Rever.
 \begin{itemize}
  \item Per dare il fit senza dare Rever, si ripete a livello il colore del compagno. Nel caso eccezionale 1 \s\ - 2 \h, l'apertore con un fit almeno terzo può dire 3 \h con il minimo (circa 12 HCP) e 4 \h negli altri casi (circa 13-15 HCP, oppure 12 HCP con una mano sbilanciata e/o fit almeno quarto). In questo modo il rispondente può passare 3 \h nel caso in cui abbia a sua volta il minimo.
  
  Attenzione: prima di dare il fit in un minore, è bene essere ragionevolmente sicuri che non si stia perdendo il fit in un colore nobile.
  
  Esempi: 1 \s\ - 2 \d\ - 3 \d (l'apertore ha 12-15 HCP, fit quarto a \d, e tendenzialmente esattamente 5 carte di \s e al massimo 3 carte di \h);
  1 \h\ - 2 \d\ - 3 \d (l'apertore ha 12-15 HCP, fit quarto a \d, e tendenzialmente esattamente 5 carte di \h e al massimo 4 di \s\ -- infatti il rispondente ha negato 4 carte di \s, per cui l'apertore licita 2 \s solo se ne ha almeno 5).
  
  \item Per non dare il fit ma dare Rever: nel caso di una distribuzione monocolore, si ripete il proprio colore a salto, cioè a livello di 3 (es: 1 \s\ - 2 \d\ - 3 \s; \note{gestire il caso particolare 1 \c\ - 2 \c}); nel caso di una distribuzione bicolore o tricolore, si dice il secondo colore a livello di 3 (es: 1 \s\ - 2 \c\ - 3 \d, oppure 1 \s\ - 2 \d\ - 3 \c). Rimane il caso di una distribuzione bilanciata, che (data l'apertura di 1 a colore) deve rientrare in uno dei seguenti due casi: 5-3-3-2 con quinta nobile e 15-17 HCP (in realtà 16-17, se si vuole dare Rever) oppure una qualsiasi bilanciata con 18-19 HCP. Nel caso bilanciato, si dà Rever licitando 3 \sa (es: 1 \s\ - 2 \d\ - 3 \sa). \note{Questa cosa non mi piace molto: uno potrebbe voler dire 3 \sa senza Rever, per fermarsi a manche. Probabilmente va cambiato con un Rever in un minore fasullo.}
  
  \item Per non dare né fit né Rever: si licita un colore a livello di 2 (quello di apertura o un altro), oppure 2 \sa (12 HCP oppure 13 brutti) o 3 \sa.
  
  \item Per dare sia fit che Rever: si licita la prima cue-bid, a partire da 3 \s. \note{Mi pare che, partendo da 3 \s, non ci sia ambiguità con i casi precedenti.}
  
  Attenzione: non è detto che sia ragionevole dare il fit in un minore in questo modo. Infatti, le cue-bid rendono più complicato giocare a \sa (manche o slam). A seconda dei casi, può essere preferibile dare Rever fingendo di non avere il fit (soprattutto se la mano è bilanciata e il fit è corto).
 \end{itemize}

\end{itemize}


\subsection{Senza fit}

\begin{itemize}
 \item Se possibile, dire un colore a livello di 1 (il pi\`u lungo, e in caso di parit\`a come da algoritmo generale).
 \item Altrimenti:
  \begin{threecol}
    $\leq 10$ HCP & 1 \sa & \\
    $\geq 11$ HCP & 2 \h & Con le \h almeno quinte.\\
    & 2 \d & Con le \d almeno quarte, se non si è nel caso precedente.\\
    & 2 \c & In tutti gli altri casi.\\
  \end{threecol}
\end{itemize}


\subsection{Con fit su un colore maggiore \j}

\begin{twocol}
 2 \j  & 5-7 HCP, fit terzo.\\
 3 \c & 10+ HCP, fit terzo.\\
 3 \d & 8-9 HCP, fit terzo.\\
 3 \j  & 5-9 HCP, fit quarto.\\
 4 \j  & 5-9 HCP, fit almeno quinto.\\
 2 \sa & 10+ HCP, fit almeno quarto (Jordan).\\
 Cue-bid & 12+ HCP, fit terzo. Forzante manche.
\end{twocol}

\paragraph{Sviluppo su 3 \c.} L'apertore licita
\begin{twocol}
 3 \d & 12-13 HCP.\\
 3 \h & 14-15 HCP.\\
 3 \s & 16-17 HCP.\\
 3 \sa & 18+ HCP.
\end{twocol}

Il rispondente può ora valutare esattamente la forza della mano e scegliere se: fermarsi a 4 \j, licitare una Cue-bid o chiedere gli Assi.

\paragraph{Sviluppo su 2 \sa (Jordan).} L'apertore licita
\begin{twocol}
 3 \c & Mano minima (12-13 HCP).\\
 3 \d & 14+ HCP, singolo o vuoto a \d.\\
 3 \h & 14+ HCP, singolo o vuoto a \h (a \c se \h è l'atout).\\
 3 \s & 14+ HCP, singolo o vuoto a \s (a \c se \s è l'atout).\\
 3 \sa & 16+ HCP, senza singoli né vuoti.\\
 4 \c & 16+ HCP, \c quinte.\\
 4 \d & 16+ HCP, \d quinte.\\
 4 \h & 16+ HCP, \h quinte (atout \s); 14-15 HCP, senza singoli né vuoti (atout \h).\\
 4 \s & 16+ HCP, \s quinte (atout \h); 14-15 HCP, senza singoli né vuoti (atout \s).
\end{twocol}

\subsection{Con fit sulle \d}

\begin{twocol}
 1 \h/\s & Se possibile, come nel caso senza fit.\\
 2 \c  & 11+ HCP.\\
 2 \d  & 5-8 HCP.\\
 3 \d  & 9-10 HCP.\\
\end{twocol}

\subsection{Con fit sulle \c (almeno seste)}

\begin{twocol}
 1 \d/\h/\s & Se possibile, come nel caso senza fit.\\
 2 \c & 11+ HCP, come nel caso senza fit.\\
 2 \sa & 5-8 HCP (barrage).\\
 3 \c & 9-10 HCP.\\
\end{twocol}



\pagebreak

\section{Risposte sull'apertura 1 SA}

\subsection{Mano sbilanciata (Transfer)}

Almeno sei carte in un colore (o anche cinque con singoli o vuoti con senno, o con pochi punti).

\begin{twocol}
 2 \d & Transfer per le \h.\\
 2 \h & Transfer per le \s.\\
 2 \s & Transfer per le \c.\\
 3 \c & Transfer per le \d.\\
\end{twocol}

L'apertore deve licitare il seme richiesto dal rispondente (che diventa automaticamente l'atout concordata), dopodiché il rispondente può proseguire passando, licitando manche o slam (a giocare), iniziando le Cue-bid, oppure chiedendo gli Assi (4 \sa).


\subsection{Mano bilanciata, senza quarte nobili e senza ambizioni di slam}

\begin{twocol}
 2 \sa & 6-8 HCP, invitante manche.\\
 3 \sa & 9-12 HCP, a giocare.
\end{twocol}


\subsection{1 SA - 2 \c (Stayman)}

Richiesta di quarte nobili, $\geq 8$ HCP (oppure 7 con senno). Seguono le risposte.

\begin{twocol}
 2 \d & Nessuna quarta nobile e minimo dell'apertura (15 HCP).\\
 2 \h & \h quarte, ma non \s quarte.\\
 2 \s & \s quarte, ma non \h quarte.\\
 2 SA & Nessuna quarta nobile e massimo dell'apertura (17 HCP).\\
 3 \c & 4\h\ - 4\s\ - 3\c\ - 2\d.\\
 3 \d & 4\h\ - 4\s\ - 3\d\ - 2\c.\\
\end{twocol}

\noindent Esaminiamo gli sviluppi della Stayman.

\paragraph{Fit in un colore nobile.}

Con il fit su un nobile \j, dopo la risposta di 2 \j:
\begin{twocol}
 3 \j & Seleziona \j come atout. Invitante manche.\\
 4 \j & A giocare.\\
 Cue-bid & A partire da 3 \s. Sottintende il nobile dell'apertore come atout.\\
\end{twocol}

\noindent Con il fit su un nobile \j, dopo la risposta di 3 \c o 3 \d: il rispondente dichiara \j, e l'apertore inizia le Cue-bid.


\paragraph{Richiesta dell'altro nobile.}
Se il richiedente aveva una quinta nobile \j, ma non c'\`e fit quarto dell'apertore, può dichiarare il primo \j disponibile (con senno). Se l'apertore ha fit (terzo) inizia le Cue-bid, altrimenti licita \sa.

\paragraph{Gioco a SA.}
Per giocare a SA il rispondente licita:
\begin{twocol}
 2 \sa & (se possibile) 7-8 HCP, invitante manche.\\
 3 \sa & A giocare.\\
 4 \sa & Richiesta d'Assi.
\end{twocol}

\paragraph{Richiesta dei minori.} Si può effettuare dopo le risposte di 2 \d, 2 \h, 2 \s o 2 \sa, dichiarando 3 \c. È forzante manche. Seguono le risposte.

\begin{itemize}
 \item Se l'apertore ha negato quarte nobili (cioè dopo 2 \d o 2 \sa):
  \begin{twocol}
    3 \d & \d almeno quarte, \c al massimo terze (5-3-3-2 oppure 4-3-3-3).\\
    3 \h & 4\c\ - 4\d\ - 3\h\ - 2\s.\\
    3 \s & 4\c\ - 4\d\ - 3\s\ - 2\h.\\
    3 \sa & \c almeno quarte, \d al massimo terze (5-3-3-2 oppure 4-3-3-3).
  \end{twocol}

 \item Se l'apertore ha un palo nobile quarto (cioè dopo 2 \h o 2 \s):
  \begin{twocol}
    3 \d & \d quarte (4-4-3-2).\\
    3 \h & Nessun minore quarto, mano minima.\\
    3 \s & Nessun minore quarto, mano massima.\\
    3 \sa & \c quarte (4-4-3-2).
  \end{twocol}
\end{itemize}

\noindent Seguono gli sviluppi dopo la richiesta dei minori.
\begin{itemize}
 \item Per giocare a \sa: 3 \sa (o passo, sulla risposta di 3 \sa), a giocare; 4 \sa richiesta d'Assi.
 \item Se c'è fit su un minore, dopo 3 \d o 3 \sa (il fit non è ambiguo): Cue-bid nel primo maggiore disponibile, se possibile; altrimenti licitare il minore (il che nega le Cue-bid nei colori maggiori, per cui l'apertore deve chiudere a manche se non ha la copertura necessaria per sperare in uno slam).
 L'apertore prosegue con la prima Cue-bid, oppure chiudendo a manche.
 \item Se c'è fit su un minore, dopo 3 \h o 3 \s nel primo caso: licitare il minore.
 L'apertore prosegue con la prima Cue-bid.
 \item Se si vuole segnalare un minore quinto \j (in cui l'apertore non ha mostrato quattro carte): dichiarare 4 \j.
 L'apertore prosegue così:
 \begin{itemize}
  \item 4 ($\bigstar+1$) senza fit terzo nel minore. Il rispondente dichiara 4 \sa (richiesta d'Assi), oppure 4 ($\bigstar+2$) per costringere l'apertore a licitare 4 \sa (a giocare).
  \item La prima Cue-bid disponibile che non sia 4 ($\bigstar+1$), ma comunque senza superare 5 \j.
  
 \end{itemize}
\end{itemize}

\pagebreak

\section{Sviluppi su apertura forte bilanciata}

\paragraph{Transfer.} Sull'apertura 2 \sa, con mano sbilanciata il rispondente può licitare una Transfer.

\paragraph{Stayman avanzata.} Su 2 \sa (20-22 HCP), oppure 1 \c\ - 1 \j\ - 2 \sa (18-19 HCP), il rispondente può giocare 3 \c (Stayman avanzata). L'apertore risponde come segue.
\begin{twocol}
 3 \d & Entrambe le quarte nobili.\\
 3 \h & \h quarte, ma non \s quarte.\\
 3 \s & \s quarte, ma non \h quarte.\\
 3 \sa & Nessuna quarta nobile.
\end{twocol}



\pagebreak

\section{Verso lo slam}

\subsection{Cue-bid}

Non si gioca se il contratto sarà a SA, e comunque occorre avere già stabilito l'atout (esplicitamente o implicitamente). A partire da 3 \s e salendo gradualmente si dichiara un seme \j (che non sia l'atout) se si possiede una delle seguenti (Cue-bid debole): Asso di \j, K di \j, singolo o vuoto a \j.

Saltare un seme disponibile è come dichiarare che non si ha la Cue-bid in quel seme. Dichiarare SA è come dichiarare l'ultimo seme detto dal compagno, ma sono esclusi 4 SA e 5 SA (richiesta d'Assi e di Re). L'unica possibilità è quindi 3 \s\ - 3 SA, che dichiara Cue-bid (debole) a \s per entrambi i giocatori. Se un giocatore ripete un seme \j su cui ha già dato una Cue-bid, possiede una delle seguenti (Cue-bid forte): Asso di \j, singolo a \j.

Quando un giocatore nega la Cue-bid in un seme, il compagno dovrebbe immediatamente chiudere a manche se a sua volta non possiede la Cue-bid in quel seme.
Come corollario, continuare le Cue-bid dopo che il compagno ne ha negate alcune, equivale a dichiarare (implicitamente) le Cue-bid negate dal compagno.

Esempio (con atout \h): 3 \s\  (Cue-bid debole a \s) - 4 \d\ (nega la Cue-bid a \s e a \c) - 4 \s (Cue-bid forte a \s e debole a \c) - 5 \h\ (nega la Cue-bid forte a \d) - 6 \c\ (Cue-bid forte a \c) - \dots

Ovviamente talvolta ha senso interrompere le Cue-bid per effettuare una richiesta d'Assi.


\subsection{Richieste d'Assi e di Re (Roman Key Card Blackwood)}

\paragraph{Con atout.} Ai fini delle richieste d'Assi e di Re, il K di atout \`e considerato un Asso. In questo caso quindi ci sono 5 Assi e 3 Re. Seguono le risposte alla richiesta d'Assi (4 \sa).
\begin{twocol}
5 \c & 0 o 3 Assi.\\
5 \d & 1 o 4 Assi.\\
5 \h & 2 o 5 Assi, senza Q di atout.\\
5 \s & 2 o 5 Assi, con Q di atout.\\
\end{twocol}

\noindent Se si riceve una risposta che non dice nulla sulla Q di atout, si può licitare il primo colore disponibile che non sia l'atout per chiedere la Q di atout. Seguono le risposte.

\begin{itemize}
 \item La prima licita ad atout disponibile: niente Q di atout.
 \item La prima licita a \j disponibile (ma necessariamente inferiore a \note{6ATOUT}): Q di atout e K di \j.
 \item 5 \sa: Q di atout, e nessun K oppure nessun K dichiarabile secondo il punto precedente.
\end{itemize}

Se l'atout è \s, questa dichiarazione permette di dare il K di ogni seme, (nel caso in cui il rispondente possieda la Q di \s). Se l'atout è \h, è comunque sempre possibile sapere i K del rispondente mediante questa tecnica: con richiesta 5\s, la risposta 5 \sa potrebbe significare nessun K, ma anche K di \s.
Allora il richiedente domanda 6\c, e il rispondente licita 6\d se non possiede il K di \s, 6\h se possiede il K di \s.


\smallspace

\noindent In alternativa alla richiesta di Q di atout, con 5 \sa si può effettuare la richiesta di Re. Seguono le risposte.

\begin{twocol}
6 \c & Nessun Re.\\
6 \d & 1 Re.\\
6 \h & 2 Re.\\
6 \s & 3 Re. \\
\end{twocol}

\paragraph{Senza atout.} Seguono le risposte alla richiesta d'Assi (4 \sa).

\begin{twocol}
5 \c & 0 o 3 Assi.\\
5 \d & 1 o 4 Assi.\\
5 \h & 2 Assi.\\
\end{twocol}

\noindent Il richiedente può successivamente licitare 5 \s per costringere il rispondente a licitare 5 \sa (a giocare).

\smallspace

\noindent In alternativa si può effettuare la richiesta di Re, dichiarando 5 \sa. Seguono le risposte.

\begin{twocol}
6 \c & Nessun Re.\\
6 \d & 1 Re.\\
6 \h & 2 Re.\\
6 \s & 3 Re. \\
6 \sa & 4 Re.
\end{twocol}





\end{document}
