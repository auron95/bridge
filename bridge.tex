\documentclass[a4paper,10pt]{article}


\usepackage[utf8x]{inputenc}
\usepackage[italian]{babel}
\usepackage[pdftex]{graphicx}
\usepackage{amsfonts}
\usepackage{amsmath}
\usepackage{amssymb}
\usepackage{amsthm}
\usepackage{color}
\usepackage{enumerate}
\usepackage{fancyhdr}
\usepackage{hyperref}
\usepackage{xspace}
\usepackage[parfill]{parskip}
\usepackage[lmargin=4.5cm, marginparwidth=3.5cm, marginparsep=0.5cm]{geometry}
\usepackage{marginnote}
\usepackage{array}
\usepackage{skak}
\usepackage{needspace}

\topmargin -1cm
% \oddsidemargin -0.5cm
\textwidth 14.5cm

\reversemarginpar

\setlength{\parindent}{0 pt} % Default 15 pt.
\setlength{\parskip}{0.15 cm} % Default 0 cm?

\renewcommand{\c}{$\clubsuit$\xspace}
\renewcommand{\d}{$\diamondsuit$\xspace}
\newcommand{\h}{$\heartsuit$\xspace}
\newcommand{\s}{$\spadesuit$\xspace}
\renewcommand{\j}{$\bigstar$\xspace}
\newcommand{\sa}{SA\xspace}
\newcommand{\M}{M\xspace}
\newcommand{\m}{m\xspace}
%

\newcommand{\smallspace}{\vskip0.3cm}

\renewcommand{\tabcolsep}{0.3cm}

\newcommand{\note}[1]{\textcolor{red}{#1}}


\newenvironment{twocol}
  {\smallspace\noindent\begin{tabular}{l p{0.78\textwidth}}}
  {\end{tabular}\smallspace}

\newenvironment{twocolind}
  {\smallspace\noindent\begin{tabular}{l p{0.68\textwidth}}}
  {\end{tabular}\smallspace}

\newenvironment{threecol}
  {\smallspace\noindent\begin{tabular}{l l p{0.78\textwidth}}}
  {\end{tabular}\smallspace}


\newcommand{\biddingtable}[2][0.4cm]{
  \needspace{1cm}
  \marginnote{
    \scriptsize{
    \def\arraystretch{1.5}
    \renewcommand{\tabcolsep}{0.1cm}
    \begin{tabular}{|>{\centering\arraybackslash}p{0.6cm}>{\centering\arraybackslash}p{0.6cm}>{\centering\arraybackslash}p{0.6cm}>{\centering\arraybackslash}p{0.6cm}|}
      \hline
      #2
    \end{tabular}
    }
  }[#1]
}

\newcommand{\biddingtablesec}[1]{\biddingtable[-0.65cm]{#1}}
% \newcommand{\biddingtablepar}[1]{\biddingtable{#1}{0cm}}
  
% Title Page
\title{Convenzioni di bridge}
\author{Ugo Bindini \and Cristoforo Caffi \and Alice Cortinovis \and Giovanni Paolini \and Francesco Veneziano}

\begin{document}
\maketitle

\tableofcontents

\pagebreak
\section{Aperture}

\paragraph{Algoritmo generale.}
Si licita il colore più lungo, escludendo \s e \h se non sono almeno quinte e le \d se non sono almeno quarte.
In caso di parità di lunghezza dei semi più lunghi, si dichiara il più alto se sono almeno quinti e si dichiara il più basso se sono quarti.
Nota: con 4\s\ - 4\h\ - 3\d\ - 2\c, si apre di 1 \c.

\subsection{Mani fortissime}

Ogni mano con forza di manche o 22+ HCP bilanciata si apre convenzionalmente di 2 \c (forzante 1 giro).

\subsection{Mani bilanciate}
(4-3-3-3, 4-4-3-2, 5-3-3-2 con quinta minore)
\smallspace

\begin{threecol}
 12-14 & 1 \j & Si licita seguendo l'algoritmo generale. \\
 15-17 & 1 \sa & \\
 18-19 & 1 \c & Poi \sa a salto sulla risposta a livello di 1 (ma dare un fit nobile ha precedenza). \\
 20-21 & 2 \sa & Anche 5-3-3-2 con quinta nobile.
\end{threecol}


\subsection{Mani monocolore}
(5-3-3-2 con quinta nobile, 6-3-2-2, 6-3-3-1, 7-3-2-1, \dots)

\begin{threecol}
 12+ & 1 \j & \j è il colore lungo.
\end{threecol}


\subsection{Mani bicolore}
(5-4-3-1, 5-4-2-2, 6-4-2-1, 5-5-2-1, \dots)

\begin{threecol}
 12+ & 1 \j & \j è il colore più lungo; in caso di parità (5-5 o 6-6) si licita il colore più alto (in accordo con l'algoritmo generale).
\end{threecol}


\subsection{Mani tricolore}
(4-4-4-1, 5-4-4-0)

\begin{threecol}
 12+ & 1 \j & Si licita seguendo l'algoritmo generale. Unica eccezione: una mano 4-4-4-1 con singolo a picche (12-15 HCP) si apre di 1 \d.
\end{threecol}


\subsection{Sottoaperture}

\begin{twocol}
 2 \j & 6-10 HCP, con \j colore almeno sesto diverso da \c.\\
 3 \j & Colore almeno settimo con 9-11 HCP (meglio se a compagno passato), seme robusto.\\
 3 \sa & Minore almeno settimo chiuso (AKQ).\\
 4 \j & Colore almeno ottavo con 9-11 HCP (meglio se a compagno passato), seme robusto.
\end{twocol}

\pagebreak

\section{Risposte sull'apertura di 1 a colore}

Si risponde con 5+ HCP oppure con un Asso.

\subsection{Con fit su un colore maggiore \M}

% Per valutare la forza della mano si calcolano i punti totali (HD), dati dalla somma degli HCP e i punti distribuzione, così intesi.
% \begin{itemize}
%  \item Per ogni atout oltre l'ottava: 1 punto;
%  \item per ogni doubleton: 1 punto;
%  \item per ogni singolo: 2 punti se si hanno tre atout, 3 punti se si hanno almeno quattro atout;
%  \item per ogni vuoto: 3 punti se si hanno tre atout, 5 punti se si hanno almeno quattro atout   
% \end{itemize}

\biddingtable{1 \M & P & * & }
\begin{twocol}
  2 \c & Fit almeno terzo, 10-11 HCP, rispondente passato di mano (\hyperref[drury]{Drury}).\\ 
  2 \M  & Fit terzo, mano non invitante.\\ 
  2 \sa & Fit almeno terzo (\hyperref[jensen]{Jensen}), mano almeno invitante.\\
  3 \M  & Fit almeno quarto (barrage). \\
  3 \sa & Splinter nell'altro nobile.\\
  4 \c/\d & Splinter a \c/\d.\\
  4 \M  & Fit almeno quinto (barrage).\\
  Cue-bid & Fit almeno terzo, forzante manche. Le Cue-bid partono da 3 \h/\s.
\end{twocol}

\paragraph{Sviluppi dopo 1 \M\ - 2 \c (Drury).} \label{drury} \hspace{1cm}

\biddingtable{& & P & P \\ 1 \M & P & 2 \c & P \\ * &&&}
\begin{twocol}
  2 \d & 12+ HCP, forzante fino a 2 \M.\\
  2 \M & Apertura leggera, sign-off.\\
  3 \sa & 5-3-3-2 passa o correggi, senza visuale di slam.\\
  4 \M & A giocare.\\
  Altro & Naturale, forzante manche.
\end{twocol}

\paragraph{Sviluppi dopo 1 \M\ - 2 \M.}
L'apertore passa senza rever. Con rever, può licitare come segue.

\biddingtable{1 \M & P & 2 \M & P \\ * &&&}
\begin{twocol}
  2 \sa & Interrogativo.\\
  Nuovo seme a livello & \emph{Trial-bid}, seme debole almeno terzo in cerca di supporto (singolo/vuoto od onori).\\
  3 \M & Invitante generico.\\
  3 \sa & 18-19 HCP, mano semi-bilanciata, ``passa o correggi a 4 \M''.\\
  4 \M & A giocare.\\
\end{twocol}

Dopo 1 \M\ - 2 \M\ - 2 \sa, il rispondente licita come segue.

\biddingtable{1 \M & P & 2 \M & P \\ 2 \sa & P & * &}
\begin{twocol}
  3 \M & Mano minima.\\
  Altro & Mano massima non bilanciata, il colore più economico nel quale possiede dei valori (senza superare 3 \M).\\
  3 \sa & Mano massima bilanciata (preferisce 3 \sa a 4 \M).\\
  4 \M & Mano massima (preferisce 4 \M a 3 \sa).\\
\end{twocol}

Dopo una \emph{trial-bid}, il rispondente rivaluta la mano e sceglie fra 3 \M e 4 \M. 


\paragraph{Sviluppo dopo 1 \M - 2 \sa (Jensen).} \label{jensen} Dopo 1 \s\ - 2 \sa, l'apertore licita come segue.

\biddingtable{1 \s & P & 2 \sa & P \\ * &&&}
\begin{twocol}
  3 \c & 12-15 HCP.\\
  3 \d & 16+ HCP, singolo o vuoto a \d, senza quinte laterali.\\
  3 \h & 16+ HCP, singolo o vuoto a \h, senza quinte laterali.\\
  3 \s & 16+ HCP singolo o vuoto a \c, senza quinte laterali.\\
  3 \sa & 18+ HCP, senza quinte laterali né singoli né vuoti.\\
  4 \c & 16+ HCP, \c almeno quinte.\\
  4 \d & 16+ HCP, \d almeno quinte.\\
  4 \h & 16+ HCP, \h almeno quinte.\\
  4 \s & 16-17 HCP, senza singoli né vuoti.
\end{twocol}

Dopo 1 \s\ - 2 \sa\ - 3 \c, il rispondente licita come segue.

\biddingtable{1 \M & P & 2 \sa & P \\ 3 \c & P & * &}
\begin{twocol}
  3 \d & Forzante manche, richiesta di singoli o vuoti.\\
  3 \s & Invitante a 4 \s.\\
  Altro & Cue-bid.
\end{twocol}

Dopo 1 \s\ - 2 \sa\ - 3 \c\ - 3 \d, l'apertore licita come segue.

\biddingtable{1 \M & P & 2 \sa & P \\ 3 \c & P & 3 \d & P  \\ * &&&}
\begin{twocol}
  3 \s & Senza singoli né vuoti.\\
  3 \sa & Senza singoli né vuoti, disponibile a giocare \sa.\\
  Altro & Singolo o vuoto (il più economico).
\end{twocol}

Se l'apertore aveva aperto di 1 \h, le sequenze sono analoghe (è esclusa la risposta di 4 \s al terzo giro).

\subsection{Dopo l'apertura 1 \c}

Se possibile, dire un colore almeno quarto a livello di 1 -- il più lungo, e in caso di parità il più economico. Attenzione al caso di una mano bicolore \s/\h (vedi oltre). Altrimenti, si licita come segue.

\biddingtable{1 \c & P & * &}
\begin{twocol}
  1 \d/\h/\s & \d/\h/\s almeno quarte.\\
  1 \sa & 5-9 HCP, senza quarte licitabili a livello di 1 (quindi \c almeno quarte). \\
  2 \c & Forzante manche, senza quarte licitabili a livello di 1 (quindi \c almeno quarte). \\
  2 \d & 0-4 HCP e un seme nobile almeno sesto. L'apertore può licitare 2 \h (passa o correggi). \\
  2 \h & 5-8 HCP, bicolore \s/\h (almeno 5-4). \\
  2 \s & 9-11 HCP, bicolore \s/\h (almeno 5-4). \\
  2 \sa & 10-11 HCP, \c quarte o quinte (senza singoli né vuoti).\\
  3 \c & 10-11 HCP, \c almeno seste.
\end{twocol}


\subsection{Dopo l'apertura 1 \d}

Se possibile, dire un colore almeno quarto a livello di 1 -- il più lungo, e in caso di parità il più economico. Attenzione al caso di una mano bicolore \s/\h (vedi oltre). Altrimenti, si licita come segue.

\biddingtable{1 \d & P & * &}
\begin{twocol}
  1 \M & \M almeno quarto.\\
  1 \sa & 5-9 HCP, senza quarte licitabili a livello di 1. \\
  2 \c & Forzante manche, senza quarte licitabili a livello di 1. \\
  2 \d & 0-4 HCP e un seme nobile almeno sesto. L'apertore può licitare 2 \h (passa o correggi). \\
  2 \h & 5-8 HCP, bicolore \s/\h (almeno 5-4). \\
  2 \s & 9-11 HCP, bicolore \s/\h (almeno 5-4). \\
  2 \sa & 10-11 HCP, invitante a 3 \sa, senza singoli né vuoti.\\
  3 \d & 10-11 HCP, \d almeno quinte.
\end{twocol}


\subsection{Dopo l'apertura 1 \h (senza fit)}

Se possibile, dare la quarta di \s.
Le risposte 2/1 da parte di un rispondente precedentemente passato di mano sono solo invitanti; in questo caso la risposta di 2\c è artificiale (\hyperref[drury]{Drury}) e dà il fit.

\biddingtable{1 \h & P & * &}
\begin{twocol}
  1 \s & \s almeno quarte.\\
  1 \sa & 5-11 HCP. \\
  2 \c & Forzante manche. \\
  2 \d & Forzante manche, \d almeno quinte.\\
  2 \s & 0-5 HCP, \s almeno seste.\\
  3 \m & 9-11 HCP, \m bello almeno sesto, invitante a 3 \sa o 5 \m.\\
  3 \s & 5-7 HCP, \s belle almeno settime.\\
  4 \s & Barrage.
  
\end{twocol}


\subsection{Dopo l'apertura di 1 \s (senza fit)}

Le risposte 2/1 da parte di un rispondente precedentemente passato di mano sono solo invitanti; in questo caso la risposta di 2\c è artificiale (\hyperref[drury]{Drury}) e dà il fit.

\biddingtable{1 \s & P & * &}
\begin{twocol}
  1 \sa & 5-11 HCP. \\
  2 \c & Forzante manche. \\
  2 \d & Forzante manche, \d almeno quinte.\\
  2 \h & Forzante manche, \h almeno quinte.\\
  3 \c/\d/\h & 9-11 HCP, \c/\d/\h belle almeno seste, invitante a 3 \sa o 4 \h o 5 \c/\d.\\
  4 \h & Barrage.
\end{twocol}

\pagebreak

\section{Sviluppi dopo una risposta 1/1}

\subsection{Dopo un'apertura minore}

La seconda dichiarazione dell'apertore descrive la distribuzione della mano secondo principi naturali.

\paragraph{Dopo 1 \c\ - 1 \j.} In ordine di priorità:

\biddingtable{1 \c & P & 1 \j & P \\ * &&&}
\begin{enumerate}[(i)]
  \item dare il fit in un seme nobile;
  \item licitare una quarta nobile;
  \item dare il fit a \d;
  \item decidere tra 2 \c (con \c almeno quinte) o 1 \sa. In caso di mano semi-bilanciata con \j terzo, licitare 1 \sa.
\end{enumerate}

\paragraph{Dopo 1 \d\ - 1 \h.} In ordine di priorità:

\biddingtable{1 \d & P & 1 \h & P \\ * &&&}
\begin{twocol}
 2 \h & fit quarto;\\
 1 \s & \s almeno quarte;\\
 2 \c & \c almeno quarte;\\
 2 \d & \d almeno seste;\\
 1 \sa & mano bilanciata.
\end{twocol}

\paragraph{Dopo 1 \d\ - 1 \s.} In ordine di priorità:

\biddingtable{1 \d & P & 1 \s & P \\ * &&&}
\begin{twocol}
 2 \s & fit quarto;\\
 2 \c & \c almeno quarte;\\
 2 \d & \d almeno quinte;\\
 1 \sa & mano bilanciata.
\end{twocol}

\subsection{Dopo un'apertura nobile}

\paragraph{Dopo 1 \h\ - 1 \s.} In ordine di priorità:

\biddingtable{1 \h & P & 1 \s & P \\ * &&&}
\begin{twocol}
 2 \sa & 16+ HCP, \h almeno seste, una quarta laterale qualsiasi (anche \s);\\
 2 \c & 12-15 HCP e \c almeno quarte, oppure 16+ HCP (\hyperref[gazzilli]{Gazzilli});\\
 3 \s & invitante, fit quarto, mano sbilanciata e buone \s;\\
 2 \s & 12-15 HCP, fit quarto;\\
 3 \m & invitante, \m buon seme almeno quinto;\\
 3 \h & invitante, \h belle almeno seste;\\
 2 \d & 12-15 HCP, \d almeno quarte;\\
 2 \h & 12-15 HCP, \h almeno seste;\\
 1 \sa & mano semi-bilanciata;
\end{twocol}





\paragraph{Senza rever.} Licitando 2 nel colore del rispondente si dà il fit. Altrimenti 1 \sa oppure (se possibile) un nuovo colore.

Dopo 1 colore - 1 colore - 1 colore, la risposta di 1 \sa esclude la manche (5-8 HCP), mentre la risposta di 2 \sa è invitante manche a \sa.

\paragraph{Roudi.} Dopo la sequenza 1 colore - 1 \h (\s) \ - 1 \sa, la licita di 2 \c (Roudi) chiede ulteriore descrizione della mano. L'apertore licita come segue.
\begin{twocol}
	2 \d & 12-13 HCP, nega la terza di \h (\s).\\
	2 \h & 12-13 HCP, terza di \h (\s).\\
	2 \s & 14-15 HCP, terza di \h (\s).\\
	2 \sa & 14-15 HCP, nega la terza di \h (\s).
\end{twocol}


\paragraph{Con rever.} Dopo 1 \c\ - 1 \h si licita
\begin{twocol}
	2 \d & \d almeno quarte, senza fit.\\
	2 \s & \s almeno quarte, senza fit.\\
	2 \sa & 18-19 HCP, mano bilanciata \emph{senza fit}.\\
	3 \c & 16-18 HCP, \c seste, senza fit.\\
	3 \d & 19+ HCP, \c seste, senza fit.\\
	3 \h & Fit a \h, 16-18 HCP, invitante.\\
	3 \s\ / 4 \c / 4 \d & Cue-bid. 19+ HCP, fit a \h, forzante manche.\\
	3 \sa & 
\end{twocol}

\noindent Se il rispondente aveva licitato 1 \s, semplicemente si scambiano i ruoli di \h e \s.
\smallspace
\noindent Dopo 1 \d\ - 1 \h si licita
\begin{twocol}
	2 \s & \s quarte, senza fit.\\
	2 \sa & 16-18 HCP, senza fit, senza singoli né vuoti, \d al massimo seste (5-4-2-2, 6-3-2-2).\\
	3 \c & \c quarte (bicolore \d/\c), senza fit, con almeno un singolo o vuoto.\\
	3 \d & \d seste con almeno un singolo o vuoto, oppure \d settime, senza fit.\\
	3 \h & Fit a \h, 16-18 HCP, invitante.\\
	3 \s\ / 4 \c\ / 4 \d & Cue-bid. 19+ HCP, fit a \h, forzante manche.\\
	3 \sa & 19+ HCP, senza fit, senza singoli né vuoti, \d al massimo seste (5-4-2-2, 6-3-2-2).
\end{twocol}

\noindent Se il rispondente aveva licitato 1 \s, semplicemente si scambiano i ruoli di \h e \s.
\smallspace
\noindent Dopo 1 \h\ - 1 \s si licita 2 \c (Gazzilli).

\paragraph{Sviluppi dopo un rever a livello di 2 (a colore).} Il rispondente può licitare 2 \sa (Ingberman), obbligando l'apertore a licitare 3 \c con mano minima (16-17 HCP). Dopo questa licita di 3 \c, il rispondente può scegliere una atout a livello di 3 (eventualmente passando) e si gioca un contratto parziale. Ogni altra licita superiore o uguale a 3 \sa è forzante manche.

\subsection{Sviluppi dopo la risposta di 1 \sa}

\color{red}
\noindent Dopo la sequenza 1 \c\ - 1 \sa:
\begin{twocol}
	Pass & Mano minima con \c al massimo terze, a giocare. \\
	2 \c & Mano minima con \c almeno quarte, a giocare. \\
	2 \sa & 15-16 HCP, invitante manche a \sa.\\
	3 \sa & 17+ HCP, a giocare.\\
	Altro & Come da algoritmo generale.
\end{twocol}

\noindent Dopo la sequenza 1 \d\ - 1 \sa:
\begin{twocol}
	2 \c & Rever generico con 16-18 HCP.\\
	2 \sa & 14-15 HCP, invitante manche a \sa.\\
	Altro a livello di 2 & Come da algoritmo generale.\\
	Altro a livello di 3 & Come da algoritmo generale, ma con 19+ HCP.
\end{twocol}

\color{black}

\noindent Sull'apertura di 1 \h o 1 \s:
\begin{twocol}
  2 \c & Gazzilli. \\
  Altro & Come da algoritmo generale, nega la rever.
\end{twocol}


\subsection{2 \c Gazzilli} \label{gazzilli}

La 2 \c Gazzilli si applica in tre situazioni: 1 \h\ - 1 \s\ - 2 \c, 1 \h\ - 1 \sa\ - 2 \c, 1 \s\ - 1 \sa\ - 2 \c.
Il rispondente licita come segue.

\begin{twocol}
  2 \d & Relais positivo, 8+ HCP. \\
  Riporto nel nobile & 5-7 HCP, doubleton. \\
  Altro nobile & 5-7 HCP, nobile almeno quinto. \\
  3 \c\ / 3 \d & 5-7 HCP, minore almeno sesto. \\
  2 \sa & Tutti gli altri casi.
\end{twocol}

In aggiunta, sulla sequenza 1 \h\ - 1 \sa\ - 2 \c, il rispondente ha a disposizione l'ulteriore risposta di 2 \s per indicare una bicolore minore debole.

\paragraph{Sviluppi dopo il relay positivo di 2 \d.}
Sull'apertura di 1 \h, l'apertore licita come segue.

\begin{twocol}
  2 \h & 12-15 HCP, \c almeno quarte. \\
  2 \s & Esattamente 3 carte di \s, rever. (*)\\
  2 \sa & 5-3-3-2 senza terza di \s, 19+ HCP. \\
  3 \c\ / 3 \d & Minore almeno quarto, senza terza di \s, rever. \\
  3 \h & \h almeno seste, senza terza di \s, rever. \\
  3 \s & \s quarte, rever. \\
  3 \sa & 5-3-3-2 senza terza di \s, 16-18 HCP.
\end{twocol}

\noindent (*) Sviluppi dopo 2 \s. Il rispondente licita 2 \sa (relais) e l'apertore licita come segue.
\begin{twocol}
  3 \c / 3 \d & Minore almeno quarto. \\
  3 \h & \h almeno seste. \\
  3 \s & 5-3-3-2, 19+ HCP. \\
  3 \sa & 5-3-3-2, 16-18 HCP.
\end{twocol}


\noindent Sull'apertura di 1 \s, l'apertore licita come segue.

\begin{twocol}
  2 \h & \h terze o quarte, rever. (*)\\
  2 \s & 12-15 HCP, \c almeno quarte. \\
  2 \sa & 5-3-3-2 senza terza di \h, 19+ HCP. \\
  3 \c\ / 3 \d & Minore almeno quarto, senza terza di \h, rever. \\
  3 \h & \h quinte, rever. \\
  3 \s & \s almeno seste, senza terza di \h, rever. \\
  3 \sa & 5-3-3-2 senza terza di \h, 16-18 HCP.
\end{twocol}

\noindent (*) Sviluppi dopo 2 \h. Il rispondente licita 2 \s (relais) e l'apertore licita come segue.
\begin{twocol}
  2 \sa & 5-3-3-2, 19+ HCP. \\
  3 \c / 3 \d & Minore almeno quarto. \\
  3 \h & \h quarte. \\
  3 \s & \s almeno seste. \\
  3 \sa & 5-3-3-2, 16-18 HCP.
\end{twocol}


\section{Dopo una risposta a livello di 2}

\subsection{1 \c\ - 2 \c}

Dopo 1 \c\ - 2 \c, l'apertore licita come segue.

\begin{twocol}
	2 \d & 6+ \c e 5+ \d. \\
	2 \h & 6+ \c e 5+ \h. \\
	2 \s & 6+ \c e 5+ \s. \\
	2 SA & 18-19 HCP, mano bilanciata. \\
	3 \c & 14+ HCP, \c almeno quinte, senza singoli né vuoti. \\
	3 \d & 14+ HCP, \c almeno quinte, singolo o vuoto a \d.\\
	3 \h & 14+ HCP, \c almeno quinte, singolo o vuoto a \h.\\
	3 \s & 14+ HCP, \c almeno quinte, singolo o vuoto a \s.\\
	3 \sa & 12-14 HCP mano bilanciata, a giocare.\\
	4 \c & Gerber.
\end{twocol}

\paragraph{Dopo 2 \d.}

Seguono le risposte.

\begin{twocol}
	2 \sa & 11-12 HCP, \c al massimo quinte. \\
	3 \c & 11-12 HCP, \c almeno seste. \\
	3 \d/\c/\s & 13+ HCP, \c almeno seste, richiesta di fermo. \\
	3 \sa & A giocare. \\
	4 \c & Gerber.
\end{twocol}

Dopo 3 \d/\h/\s, l'apertore licita 3 \sa con il fermo, 4 \c senza fermo e con mano minima, 5 \c senza fermo e con mano massima. Il rispondente può correggere a 5 \c con almeno 14-15 HCP.

\paragraph{Dopo 2 \h o 2 \s.}

Dopo 2 \h, il rispondente licita come segue.

\begin{twocol}
	2 \sa & 11-12 HCP, fermo a \h. \\
	3 \c & 11-12 HCP, nega il fermo a \h. \\
	3 \d & 13-14 HCP, nega il fermo a \h e la terza di \s, invitante manche a \c. \\
	3 \h & 13-14 HCP, nega il fermo a \h, terza di \s, invitante manche a \c (o dire 4 \s con la 6-5). \\
	3 \s & 13-14 HCP, fermo a \h, terza di \s. \\
	3 \sa & 13-14 HCP, fermo a \h, nega la terza di \s. \\
	4 \c & Gerber. \\
	4 \d/\h/\s & Cue-bid, interesse di slam a \c. \\
	4 \sa & Richiesta d'Assi (con atout \c). \\
	5 \c & A giocare. \\
\end{twocol}

Dopo 2 \s, le risposte sono le stesse, invertendo però \h e \s.

Nota: con mano minima l'apertore potrebbe dover chiedere il fermo in un colore in cui lo ha già, per cui può licitare 3 \sa (da passare) anche dopo una risposta negativa.

\paragraph{Dopo 2 \sa.} La manche è assicurata, si può cercare lo slam a \c o a \sa.

\begin{twocol}
	3 \c & Naturale, quinta di \c. \\
	3 \d/\h/\s & Cue-bid, sesta di \c. \\
	3 \sa & A giocare, con mano minima bilanciata (e poco coraggio). \\
	4 \c & Gerber.
\end{twocol}

Dopo 3 \c l'apertore inizia le cue-bid con il fit, o licita 3 \sa altrimenti. %Dopo la risposta su 4 \c, 5 \sa è un ``Pick A Small Slam'' che invita a licitare 6 \c con la terza di \c, 6 \d con la quinta di \d (e senza terza di \c, perché il rispondente dopo \c può licitare 6 \d senza la quinta di \c ma con la terza di \d) e 6 \sa altrimenti.

\paragraph{Dopo 3 \d.} Seguono le risposte.
%TODO: No, questa cosa non funziona.

\begin{twocol}
	3 \sa & Fermo a \d, a giocare. \\
	4 \c & Nega il fermo a \d, mano minima, invitante manche a \c. \\
	4 \d/\h/\s & Cue-bid con interesse di slam a \c. \\
	4 \sa & Richiesta d'Assi (con atout \c). \\
	5 \c & Nega il fermo a \d, a giocare.
\end{twocol}

\paragraph{Dopo 3 \h o 3 \s.} Senza interesse di slam (con mano minima) si licita 3 \sa senza il fit nobile, o 4 \h/\s con il fit. Con interesse di slam (che sarà a \c, visto il fit almeno decimo) si licita 4 \c, che fissa l'atout ed invita ad iniziare le cue-bid.

\paragraph{Dopo 3 \sa.} Si può passare o chiedere gli Assi con 4 \c. Con mano forte e sbilanciata si può anche licitare una cue-bid, a partire da 4 \d (forzante manche e con interesse di slam a \c).

\noindent Su 1 \h\ - 2 \sa semplicemente si scambiano \h e \s nelle risposte precedenti, escludendo però 4 \s: in caso di singolo o vuoto a \s con 16+ HCP si licita 3 \sa.

Dopo la risposta su 2 \sa, il rispondente può: chiudere a 3 \j o 4 \j (a giocare), iniziare le Cue-bid (direttamente dalla licita più economica disponibile, diversa da \j e da 4 \sa) oppure chiedere gli assi licitando 4 \sa. La licita di 3 \sa in questo caso conta come Cue-bid nell'ultimo seme licitato dall'apertore, che nega le Cue-bid saltate (es: 1 \s\ - \mbox{2 \sa}\ - 3 \c\ - 3 \sa dà Cue-bid a \c e la nega a \d e \h).

Attenzione: le risposte che garantiscono un singolo o un vuoto danno implicitamente una Cue-bid debole nel colore (questo è importante nel caso in cui, successivamente, si inizino le Cue-bid).

\pagebreak

\section{Risposte sull'apertura 1 SA}

\subsection{Mano sbilanciata (Transfer)}

Almeno sei carte in un colore (o anche cinque con singoli o vuoti con senno, o con pochi punti).

\begin{twocol}
 2 \d & Transfer per le \h.\\
 2 \h & Transfer per le \s.\\
 2 \s & Transfer per le \c.\\
 3 \c & Transfer per le \d.\\
\end{twocol}

L'apertore deve licitare il seme richiesto dal rispondente (che diventa automaticamente l'atout concordata), dopodiché il rispondente può proseguire passando, licitando manche o slam (a giocare), iniziando le Cue-bid, oppure chiedendo gli Assi (4 \sa).


\subsection{Mano bilanciata, senza quarte nobili e senza ambizioni di slam}

\begin{twocol}
 2 \sa & 7-8 HCP, invitante manche.\\
 3 \sa & 9-12 HCP, a giocare.
\end{twocol}


\subsection{1 SA - 2 \c (Stayman)}

Richiesta di quarte nobili, 8+ HCP (oppure 7 con senno). Seguono le risposte.

\begin{twocol}
	2 \d & Nessuna quarta nobile. \\
	2 \h & \h quarte, ma non \s quarte. \\
	2 \s & \s quarte, ma non \h quarte. \\
	2 \sa & Entrambe le quarte nobili. \\
	3 \c & 5-3-3-2 con la quinta di \c, mano massima. \\
	3 \d & 5-3-3-2 con la quinta di \d, mano massima. \\
\end{twocol}

\noindent Esaminiamo gli sviluppi della Stayman.

\paragraph{Fit in un colore nobile.}

Con il fit su un nobile \j, dopo la risposta di 2 \j.
\begin{twocol}
	3 \j & Seleziona \j come atout. Invitante manche. \\
	Cue-bid & A partire da 3 \c. Sottintende il nobile dell'apertore come atout. \\
	4 \j & A giocare. \\
\end{twocol}

Con il fit su un nobile dopo la risposta di 2 \sa.

\begin{twocol}
	3 \c & Interesse di slam. \\
	3 \d & Transfer per le \h, senza interesse di slam. \\
	3 \h & Transfer per le \s, senza interesse di slam. \\
\end{twocol}

Dopo 3 \c, l'apertore è obbligato a licitare 3 \d, il rispondente fissa l'atout con 3 \h/\s, quindi iniziano le cue-bid.
Dopo 3 \d/\h, l'apertore licita 3 \h/\s con mano minima, e 4 \h/\s con mano massima (a giocare). Con 9+ HCP il rispondente corregge a 4 \h/\s.

\paragraph{Richiesta dell'altro nobile.}
Sulla risposta di 2 \h, se il rispondente ha cinque carte di \s ma non quattro carte di \h, licita 2 \s. L'apertore licita come segue.

\begin{twocol}
	2 \sa & Nega il fit a \s, mano minima. \\
	3 \c/\d/\h & Cue-bid, fit a \s, mano massima. \\
	3 \s & Fit a \s, mano minima. \\
	3 \sa & Nega il fit a \s, mano massima.
\end{twocol}

Dopo 1 \sa\ - 2 \c\ - 2 \s, se il rispondente ha cinque carte di \h ma non quattro carte di \s, licita 2 \sa con mano minima, e 3 \h altrimenti (forzante manche).

Su 1 \sa\ - 2 \c\ - 2 \s\ - 2 \sa, l'apertore licita come segue.

\begin{twocol}
	Pass & Mano minima. \\
	3 \h & Mano massima, terza di \h. Invita a scegliere fra 3 \sa e 4 \h. \\
	3 \sa & Mano massima, nega la terza di \h. \\
\end{twocol}

Dopo la licita di 3 \h, il rispondente corregge a 3 \sa senza la quinta di \h, e a 4 \h con la quinta di \h.

\paragraph{Gioco a SA.}
Per giocare a SA il rispondente licita:
\begin{twocol}
	2 \sa & 7-8 HCP, invitante manche. \\
	3 \sa & A giocare. \\
	4 \c & Richiesta d'Assi (Gerber).
\end{twocol}

\paragraph{Fit miracoloso nel minore quinto.}
Dopo la risposta di 3 \c/\d, il rispondente può confermare tale atout licitando una Cue-bid (a partire da 3 \d\ / 4 \h, con senno).


\paragraph{Ricerca del fit in un minore.}
Dopo le risposte di 2 \d, 2 \h o 2 \s, il rispondente può cercare una quarta minore licitando il minore. L'apertore dà la quarta licitando una Cue-bid e la nega licitando 3 \sa.
\note{Ma su 2 \h e 2 \s, 3 \c e 3 \d non sono cue-bid? Secondo me ha senso che non lo siano.}

\note{Questa sezione ha dei bug.}

\pagebreak

\section{Sviluppi sulle aperture forti}


\subsection{Aperture forti bilanciate}

\paragraph{Transfer.} Sull'apertura 2 \sa, con mano sbilanciata (seme almeno sesto) il rispondente può licitare una Transfer (anche 4 \c, che in questo caso non è una Gerber).

\paragraph{Stayman avanzata.} Su 2 \sa (20-22 HCP), oppure 1 \c\ - 1 \j\ - 2 \sa (18-19 HCP), il rispondente può giocare 3 \c (Stayman avanzata). L'apertore risponde come segue.
\begin{twocol}
	3 \d & Nessuna quarta nobile.\\
	3 \h & \h quarte, ma non \s quarte.\\
	3 \s & \s quarte, ma non \h quarte.\\
	3 \sa & Entrambe le quarte nobili.\\
\end{twocol}

\noindent Distinguiamo ora i vari casi.
\begin{itemize}
  \item Risposta di 3 \sa.
  \begin{twocol}
    Pass & \\
    4 \c & Gerber (per giocare a \sa).\\
    4 \d & Obbliga a dichiarare 4 \h (per giocare a \h).\\
    4 \h & Obbliga a dichiarare 4 \s (per giocare a \s).\\
  \end{twocol}
  
  \item Risposta di 3 \h/\s.
  \begin{twocol}
    3 \sa & A giocare.\\
    4 \c & Gerber.\\
%     4 \d & Ho una quinta nell'altro maggiore \j! L'apertore risponde 4 \j se ha la terza, altrimenti risponde licitando l'altro maggiore. In quest'ultimo caso, 4 \sa è a giocare e 5 \c è una 5-Gerber.\\
    4 \h/\s & A giocare.\\
    4 \sa & RKCB (sottintendendo il fit).
  \end{twocol}
  
  \item Risposta di 3 \d.
  \begin{twocol}
    3 \h & Quinta di \h. L'apertore risponde 3 \sa se non ha il fit, altrimenti con la prima Cue-bid.\\
    3 \s & Quinta di \s. L'apertore risponde 3 \sa se non ha il fit, altrimenti con la prima Cue-bid.\\
    3 \sa & A giocare.\\
    4 \c & Gerber.\\
    4 \d & Entrambe le quinte maggiori. L'apertore risponde con il suo maggiore più lungo (sarà un fit).\\
  \end{twocol}
\end{itemize}



\subsection{Apertura di 2 \c}

Ricordiamo che l'apertura di 2 \c si effettua con 23+ HCP e mano bilanciata, oppure con una mano (anche sbilanciata) con cui si può giocare la manche anche se il compagno ha delle pessime carte.
È l'unica apertura sulla quale il compagno è obbligato a rispondere, anche con 0 HCP.
Come filosofia generale, è il compagno dell'apertore a condurre la dichiarazione.
In ogni caso, dopo l'apertura di 2 \c ogni licita sotto la manche è automaticamente forzante (con unica eccezione 2 \c\ - 2 \d\ - 2 \sa, vedi oltre).

Seguono le risposte.
\begin{twocol}
  2 \d & 0-6 HCP.\\
  2 \h/\s & 7+ HCP, nobile almeno quarto.\\
  2 \sa & 7+ HCP, mano bilanciata (nobili al più terzi, minori al più quarti).\\
  3 \c/\d & 7+ HCP, minore almeno quinto, nessun nobile quarto.\\
\end{twocol}

Sulla risposta di 2 \d, se l'apertore ha mano bilanciata licita 2 \sa (passabile) nel caso in cui non può garantire la manche a \sa, altrimenti licita 3 \sa.

Sulle risposte di 2 \d/\h/\s, se l'apertore ha mano bilanciata licita 2 \sa (oppure 3 \sa nel caso descritto sopra); il rispondente può allora licitare 3 \c (Stayman avanzata).

In tutti gli altri casi si prosegue in modo naturale.


\pagebreak

\section{Sviluppi sulle sottoaperture}

Sulle sottoaperture le Cue-bid si iniziano solo a salto. Dichiarare 4 \sa sottintende il seme dell'apertore come atout. Ogni altra dichiarazione a colore è naturale (con seme almeno quinto e con 14+ HCP) e forzante 1 giro.

\paragraph{Su 2 \j.} Se il rispondente ha almeno 14 HCP può dichiarare 2 \sa. Seguono le risposte.
\begin{twocol}
 3 \c & Mano minima, carte deboli a \j.\\
 3 \d & Mano minima, carte buone a \j.\\
 3 \h & Mano massima, carte deboli a \j.\\
 3 \s & Mano massima, carte buone a \j.\\
 3 \sa & A, K, Q di \j.
\end{twocol}

\noindent Se dopo 2 \sa c'è un intervento, Pass dà mano minima, X o XX mano massima.



\pagebreak

\section{Verso lo slam}

\subsection{Cue-bid}

Non si gioca se il contratto sarà a SA, e comunque occorre avere già stabilito l'atout (esplicitamente o implicitamente). A partire da 3 \s e salendo gradualmente si dichiara un seme \j (che non sia l'atout) se si possiede una delle seguenti (Cue-bid debole): Asso di \j, K di \j, singolo o vuoto a \j.

Saltare un seme disponibile è come dichiarare che non si ha la Cue-bid in quel seme. Dichiarare SA è come dichiarare l'ultimo seme detto dal compagno, ma sono esclusi 4 SA e 5 SA (richiesta d'Assi e di Re). L'unica possibilità è quindi 3 \s\ - 3 SA, che dichiara Cue-bid (debole) a \s per entrambi i giocatori. Se un giocatore ripete un seme \j su cui ha già dato una Cue-bid, possiede una delle seguenti (Cue-bid forte): Asso di \j, vuoto a \j.

Quando un giocatore nega la Cue-bid in un seme, il compagno dovrebbe immediatamente chiudere a manche se a sua volta non possiede la Cue-bid in quel seme.
Come corollario, continuare le Cue-bid dopo che il compagno ne ha negate alcune, equivale a dichiarare (implicitamente) le Cue-bid negate dal compagno.

Esempio (con atout \h): 3 \s\  (Cue-bid debole a \s) - 4 \d\ (nega la Cue-bid a \s e a \c) - 4 \s (Cue-bid forte a \s e debole a \c) - 5 \h\ (nega la Cue-bid forte a \d) - 6 \c\ (Cue-bid forte a \c) - \dots \note{Questo esempio ha un bug!}

\paragraph{Cue-bid avanzate} Se \j è un seme diverso dall'atout, un giocatore A si dice \j-completo se ha dato Cue-bid forte a \j oppure ha già negato una Cue-bid a \j. Se A è \j-completo, \j diventa Cue-bid nel colore in cui ha dato meno informazioni (in caso di parità, il più basso in rango). Come corollario, se A salta \j sta negando una Cue-bid.
\vspace{4mm}

Ovviamente talvolta ha senso interrompere le Cue-bid per effettuare una richiesta d'Assi.


\subsection{Richieste d'Assi e di Re a colore (Roman Key Card Blackwood)}

Ai fini delle richieste d'Assi e di Re, il K di atout è considerato un Asso. In questo caso quindi ci sono 5 Assi e 3 Re. Seguono le risposte alla richiesta d'Assi (4 \sa) nel caso in cui l'atout sia un colore nobile.
\begin{twocol}
5 \c & 1 o 4 Assi.\\
5 \d & 0 o 3 Assi.\\
5 \h & 2 o 5 Assi, senza Q di atout.\\
5 \s & 2 o 5 Assi, con Q di atout.\\
\end{twocol}
\noindent Seguono le risposte nel caso in cui l'atout sia un colore minore.
\begin{twocol}
5 \c & 0 o 3 Assi.\\
5 \d & 1 o 4 Assi.\\
5 \h & 2 o 5 Assi, senza Q di atout.\\
5 \s & 2 o 5 Assi, con Q di atout.\\
\end{twocol}

\noindent Se si riceve una risposta che non dice nulla sulla Q di atout, si può licitare il primo colore disponibile che non sia l'atout per chiedere la Q di atout. Seguono le risposte.

\begin{itemize}
 \item La prima licita ad atout disponibile: niente Q di atout.
 \item La prima licita a \j disponibile (ma necessariamente inferiore a 6 nel colore di atout): Q di atout e K di \j.
 \item 5 \sa: Q di atout, e nessun K oppure nessun K dichiarabile secondo il punto precedente.
\end{itemize}

Se l'atout è \s, questa dichiarazione permette di dare il K di ogni seme, (nel caso in cui il rispondente possieda la Q di \s). Se l'atout è \h, è comunque sempre possibile sapere i K del rispondente mediante questa tecnica: con richiesta 5 \s, la risposta 5 \sa potrebbe significare nessun K, ma anche K di \s.
Allora il richiedente domanda 6 \c, e il rispondente licita 6 \d se non possiede il K di \s, 6 \h se possiede il K di \s.


\smallspace

\noindent In alternativa alla richiesta di Q di atout, con 5 \sa si può effettuare la richiesta di Re. Seguono le risposte.

\begin{twocol}
6 \c & Nessun Re.\\
6 \d & 1 Re.\\
6 \h & 2 Re.\\
6 \s & 3 Re. \\
\end{twocol}


\subsection{Richieste d'Assi e di Re a \sa (Gerber)}

La richiesta d'Assi si effettua licitando 4 \c, dopo che è stato concordato di giocare a \sa. Seguono le risposte.
\begin{twocol}
  4 \d & 0 o 4 Assi.\\
  4 \h & 1 Asso.\\
  4 \s & 2 Assi.\\
  4 \sa & 3 Assi.
\end{twocol}

\noindent Dopo la richiesta d'Assi, si può effettuare la richiesta di Re licitando 5 \c. Seguono le risposte.
\begin{twocol}
  5 \d & 0 o 4 Re.\\
  5 \h & 1 Re.\\
  5 \s & 2 Re.\\
  5 \sa & 3 Re.
\end{twocol}


\subsection{5 \sa ``Pick A Small Slam''}

In una situazione in cui non vuole già dire qualcos'altro (richiesta di Re, a giocare dopo 5 \c, Cue-bid, ...), la licita di 5 \sa invita a dichiarare in modo naturale proponendo un piccolo slam. Si prosegue la licita in modo naturale.



\pagebreak
\section{Interventi}

Dopo un'apertura a livello di 1 degli avversari, sono previsti i seguenti interventi.

\begin{twocol}
  Dbl & 12+ HCP, (indicativamente) al massimo 2 carte nei colori licitati dagli avversari, almeno 3 negli altri; se gli avversari hanno licitato un nobile, quattro carte nell'altro. \\
  & In alternativa, 16+ HCP e distribuzione qualsiasi.\\
  1 \j & 8+ HCP, \j almeno quinto.\\
  1 \sa & Come apertura, con fermo nel palo degli avversari.\\
  2 \j a livello & 11+ HCP, \j almeno quinto.\\
  2 \j a salto & Come apertura.\\
  3 \j, 4 \j & Come apertura.
\end{twocol}

\subsection{Mani bicolore}

I seguenti interventi si applicano con una mano bicolore (almeno 5-5) e 12+ HCP. Tenere in considerazione le vulnerabilità.

Sull'apertura di 1 \c degli avversari si licita come segue.
\begin{twocol}
  2 \c & Naturale: 11+ HCP, \c almeno quinti.\\
  2 \sa & Bicolore \d e un seme nobile. Il compagno può licitare 3 \c (richiesta del seme nobile), 3 \d (a giocare).\\
  3 \c & Bicolore \h - \s. Il compagno può licitare 3 \h/\s (a giocare), ogni altra licita è forzante manche.
\end{twocol}

Sull'apertura di 1 \d degli avversari si licita come segue.
\begin{twocol}
  2 \d & Bicolore \h - \s.\\
  2 \sa & Bicolore \c e un seme nobile. Il compagno può licitare 3 \c (a giocare), 3 \d (richiesta del nobile).
\end{twocol}

Sull'apertura di 1 \h (\s) degli avversari si licita come segue.
\begin{twocol}
  2 \h (\s) & Bicolore \s (\h) e minore. Il compagno può licitare 2 \sa (richiesta del minore).\\
  2 \sa & Bicolore minore.
\end{twocol}


\subsection{Sviluppi dopo il contre informativo}

\paragraph{A livello di 1, se il RHO passa.} Dopo 1 \j\ - Dbl - Pass, il compagno del contrante licita come segue.

\begin{twocol}
	Pass & Mano forte sbilanciata, buone carte di \j, nessun colore licitabile. \\
	Colore a livello & 0-8 HCP, colore quarto (se possibile, altrimenti il colore più lungo). \\
	Colore a salto & 9+ HCP, colore quarto, forzante. \\
	Colore a doppio salto & 12+ HCP, colore quinto, forzante. Mai a livello di 4. \\
	1 \sa & 8-11 HCP, mano bilanciata, fermo nel colore dell'avversario. \\
	2 \j & 10+ HCP, mano tendenzialmente sbilanciata, senza licite migliori. \\
	2 \sa & 12+ HCP, mano bilanciata, fermo nel colore avversario. \\
\end{twocol}

\paragraph{A livello di 1, se il RHO surcontra.} Dopo 1 \j\ - Dbl - Rbl, il compagno del contrante licita come segue.

\begin{twocol}
	Pass & 0-7 HCP. \\
	Colore a livello & 8-11 HCP, colore quarto. \\
	Colore a salto & 0-7 HCP, colore quinto (barrage), non a sfavore di zona. \\
	2 \j & 12+ HCP, qualsiasi distribuzione, forzante. \\
\end{twocol}

\paragraph{A livello di 2. (Lebensohl)} Dopo 2 \j\ - Dbl - Pass, il compagno del contrante licita come segue.


\begin{twocol}
	Colore a livello di 2 & 0-7 HCP, colore quarto. \\
	2 \sa & Artificiale, forzante, vedi oltre. \\
	Colore a livello di 3, a livello & 8-11 HCP, colore quarto. \\
	Colore a livello di 3, a salto & 12+ HCP, colore quarto, forzante manche. \\
	3 \sa & 12+ HCP, mano bilanciata, senza fermo nel colore avversario.
\end{twocol}

Sulla risposta di 2 \sa, il contrante è obbligato a licitare 3 \c senza Rever, ma può licitare naturalmente un colore a livello di 3 con Rever. Se il contrante licita un nuovo colore, gli sviluppi sono naturali. Dopo 3 \c, invece, il compagno del contrante risponde come segue.

\begin{twocol}
	Pass & 0-7 HCP, almeno quattro carte di \c. \\
	Colore a livello di 3, rango minore di \j & 0-7 HCP, colore quarto. \\
	Colore a livello di 3, rango maggiore di \j & 8-11 HCP, colore quarto. \\
	3 \sa & 12+ HCP, mano bilanciata, fermo nel colore avversario.
\end{twocol}

\end{document}
