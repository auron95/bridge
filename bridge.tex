\documentclass[a4paper,10pt]{article}


\usepackage[utf8]{inputenc}
\usepackage[italian]{babel}
\usepackage[T1]{fontenc}
\usepackage[dvips]{graphicx}
\usepackage{amsmath}
\usepackage{amsthm}
\usepackage{fancyhdr}
\usepackage{amsfonts}
\usepackage{amssymb}


\topmargin -1cm
\oddsidemargin -0.5cm
%\evensidemargin	-1cm
\textwidth 17cm


% Title Page
\title{Appunti di bridge}
\author{Giove}

\begin{document}
\maketitle


\section{Aperture}

\subsection{Mani bilanciate}
(4-3-3-3, 4-4-3-2, 5-3-3-2)\\

\begin{tabular}{|p{0.05\textwidth} p{0.05\textwidth} p{0.7\textwidth}|}

 12-14 & 1 $\bigstar$ & $\bigstar$ \`e il colore pi\`u lungo; in caso di parit\`a si licita il colore pi\`u basso se quarto, quello pi\`u alto se almeno quinto. Con questo algoritmo si aprono anche mani non bilanciate (vedi sotto).\\

 15-17 & 1 SA & \\

 18-19 & 1 $\bigstar$ & 5-3-3-2, $\bigstar$ \`e il seme quinto.\\
       & 1 $\clubsuit$ & Tutte le altre distribuzioni.\\

 20-22 & 2 SA & \\

 23+ & 2 $\clubsuit$ & \\

 \end{tabular}

\subsection{Mani monocolore}
(6-3-2-2, 6-3-3-1, 7-3-2-1, \dots)\\

\begin{tabular}{|p{0.05\textwidth} p{0.05\textwidth} p{0.7\textwidth}|}
 \hline

 12+ & 1 $\bigstar$ & $\bigstar$ \`e il colore lungo.\\
\hline
\end{tabular}



\end{document}
